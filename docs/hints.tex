% -----------------------------------------------------------------------
% hints.tex: Section giving some tips & hints on how Duchamp is best
%            used. 
% -----------------------------------------------------------------------
% Copyright (C) 2006, Matthew Whiting, ATNF
%
% This program is free software; you can redistribute it and/or modify it
% under the terms of the GNU General Public License as published by the
% Free Software Foundation; either version 2 of the License, or (at your
% option) any later version.
%
% Duchamp is distributed in the hope that it will be useful, but WITHOUT
% ANY WARRANTY; without even the implied warranty of MERCHANTABILITY or
% FITNESS FOR A PARTICULAR PURPOSE.  See the GNU General Public License
% for more details.
%
% You should have received a copy of the GNU General Public License
% along with Duchamp; if not, write to the Free Software Foundation,
% Inc., 59 Temple Place, Suite 330, Boston, MA 02111-1307, USA
%
% Correspondence concerning Duchamp may be directed to:
%    Internet email: Matthew.Whiting [at] atnf.csiro.au
%    Postal address: Dr. Matthew Whiting
%                    Australia Telescope National Facility, CSIRO
%                    PO Box 76
%                    Epping NSW 1710
%                    AUSTRALIA
% -----------------------------------------------------------------------
\secA{Notes and hints on the use of \duchamp}
\label{sec-notes}

In using \duchamp, the user has to make a number of decisions about
the way the program runs. This section is designed to give the user
some idea about what to choose.

\secB{Memory usage}

A lot of attention has been paid to the memory usage in \duchamp,
recognising that data cubes are going to be increasing in size with
new generation correlators and wider fields of view. However, users
with large cubes should be aware of the likely usage for different
modes of operation and plan their \duchamp execution carefully. 

At the start of the program, memory is allocated sufficient for:
\begin{itemize}
\item The entire pixel array (as requested, subject to any
subsection).
\item The spatial extent, which holds the map of detected pixels (for
output into the detection map).
\item If smoothing or reconstruction has been selected, another array
of the same size as the pixel array. This will hold the
smoothed/reconstructed array (the original needs to be kept to do the
correct parameterisation of detected sources).
\item If baseline-subtraction has been selected, a further array of
the same size as the pixel array. This holds the baseline values,
which need to be added back in prior to parameterisation.
\end{itemize}
All of these will be float type, except for the detection map, which
is short.

There will, of course, be additional allocation during the course of
the program. The detection list will progressively grow, with each
detection having a memory footprint as described in
\S\ref{sec-scan}. But perhaps more important and with a larger
impact will be the temporary space allocated for various algorithms.

The largest of these will be the wavelet reconstruction. This will
require an additional allocation of twice the size of the array being
reconstructed, one for the coefficients and one for the wavelets -
each scale will overwrite the previous one. So, for the 1D case, this
means an additional allocation of twice the spectral dimension (since
we only reconstruct one spectrum at a time), but the 3D case will
require an additional allocation of twice the cube size (this means
there needs to be available at least four times the size of the input
cube for 3D reconstruction, plus the additional overheads of
detections and so forth).

The smoothing has less of an impact, since it only operates on the
lower dimensions, but it will make an additional allocation of twice
the relevant size (spectral dimension for spectral smoothing, or
spatial image size for the spatial Gaussian smoothing).

The other large allocation of temporary space will be for calculating
robust statistics. The median-based calculations require at least
partial sorting of the data, and so cannot be done on the original
image cube. This is done for the entire cube and so the temporary
memory increase can be large.


\secB{Timing considerations}

Another intersting question from a user's perspective is how long you
can expect \duchamp to take. This is a difficult question to answer in
general, as different users will have different sized data sets, as
well as machines with different capabilities (in terms of the CPU
speed and I/O \& memory bandwidths). Additionally, the time required
will depend slightly on the number of sources found and their size
(very large sources can take a while to fully parameterise). 

Having said that, in \citet{whiting12} a brief analysis was done
looking at different modes of execution applied to a single HIPASS
cube (\#201) using a MacBook Pro (2.66GHz, 8MB RAM). Two sets of
thresholds were used, either $10^8$~Jy~beam$^{-1}$ (no sources will be
found, so that the time taken is dominated by preprocessing), or
35~mJy~beam$^{-1}$ (or $\sim2.58\sigma$, chosen so that the time taken
will include that required to process sources).  The basic searches,
with no pre-processing done, took less than a second for the
high-threshold search, but between 1 and 3~min for the low-threshold
case -- the numbers of sources detected ranged from 3000 (rejecting
sources with less than 3 channels and 2 spatial pixels) to 42000
(keeping all sources).

When smoothing, the raw time for the spectral smoothing was only a few
seconds, with a small dependence on the width of the smoothing
filter. And because the number of spurious sources is markedly
decreased (the final catalogues ranged from 17 to 174 sources,
depending on the width of the smoothing), searching with the low
threshold did not add much more than a second to the time. The spatial
smoothing was more computationally intensive, taking about 4 minutes
to complete the high-threshold search.

The wavelet reconstruction time primarily depended on the
dimensionality of the reconstruction, with the 1D taking 20~s, the 2D
taking 30 - 40~s and the 3D taking 2 - 4~min. The spread in times for
a given dimensionality was caused by different reconstruction
thresholds, with lower thresholds taking longer (since more pixels are
above the threshold and so need to be added to the final spectrum). In
all cases the reconstruction time dominated the total time for the
low-threshold search, since the number of sources found was again
smaller than the basic searches.


\secB{Why do preprocessing?}

The preprocessing options provided by \duchamp, particularly the
ability to smooth or reconstruct via multi-resolution wavelet
decomposition, provide an opportunity to beat the effects of the
random noise that will be present in the data. This noise will
ultimately limit ones ability to detect objects and form a complete
and reliable catalogue. Two effects are important here. First, the
noise reduces the completeness of the final catalogue by suppressing
the flux of real sources such that they fall below the detection
threshold. Secondly, the noise provides false positive detections
through noise peaks that fall above the threshold, thereby reducing
the reliability of the catalogue.

\citet{whiting12} examined the effect on completeness and reliability
for the reconstruction and smoothing (1D cases only) when applied to a
simple simulated dataset. Both had the effect of reducing the number
of spurious sources, which means the searches can be done to fainter
thresholds. This led to completeness levels of about one flux unit
(equal to one standard-deviation of the noise) fainter than searches
without pre-processing, with $>95\%$ reliability. The smoothing did
slightly better, with the completeness level nearly half a flux unit
fainter than the reconstruction, although this was helped by the
sources in the simulation all having the same spectral size.

\secB{Reconstruction considerations}

The \atrous wavelet reconstruction approach is designed to remove a
large amount of random noise while preserving as much structure as
possible on the full range of spatial and/or spectral scales present
in the data. While it is relatively more expensive in terms of memory
and CPU usage (see previous sections), its effect on, in particular,
the reliability of the final catalogue makes it worth investigating.

There are, however, a number of subtleties to it that need to be
considered by potential users. \citet{whiting12} shows a set of
examples of reconstruction applied to simulated and real data. The
real data, in this case a HIPASS cube, shows differences in the
quality of the reconstructed spectrum depending on the dimensionality
of the reconstruction. The two-dimensional reconstruction (where the
cube is reconstructed one channel map at a time) shows much larger
channel-to-channel noise, with a number of narrow peaks surviving the
reconstruction process. The problem here is that there are spatial
correlations between pixels due to the beam, which allow beam-sized
noise fluctuations to rise above the threshold more frequently in
one-dimension. The other effect is that when compared to a spectrum
from the 1D reconstruction, each channel is independently
reconstructed, and does not depend on its neighbouring channels. This
is also why the 3D reconstruction (which also suffers from the beam
effects) has improved noise in the output spectrum, since the
information on neighbouring channels is taken into account.

Caution is also advised when looking at subsections of a cube. Due to
the multi-scale nature of the algorithm, the wavelet coefficients at a
given pixel are influenced by pixels at very large separations,
particularly given that edges are dealt with by assuming reflection
(so the whole array is visible to all pixels). Also, if one decreases
the dimensions of the array being reconstructed, there may be fewer
scales used in the reconstruction. These points mean that the
reconstruction of a subsection of a cube will differ from the same
subsection of the reconstructed cube. The difference may be small
(depending on the relative size difference and the amount of structure
at large scales), but there will be differences at some level.

Note also that BLANK pixels have no effect on the reconstruction: they
remain as BLANK in the output, and do not contribute to the discrete
convolution when they otherwise would. Flagging channels with the
\texttt{flaggedChannels} parameter, however, has no effect on the
reconstruction -- this are applied after the preprocessing, either in
the searching or the rejection stage.

\secB{Smoothing considerations}

The smoothing approach differs from the wavelet reconstruction in that
it has a single scale associated with it. The user has two choices to
make - which dimension to smooth in (spatially or spectrally), and
what size kernel to smooth with. \citet{whiting12} show examples of
how different smoothing widths (in one-dimension in this case) can
highlight sources of different sizes. If one has some \textit{a
  priori} idea of the typical size scale of objects one wishes to
detect, then choosing a single smoothing scale can be quite
beneficial.

Note also that beam effects can be important here too, when smoothing
spatial data on scales close to that of the beam. This can enhance
beam-sized noise fluctuations and potentially introduce spurious
sources. As always, examining the smoothed array (after saving via
\texttt{flagOutputSmooth}) is a good idea.


\secB{Threshold method}

When it comes to searching, the FDR method produces more reliable
results than simple sigma-clipping, particularly in the absence of
reconstruction.  However, it does not work in exactly the way one
would expect for a given value of \texttt{alpha}. For instance,
setting fairly liberal values of \texttt{alpha} (say, 0.1) will often
lead to a much smaller fraction of false detections (\ie much less
than 10\%). This is the effect of the merging algorithms, that combine
the sources after the detection stage, and reject detections not
meeting the minimum pixel or channel requirements.  It is thus better
to aim for larger \texttt{alpha} values than those derived from a
straight conversion of the desired false detection rate.

If the FDR method is not used, caution is required when choosing the
S/N cutoff. Typical cubes have very large numbers of pixels, so even
an apparently large cutoff will still result in a not-insignificant
number of detections simply due to random fluctuations of the noise
background. For instance, a $4\sigma$ threshold on a cube of Gaussian
noise of size $100\times100\times1024$ will result in $\sim340$
single-pixel detections. This is where the minimum channel and pixel
requirements are important in rejecting spurious detections.


%%% Local Variables: 
%%% mode: latex
%%% TeX-master: "Guide"
%%% End: 
