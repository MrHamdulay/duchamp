\secA{Notes and hints on the use of \duchamp}
\label{sec-notes}

In using \duchamp, the user has to make a number of decisions about
the way the program runs. This section is designed to give the user
some idea about what to choose.

The main choice is whether to alter the cube to try and enhance the
detectability of objects, by either smoothing or reconstructing via
the \atrous method. The main benefits of both methods are the marked
reduction in the noise level, leading to regularly-shaped detections,
and good reliability for faint sources.

The main drawback with the \atrous method is the long execution time:
to reconstruct a $170\times160\times1024$ (\hipass) cube often
requires three iterations and takes about 20-25 minutes to run
completely. Note that this is for the more complete three-dimensional
reconstruction: using \texttt{reconDim = 1} makes the reconstruction
quicker (the full program then takes less than 5 minutes), but it is
still the largest part of the time.

The smoothing procedure is computationally simpler, and thus quicker,
than the reconstruction. The spectral Hanning method adds only a very
small overhead on the execution, and the spatial Gaussian method,
while taking longer, will be done (for the above example) in less than
2 minutes. Note that these times will depend on the size of the
filter/kernel used: a larger filter means more calculations.

The searching part of the procedure is much quicker: searching an
un-reconstructed cube leads to execution times of less than a
minute. Alternatively, using the ability to read in previously-saved
reconstructed arrays makes running the reconstruction more than once a
more feasible prospect.

On the positive side, the shape of the detections in a cube that has
been reconstructed or smoothed will be much more regular and smooth --
the ragged edges that objects in the raw cube possess are smoothed by
the removal of most of the noise. This enables better determination of
the shapes and characteristics of objects.

While the time overhead is larger for the reconstruction case, it will
potentially provide a better recovery of real sources than the
smoothing case. This is because it probes the full range of scales
present in the cube (or spectral domain), rather than the specific
scale determined by the Hanning filter or Gaussian kernel used in the
smoothing. 

When considering the reconstruction method, note that the 2D
reconstruction (\texttt{reconDim = 2}) can be susceptible to edge
effects. If the valid area in the cube (\ie the part that is not
BLANK) has non-rectangular edges, the convolution can produce
artefacts in the reconstruction that mimic the edges and can lead
(depending on the selection threshold) to some spurious
sources. Caution is advised with such data -- the user is advised to
check carefully the reconstructed cube for the presence of such
artefacts. Note, however, that the 1- and 3-dimensional
reconstructions are \emph{not} susceptible in the same way, since the
spectral direction does not generally exhibit these BLANK edges, and
so we recommend the use of either of these.

If one chooses the reconstruction method, a further decision is
required on the signal-to-noise cutoff used in determining acceptable
wavelet coefficients. A larger value will remove more noise from the
cube, at the expense of losing fainter sources, while a smaller value
will include more noise, which may produce spurious detections, but
will be more sensitive to faint sources. Values of less than about
$3\sigma$ tend to not reduce the noise a great deal and can lead to
many spurious sources (this depends, of course on the cube itself).

The smoothing options have less parameters to consider: basically just
the size of the smoothing function or kernel. Spectrally smoothing
with a Hanning filter of width 3 (the smallest possible) is very
efficient at removing spurious one-channel objects that may result
just from statistical fluctuations of the noise. One may want to use
larger widths or kernels of larger size to look for features of a
particular scale in the cube.

When it comes to searching, the FDR method produces more reliable
results than simple sigma-clipping, particularly in the absence of
reconstruction.  However, it does not work in exactly the way one
would expect for a given value of \texttt{alpha}. For instance,
setting fairly liberal values of \texttt{alpha} (say, 0.1) will often
lead to a much smaller fraction of false detections (\ie much less
than 10\%). This is the effect of the merging algorithms, that combine
the sources after the detection stage, and reject detections not
meeting the minimum pixel or channel requirements.  It is thus better
to aim for larger \texttt{alpha} values than those derived from a
straight conversion of the desired false detection rate.

If the FDR method is not used, caution is required when choosing the
S/N cutoff. Typical cubes have very large numbers of pixels, so even
an apparently large cutoff will still result in a not-insignificant
number of detections simply due to random fluctuations of the noise
background. For instance, a $4\sigma$ threshold on a cube of Gaussian
noise of size $100\times100\times1024$ will result in $\sim340$
detections. This is where the minimum channel and pixel requirements
are important in rejecting spurious detections.

Finally, as \duchamp is still undergoing development, there are some
elements that are not fully developed. In particular, it is not as
clever as I would like at avoiding interference. The ability to place
requirements on the minimum number of channels and pixels partially
circumvents this problem, but work is being done to make \duchamp
smarter at rejecting signals that are clearly (to a human eye at
least) interference. See the following section for further
improvements that are planned.
