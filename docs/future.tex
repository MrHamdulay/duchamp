\secA{Future developments}

Here are lists of planned improvements and a wish-list of
features that would be nice to include (but are not planned in the
immediate future). Let me know if there are items not on these lists,
or items on the list you would like prioritised.

Planned developments:
\begin{itemize}
\item Parallelisation of the code, to improve speed particularly on
multi-core machines.

\item Better determination of the noise characteristics of
  spectral-line cubes, including understanding how the noise is
  generated and developing a model for it. 
  
\item Include more source analysis. Examples could be: shape
  information; measurements of HI mass; more variety of measurements
  of velocity width and profile. 

\item Improved ability to reject interference, possibly on the
  spectral shape of features.

\item Ability to separate (de-blend) distinct sources that have been
  merged.
\end{itemize}

Wish-list:
\begin{itemize}
\item Incorporation of Swinburne's S2PLOT
\footnote{\href{http://astronomy.swin.edu.au/s2plot/}
{http://astronomy.swin.edu.au/s2plot/}} code for improved
visualisation. 
\item Link to lists of possible counterparts (\eg via NED/SIMBAD/other
  VO tools?). 

\item On-line web service interface, so a user can upload a cube and
  get back a source-list.

\item Embed \duchamp\ in a GUI, to move away from the text-based
  interaction.
\end{itemize}

