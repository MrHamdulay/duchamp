\secA{Outputs}
\label{sec-output}

\secB{During execution}

\duchamp\ provides the user with feedback whilst it is running, to
keep the user informed on the progress of the analysis. Most of this
consists of self-explanatory messages about the particular stage the
program is up to. The relevant parameters are printed to the screen at
the start (once the file has been successfully read in), so the user
is able to make a quick check that the setup is correct (see
Appendix~{app-input} for an example).

If the cube is being trimmed (\S\ref{sec-modify}), the resulting
dimensions are printed to indicate how much has been trimmed. If a
reconstruction is being done, a continually updating message shows
either the current iteration and scale, compared to the maximum scale
(when \texttt{reconDim=3}), or a progress bar showing the amount of
the cube that has been reconstructed (for smaller values of
\texttt{reconDim}).

During the searching algorithms, the progress through the 1D and 2D
searches are shown. When the searches have completed, the number of
objects found in both the 1D and 2D searches are reported (see
\S\ref{sec-detection} for details).

In the merging process (where multiple detections of the same object
are combined -- see \S\ref{sec-merger}), two stages of output
occur. The first is when each object in the list is compared with all
others. The output shows two numbers: the first being how far through
the list the current object is, and the second being the length of the
list. As the algorithm proceeds, the first number should increase and
the second should decrease (as objects are combined). When the numbers
meet (\ie the whole list has been compared), the second phase begins,
in which multiply-appearing pixels in each object are removed, as are
objects not meeting the minimum channels requirement. During this
phase, the total number of accepted objects is shown, which should
steadily increase until all have been accepted or rejected. Note that
these steps can be very quick for small numbers of detections.

Since this continual printing to screen has some overhead of time and
CPU involved, the user can elect to not print this information by
setting the parameter \texttt{verbose = 0}. In this case, the user is
still informed as to the steps being undertaken, but the details of
the progress are not shown.

There may also be Warning or Error messages printed to screen. The
Warning messages occur when something happens that is unexpected (for
instance, a desired keyword is not present in the FITS header), but
not detrimental to the execution. An Error message is something more
serious, and indicates some part of the program was not able to
complete its task. The message will also indicate which function or
subroutine generated it -- this is primarily a tool for debugging, but
can be useful in determining what went wrong.

\secB{Results}

\secC{Table of results}

Finally, we get to the results -- the reason for running \duchamp\ in
the first place. Once the detection list is finalised, it is sorted by
the mean velocity of the detections (or, if there is no good WCS
associated with the cube, by the mean $z$-pixel position). The results
are then printed to the screen and to the output file, given by the
\texttt{OutFile} parameter. 

The output consists of three parts. First, a list of the parameters
are printed to the output file, for future reference. Next, the
detection level that was used is given, so comparison can be made with
other searches. The noise level and its spread are also reported.

The most interesting part, however, is the list of detected
objects. This list, an example of which can be seen in
Appendix~\ref{app-output}, contains the following columns (note that
the title of the columns depending on WCS information will depend on
the details of the WCS projection: they are shown below for the
Equatorial and Galactic projection cases).

\begin{entry}
\item[Obj\#] The ID number of the detection (simply the sequential
  count for the list, which is ordered by increasing velocity, or
  channel number, if the WCS is not good enough to find the velocity).
\item[Name] The ``IAU''-format name of the detection (derived from the
  WCS position -- see below for a description of the format).
\item[X] The average X-pixel position (averaged over all detected
voxels).
\item[Y] The average Y-pixel position.
\item[Z] The average Z-pixel position.
\item[RA/GLON] The Right Ascension or Galactic Longitude of the centre
of the object.
\item[DEC/GLAT] The Declination or Galactic Latitude of the centre of
the object.
\item[VEL] The mean velocity of the object [units given by the
  \texttt{spectralUnits} parameter].
\item[w\_RA/w\_GLON] The width of the object in Right Ascension or
Galactic Longitude [arcmin].
\item[w\_DEC/w\_GLAT] The width of the object in Declination Galactic
  Latitude [arcmin].
\item[w\_VEL] The full velocity width of the detection (max channel
  $-$ min channel, in velocity units [see note below]).
\item[F\_int] The integrated flux over the object, in the units of
  flux times velocity, corrected for the beam if necessary.
\item[F\_peak] The peak flux over the object, in the units of flux.
\item[S/Nmax] The signal-to-noise ratio at the peak pixel.
\item[X1, X2] The minimum and maximum X-pixel coordinates.
\item[Y1, Y2] The minimum and maximum Y-pixel coordinates.
\item[Z1, Z2] The minimum and maximum Z-pixel coordinates.
\item[Npix] The number of voxels (\ie distinct $(x,y,z)$ coordinates)
  in the detection.
\item[Flag] Whether the detection has any warning flags (see below).
\end{entry}

The Name is derived from the WCS position. For instance, a source that
is centred on the RA,Dec position 12$^h$53$^m$45$^s$,
-36$^\circ$24$'$12$''$ will be given the name J125345$-$362412, if the
epoch is J2000, or the name B125345$-$362412 if it is B1950. An
alternative form is used for Galactic coordinates: a source centred on
the position ($l$,$b$) = (323.1245, 5.4567) will be called
G323.124$+$05.457. If the WCS is not valid (\ie is not present or does
not have all the necessary information), the Name, RA, DEC, VEL and
related columns are not printed, but the pixel coordinates are still
provided.

The velocity units can be specified by the user, using the parameter
\texttt{spectralUnits} (enter it as a single string). The default
value is km/s, which should be suitable for most users. These units
are also used to give the units of integrated flux. Note that if there
is no rest frequency specified in the FITS header, the \duchamp\
output will instead default to using Frequency, with units of MHz.

If the WCS is absent or not sufficiently specified, then all columns
from RA/GLON to w\_VEL will be left blank. Also, F\_int will be
replaced with the more simple F\_tot -- the total flux in the
detection, being the sum of all detected voxels.

%The last column contains any warning flags about the detection. There
%are currently three options here. An `E' is printed if the detection is
%next to the edge of the image, meaning either the limit of the pixels,
%or the limit of the non-BLANK pixel region. An `S' is printed if the
%detection lies at the edge of the spectral region. An `N' is printed
%if the total flux, summed over all the (non-BLANK) pixels in the
%smallest box that completely encloses the detection, is negative. Note
%that this sum is likely to include non-detected pixels. It is of use
%in pointing out detections that lie next to strongly negative pixels,
%such as might arise due to interference -- the detected pixels might
%then also be due to the interference, so caution is advised.

The last column contains any warning flags about the detection, such
as: 
\begin{itemize}
\item \textbf{E} -- The detection is next to the spatial edge of the image,
meaning either the limit of the pixels, or the limit of the non-BLANK
pixel region.
\item \textbf{S} -- The detection lies at the edge of the spectral region. 
\item \textbf{N} -- The total flux, summed over all the (non-BLANK)
pixels in the smallest box that completely encloses the detection, is
negative. Note that this sum is likely to include non-detected
pixels. It is of use in pointing out detections that lie next to
strongly negative pixels, such as might arise due to interference --
the detected pixels might then also be due to the interference, so
caution is advised.
\end{itemize}

\secC{Other results lists}

Two additional results files can also be requested. One option is a
VOTable-format XML file, containing just the RA, Dec, Velocity and the
corresponding widths of the detections, as well as the fluxes. The
user should set \texttt{flagVOT = 1}, and put the desired filename in
the parameter \texttt{votFile} -- note that the default is for it not
to be produced. This file should be compatible with all Virtual
Observatory tools (such as Aladin\footnote{ Aladin can be found on the
web at
\href{http://aladin.u-strasbg.fr/}{http://aladin.u-strasbg.fr/}}). The
second option is an annotation file for use with the Karma toolkit of
visualisation tools (in particular, with \texttt{kvis}). This will
draw a circle at the position of each detection, scaled by the spatial
size of the detection, and number it according to the Obj\# given
above. To make use of this option, the user should set
\texttt{flagKarma = 1}, and put the desired filename in the parameter
\texttt{karmaFile} -- again, the default is for it not to be produced.

As the program is running, it also (optionally) records the detections
made in each individual spectrum or channel (see \S\ref{sec-detection}
for details on this process). This is recorded in the file given by
the parameter \texttt{LogFile}. This file does not include the columns
\texttt{Name, RA, DEC, w\_RA, w\_DEC, VEL, w\_VEL}. This file is
designed primarily for diagnostic purposes: \eg to see if a given set
of pixels is detected in, say, one channel image, but does not survive
the merging process. The list of pixels (and their fluxes) in the
final detection list are also printed to this file, again for
diagnostic purposes. The file also records the execution time, as well
as the command-line statement used to run \duchamp. The creation of
this log file can be prevented by setting \texttt{flagLog = false}.

\secC{Graphical output -- spectra}

\begin{figure}[t]
\begin{center}
\includegraphics[width=\textwidth]{example_spectrum}
\end{center}
\caption{\footnotesize An example of the spectrum output. Note several
  of the features discussed in the text: the red lines indicating the
  reconstructed spectrum; the blue dashed lines indicating the
  spectral extent of the detection; the green hashed area indicating
  the Milky Way channels that are ignored by the searching algorithm;
  the blue border showing its spatial extent on the 0th moment map;
  and the 15~arcmin-long scale bar.}
\label{fig-spect}
\end{figure}

\begin{figure}[!t]
\begin{center}
\includegraphics[width=\textwidth]{example_moment_map}
\end{center}
\caption{\footnotesize An example of the moment map created by
  \duchamp. The full extent of the cube is covered, and the 0th moment
  of each object is shown (integrated individually over all the
  detected channels). The purple line indicates the limit of the
  non-BLANK pixels.}
\label{fig-moment}
\end{figure}

As well as the output data file, a postscript file is created that
shows the spectrum for each detection, together with a small cutout
image (the 0th moment) and basic information about the detection (note
that any flags are printed after the name of the detection, in the
format \texttt{[E]}). If the cube was reconstructed, the spectrum from
the reconstruction is shown in red, over the top of the original
spectrum. The spectral extent of the detected object is indicated by
two dashed blue lines, and the region covered by the ``Milky Way''
channels is shown by a green hashed box. An example detection can be
seen below in Fig.~\ref{fig-spect}.

The spectrum that is plotted is governed by the
\texttt{spectralMethod} parameter. It can be either \texttt{peak} (the
default), where the spectrum is from the spatial pixel containing the
detection's peak flux; or \texttt{sum}, where the spectrum is summed
over all spatial pixels, and then corrected for the beam size.  The
spectral extent of the detection is indicated with blue lines, and a
zoom is shown in a separate window.

The cutout image can optionally include a border around the spatial
pixels that are in the detection (turned on and off by the
\texttt{drawBorders} parameter -- the default is \texttt{true}). It
includes a scale bar in the bottom left corner to indicate size -- its
length is indicated next to it (the choice of length depends on the
size of the image).

There may also be one or two extra lines on the image. A yellow line
shows the limits of the cube's spatial region: when this is shown, the
detected object will lie close to the edge, and the image box will
extend outside the region covered by the data. A purple line, however,
shows the dividing line between BLANK and non-BLANK pixels. The BLANK
pixels will always be shown in black. The first type of line is always
drawn, while the second is governed by the parameter
\texttt{drawBlankEdges} (whose default is \texttt{true}), and
obviously whether there are any BLANK pixel present.

\secC{Graphical output -- maps}

Finally, a couple of images are optionally produced: a 0th moment map
of the cube, combining just the detected channels in each object,
showing the integrated flux in grey-scale; and a ``detection image'',
a grey-scale image where the pixel values are the number of channels
that spatial pixel is detected in. In both cases, if
\texttt{drawBorders = true}, a border is drawn around the spatial
extent of each detection, and if \texttt{drawBlankEdges = true}, the
purple line dividing BLANK and non-BLANK pixels (as described above)
is drawn. An example moment map is shown in Fig.~\ref{fig-moment}.
The production or otherwise of these images is governed by the
\texttt{flagMaps} parameter.

The moment map is also displayed in a PGPlot XWindow. This feature can
be turned off by setting the \texttt{flagXOutput} parameter to
\texttt{false} -- this might be useful if running \duchamp\ on a
terminal with no window display capability, or if you have set up a
script to run it in a batch mode.

The purpose of these images are to provide a visual guide to where the
detections have been made, and, particularly in the case of the moment
map, to provide an indication of the strength of the source. In both
cases, the detections are numbered (in the same sense as the output
list and as the spectral plots), and the spatial borders are marked
out as for the cutout images in the spectra file. Both these images
are saved as postscript files (given by the parameters
\texttt{momentMap} and \texttt{detectionMap} respectively), with the
latter also displayed in a \textsc{pgplot} window (regardless of the
state of \texttt{flagMaps}).
