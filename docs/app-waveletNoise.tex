\secA{How Gaussian noise changes with wavelet scale}
\label{app-scaling}

The key element in the wavelet reconstruction of an array is the
thresholding of the individual wavelet coefficient arrays. This is
usually done by choosing a level to be some number of standard
deviations above the mean value.

However, since the wavelet arrays are produced by convolving the input
array by an increasingly large filter, the pixels in the coefficient
arrays become increasingly correlated as the scale of the filter
increases. This results in the measured standard deviation from a
given coefficient array decreasing with increasing scale. To calculate
this, we need to take into account how many other pixels each pixel in
the convolved array depends on.

To demonstrate, suppose we have a 1-D array with $N$ pixel values
given by $F_i,\ i=1,...,N$, and we convolve it with the B$_3$-spline
filter, defined by the set of coefficients
$\{1/16,1/4,3/8,1/4,1/16\}$. The flux of the $i$th pixel in the
convolved array will be
\[
F'_i = \frac{1}{16}F_{i-2} + \frac{1}{4}F_{i-1} + \frac{3}{8}F_{i}
+ \frac{1}{4}F_{i+1} + \frac{1}{16}F_{i+2}
\]
and the flux of the corresponding pixel in the wavelet array will be 
\[
W'_i = F_i - F'_i = \frac{-1}{16}F_{i-2} - \frac{1}{4}F_{i-1} 
+ \frac{5}{8}F_{i} - \frac{1}{4}F_{i+1} - \frac{1}{16}F_{i+2}
\]
Now, assuming each pixel has the same standard deviation
$\sigma_i=\sigma$, we can work out the standard deviation for the
wavelet array:
\[
\sigma'_i = \sigma \sqrt{\left(\frac{1}{16}\right)^2 
  + \left(\frac{1}{4}\right)^2 + \left(\frac{5}{8}\right)^2 
  + \left(\frac{1}{4}\right)^2 + \left(\frac{1}{16}\right)^2}
          = 0.72349\ \sigma
\]
Thus, the first scale wavelet coefficient array will have a standard
deviation of 72.3\% of the input array. This procedure can be followed
to calculate the necessary values for all scales, dimensions and
filters used by \duchamp.

Calculating these values is clearly a critical step in performing the
reconstruction. The method used by \citet{starck02:book} was to
simulate data sets with Gaussian noise, take the wavelet transform,
and measure the value of $\sigma$ for each scale. We take a different
approach, by calculating the scaling factors directly from the filter
coefficients by taking the wavelet transform of an array made up of a
1 in the central pixel and 0s everywhere else. The scaling value is
then derived by taking the square root of the sum (in quadrature) of
all the wavelet coefficient values at each scale. We give the scaling
factors for the three filters available to \duchamp below. These
values are hard-coded into \duchamp, so no on-the-fly calculation of
them is necessary.

Memory limitations prevent us from calculating factors for large
scales, particularly for the three-dimensional case (hence the --
symbols in the tables). To calculate factors for higher scales than
those available, we divide the previous scale's factor by either
$\sqrt{2}$, $2$, or $\sqrt{8}$ for 1D, 2D and 3D respectively. 

%\newpage
{\small
\begin{itemize}
\item \textbf{B$_3$-Spline Function:} $\{1/16,1/4,3/8,1/4,1/16\}$

\begin{tabular}{llll}
Scale & 1 dimension      & 2 dimension     & 3 dimension\\ \hline
1     & 0.723489806      & 0.890796310     & 0.956543592\\
2     & 0.285450405	 & 0.200663851	   & 0.120336499\\
3     & 0.177947535	 & 0.0855075048	   & 0.0349500154\\
4     & 0.122223156	 & 0.0412474444	   & 0.0118164242\\
5     & 0.0858113122	 & 0.0204249666	   & 0.00413233507\\
6     & 0.0605703043	 & 0.0101897592	   & 0.00145703714\\
7     & 0.0428107206	 & 0.00509204670   & 0.000514791120\\
8     & 0.0302684024	 & 0.00254566946   & --\\
9     & 0.0214024008	 & 0.00127279050   & --\\
10    & 0.0151336781	 & 0.000636389722  & --\\
11    & 0.0107011079	 & 0.000318194170  & --\\
12    & 0.00756682272	 & --		   & --\\
13    & 0.00535055108	 & --		   & --\\
%14    & 0.00378341085	 & --		   & --\\
%15    & 0.00267527545	 & --		   & --\\
%16    & 0.00189170541	 & --		   & --\\
%17    & 0.00133763772	 & --		   & --\\
%18    & 0.000945852704   & --		   & --
\end{tabular}

\item \textbf{Triangle Function:} $\{1/4,1/2,1/4\}$

\begin{tabular}{llll}
Scale & 1 dimension      & 2 dimension     & 3 dimension\\ \hline
1     & 0.612372436      & 0.800390530     & 0.895954449  \\
2     & 0.330718914	 & 0.272878894     & 0.192033014\\
3     & 0.211947812	 & 0.119779282     & 0.0576484078\\
4     & 0.145740298	 & 0.0577664785    & 0.0194912393\\
5     & 0.102310944	 & 0.0286163283    & 0.00681278387\\
6     & 0.0722128185	 & 0.0142747506    & 0.00240175885\\
7     & 0.0510388224	 & 0.00713319703   & 0.000848538128 \\
8     & 0.0360857673	 & 0.00356607618   & 0.000299949455 \\
9     & 0.0255157615	 & 0.00178297280   & -- \\
10    & 0.0180422389	 & 0.000891478237  & --  \\
11    & 0.0127577667	 & 0.000445738098  & --  \\
12    & 0.00902109930	 & 0.000222868922  & --  \\
13    & 0.00637887978	 & --		   & -- \\
%14   & 0.00451054902	 & --		   & -- \\
%15   & 0.00318942978	 & --		   & -- \\
%16   & 0.00225527449	 & --		   & -- \\
%17   & 0.00159471988	 & --		   & -- \\
%18   & 0.000112763724	 & --		   & -- 

\end{tabular}

\item \textbf{Haar Wavelet:} $\{0,1/2,1/2\}$

\begin{tabular}{llll}
Scale & 1 dimension      & 2 dimension     & 3 dimension\\ \hline
1     & 0.707167810      & 0.433012702     & 0.935414347 \\
2     & 0.500000000	 & 0.216506351     & 0.330718914\\
3     & 0.353553391	 & 0.108253175     & 0.116926793\\
4     & 0.250000000	 & 0.0541265877    & 0.0413398642\\
5     & 0.176776695	 & 0.0270632939    & 0.0146158492\\
6     & 0.125000000	 & 0.0135316469    & 0.00516748303

\end{tabular}

\end{itemize}
}
