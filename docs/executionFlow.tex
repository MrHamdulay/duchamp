% -----------------------------------------------------------------------
% executionFlow.tex: Section detailing each of the main algorithms
%                    used by Duchamp.
% -----------------------------------------------------------------------
% Copyright (C) 2006, Matthew Whiting, ATNF
%
% This program is free software; you can redistribute it and/or modify it
% under the terms of the GNU General Public License as published by the
% Free Software Foundation; either version 2 of the License, or (at your
% option) any later version.
%
% Duchamp is distributed in the hope that it will be useful, but WITHOUT
% ANY WARRANTY; without even the implied warranty of MERCHANTABILITY or
% FITNESS FOR A PARTICULAR PURPOSE.  See the GNU General Public License
% for more details.
%
% You should have received a copy of the GNU General Public License
% along with Duchamp; if not, write to the Free Software Foundation,
% Inc., 59 Temple Place, Suite 330, Boston, MA 02111-1307, USA
%
% Correspondence concerning Duchamp may be directed to:
%    Internet email: Matthew.Whiting [at] atnf.csiro.au
%    Postal address: Dr. Matthew Whiting
%                    Australia Telescope National Facility, CSIRO
%                    PO Box 76
%                    Epping NSW 1710
%                    AUSTRALIA
% -----------------------------------------------------------------------
\secA{What \duchamp is doing}
\label{sec-flow}

Each of the steps that \duchamp goes through in the course of its
execution are discussed here in more detail. This should provide
enough background information to fully understand what \duchamp is
doing and what all the output information is. For those interested in
the programming side of things, \duchamp is written in C/C++ and makes
use of the \textsc{cfitsio}, \textsc{wcslib} and \textsc{pgplot}
libraries.

\secB{Image input}
\label{sec-input}

The cube is read in using basic \textsc{cfitsio} commands, and stored
as an array in a special C++ class. This class keeps track of the list
of detected objects, as well as any reconstructed arrays that are made
(see \S\ref{sec-recon}). The World Coordinate System
(WCS)\footnote{This is the information necessary for translating the
  pixel locations to quantities such as position on the sky,
  frequency, velocity, and so on.} information for the cube is also
obtained from the FITS header by \textsc{wcslib} functions
\citep{greisen02, calabretta02,greisen06}, and this information, in
the form of a \texttt{wcsprm} structure, is also stored in the same
class. See Sec.~\ref{sec-wcs} for more details.

A sub-section of a cube can be requested by defining the subsection
with the \texttt{subsection} parameter and setting
\texttt{flagSubsection = true} -- this can be a good idea if the cube
has very noisy edges, which may produce many spurious detections.

There are two ways of specifying the \texttt{subsection} string. The
first is the generalised form
\texttt{[x1:x2:dx,y1:y2:dy,z1:z2:dz,...]}, as used by the
\textsc{cfitsio} library. This has one set of colon-separated numbers
for each axis in the FITS file. In this manner, the x-coordinates run
from \texttt{x1} to \texttt{x2} (inclusive), with steps of
\texttt{dx}. The step value can be omitted, so a subsection of the
form \texttt{[2:50,2:50,10:1000]} is still valid. In fact, \duchamp
does not make use of any step value present in the subsection string,
and any that are present are removed before the file is opened.

If the entire range of a coordinate is required, one can replace the
range with a single asterisk, \eg \texttt{[2:50,2:50,*]}. Thus, the
subsection string \texttt{[*,*,*]} is simply the entire cube. Note
that the pixel ranges for each axis start at 1, so the full pixel
range of a 100-pixel axis would be expressed as 1:100. A complete
description of this section syntax can be found at the
\textsc{fitsio} web site%
\footnote{%
\href%
{http://heasarc.gsfc.nasa.gov/docs/software/fitsio/c/c\_user/node91.html}%
{http://heasarc.gsfc.nasa.gov/docs/software/fitsio/c/c\_user/node91.html}}.


Making full use of the subsection requires knowledge of the size of
each of the dimensions. If one wants to, for instance, trim a certain
number of pixels off the edges of the cube, without examining the cube
to obtain the actual size, one can use the second form of the
subsection string. This just gives a number for each axis, \eg
\texttt{[5,5,5]} (which would trim 5 pixels from the start \emph{and}
end of each axis).

All types of subsections can be combined \eg \texttt{[5,2:98,*]}. 

Typically, the units of pixel brightness are given by the FITS file's
BUNIT keyword. However, this may often be unwieldy (for instance, the
units are Jy/beam, but the values are around a few mJy/beam). It is
therefore possible to nominate new units, to which the pixel values
will be converted, by using the \texttt{newFluxUnits} input
parameter. The units must be directly translatable from the existing
ones -- for instance, if BUNIT is Jy/beam, you cannot specify mJy, it
must be mJy/beam. If an incompatible unit is given, the BUNIT value is
used instead.

\secB{World Coordinate System}
\label{sec-wcs}

\duchamp uses the \textsc{wcslib} package to handle the conversions
between pixel and world coordinates. This package uses the
transformations described in the WCS papers
\citep{greisen02,calabretta02,greisen06}. The same package handles the
WCS axes in the spatial plots. The conversions used are governed by
the information in the FITS header -- this is parsed by
\textsc{wcslib} to create the appropriate transformations.

For the spectral axis, however, \duchamp provides the ability to change the
type of transformation used, so that different spectral quantities can
be calculated. By using the parameter \texttt{spectralType}, the user
can change from the type given in the FITS header. This should be done
in line with the conventions outlined in \citet{greisen06}. The
spectral type can be either a full 8-character string (\eg
'VELO-F2V'), or simply the 4-character ``S-type'' (\eg 'VELO'), in
which case \textsc{wcslib} will handle the conversion.

The rest frequency can be provided as well. This may be necessary, if
the FITS header does not specify one and you wish to transform to
velocity. Alternatively, you may want to make your measurements based
on a different spectral line (\eg OH1665 instead of
H\textsc{i}-21cm). The input parameter \texttt{restFrequency} is used,
and this will override the FITS header value.

Finally, the user may also request different spectral units from those
in the FITS file, or from the defaults arising from the
\textsc{wcslib} transformation. The input parameter
\texttt{spectralUnits} should be used, and \citet{greisen02} should be
consulted to ensure the syntax is appropriate.

\secB{Image modification}
\label{sec-modify}

Several modifications to the cube can be made that improve the
execution and efficiency of \duchamp (their use is optional, governed
by the relevant flags in the parameter file).

\secC{BLANK pixel removal}
\label{sec-blank}

If the imaged area of a cube is non-rectangular (see the example in
Fig.~\ref{fig-moment}, a cube from the HIPASS survey), BLANK pixels
are used to pad it out to a rectangular shape. The value of these
pixels is given by the FITS header keywords BLANK, BSCALE and
BZERO. While these pixels make the image a nice shape, they will take
up unnecessary space in memory, and so to potentially speed up the
processing we can trim them from the edge. This is done when the
parameter \texttt{flagTrim = true}. If the above keywords are not
present, the trimming will not be done (in this case, a similar effect
can be accomplished, if one knows where the ``blank'' pixels are, by
using the subsection option).

The amount of trimming is recorded, and these pixels are added back in
once the source-detection is completed (so that quoted pixel positions
are applicable to the original cube). Rows and columns are trimmed one
at a time until the first non-BLANK pixel is reached, so that the
image remains rectangular. In practice, this means that there will be
some BLANK pixels left in the trimmed image (if the non-BLANK region
is non-rectangular). However, these are ignored in all further
calculations done on the cube.

\secC{Baseline removal}

Second, the user may request the removal of baselines from the
spectra, via the parameter \texttt{flagBaseline}. This may be
necessary if there is a strong baseline ripple present, which can
result in spurious detections at the high points of the ripple. The
baseline is calculated from a wavelet reconstruction procedure (see
\S\ref{sec-recon}) that keeps only the two largest scales. This is
done separately for each spatial pixel (\ie for each spectrum in the
cube), and the baselines are stored and added back in before any
output is done. In this way the quoted fluxes and displayed spectra
are as one would see from the input cube itself -- even though the
detection (and reconstruction if applicable) is done on the
baseline-removed cube.

The presence of very strong signals (for instance, masers at several
hundred Jy) could affect the determination of the baseline, and would
lead to a large dip centred on the signal in the baseline-subtracted
spectrum. To prevent this, the signal is trimmed prior to the
reconstruction process at some standard threshold (at $8\sigma$ above
the mean). The baseline determined should thus be representative of
the true, signal-free baseline. Note that this trimming is only a
temporary measure which does not affect the source-detection.

\secC{Ignoring bright Milky Way emission}
\label{sec-MW}

Finally, a single set of contiguous channels can be ignored -- these
may exhibit very strong emission, such as that from the Milky Way as
seen in extragalactic \hi cubes (hence the references to ``Milky
Way'' in relation to this task -- apologies to Galactic
astronomers!). Such dominant channels will produce many detections
that are unnecessary, uninteresting (if one is interested in
extragalactic \hi) and large (in size and hence in memory usage), and
so will slow the program down and detract from the interesting
detections. 

The use of this feature is controlled by the \texttt{flagMW}
parameter, and the exact channels concerned are able to be set by the
user (using \texttt{maxMW} and \texttt{minMW} -- these give an
inclusive range of channels). When employed, these channels are
ignored for the searching, and the scaling of the spectral output (see
Fig.~\ref{fig-spect}) will not take them into account. They will be
present in the reconstructed array, however, and so will be included
in the saved FITS file (see \S\ref{sec-reconIO}). When the final
spectra are plotted, the range of channels covered by these parameters
is indicated by a green hashed box. Note that these channels refer to
channel numbers in the full cube, before any subsection is applied.

\secB{Image reconstruction}
\label{sec-recon}

The user can direct \duchamp to reconstruct the data cube using the
\atrous wavelet procedure. A good description of the procedure can be
found in \citet{starck02a}. The reconstruction is an effective way
of removing a lot of the noise in the image, allowing one to search
reliably to fainter levels, and reducing the number of spurious
detections. This is an optional step, but one that greatly enhances
the source-detection process, with the payoff that it can be
relatively time- and memory-intensive.

\secC{Algorithm}

The steps in the \atrous reconstruction are as follows:
\begin{enumerate}
\item The reconstructed array is set to 0 everywhere.
\item The input array is discretely convolved with a given filter
  function. This is determined from the parameter file via the
  \texttt{filterCode} parameter -- see Appendix~\ref{app-param} for
  details on the filters available. Edges are dealt with by assuming
  reflection at the boundary.
\item The wavelet coefficients are calculated by taking the difference
  between the convolved array and the input array.
\item If the wavelet coefficients at a given point are above the
  requested threshold (given by \texttt{snrRecon} as the number of
  $\sigma$ above the mean and adjusted to the current scale -- see
  Appendix~\ref{app-scaling}), add these to the reconstructed array.
\item The separation between the filter coefficients is doubled. (Note
  that this step provides the name of the procedure\footnote{\atrous
  means ``with holes'' in French.}, as gaps or holes are created in
  the filter coverage.)
\item The procedure is repeated from step 2, using the convolved array
  as the input array.
\item Continue until the required maximum number of scales is reached.
\item Add the final smoothed (\ie convolved) array to the
  reconstructed array. This provides the ``DC offset'', as each of the
  wavelet coefficient arrays will have zero mean.
\end{enumerate}

The range of scales at which the selection of wavelet coefficients is
made is governed by the \texttt{scaleMin} and \texttt{scaleMax}
parameters. The minimum scale used is given by \texttt{scaleMin},
where the default value is 1 (the first scale). This parameter is
useful if you want to ignore the highest-frequency features
(e.g. high-frequency noise that might be present). Normally the
maximum scale is calculated from the size of the input array, but it
can be specified by using \texttt{scaleMax}. A value $\le0$ will
result in the use of the calculated value, as will a value of
\texttt{scaleMax} greater than the calculated value. Use of these two
parameters can allow searching for features of a particular scale
size, for instance searching for narrow absorption features.

The reconstruction has at least two iterations. The first iteration
makes a first pass at the wavelet reconstruction (the process outlined
in the 8 stages above), but the residual array will likely have some
structure still in it, so the wavelet filtering is done on the
residual, and any significant wavelet terms are added to the final
reconstruction. This step is repeated until the change in the measured
standard deviation of the background (see note below on the evaluation
of this quantity) is less than some fiducial amount.

It is important to note that the \atrous decomposition is an example
of a ``redundant'' transformation. If no thresholding is performed,
the sum of all the wavelet coefficient arrays and the final smoothed
array is identical to the input array. The thresholding thus removes
only the unwanted structure in the array.

Note that any BLANK pixels that are still in the cube will not be
altered by the reconstruction -- they will be left as BLANK so that
the shape of the valid part of the cube is preserved.

\secC{Note on Statistics}

The correct calculation of the reconstructed array needs good
estimators of the underlying mean and standard deviation (or rms) of
the background noise distribution. The methods used to estimate these
quantities are detailed in \S\ref{sec-stats} -- the default behaviour
is to use robust estimators, to avoid biasing due to bright pixels.

%These statistics are estimated using
%robust methods, to avoid corruption by strong outlying points. The
%mean of the distribution is actually estimated by the median, while
%the median absolute deviation from the median (MADFM) is calculated
%and corrected assuming Gaussianity to estimate the underlying standard
%deviation $\sigma$. The Gaussianity (or Normality) assumption is
%critical, as the MADFM does not give the same value as the usual rms
%or standard deviation value -- for a Normal distribution
%$N(\mu,\sigma)$ we find MADFM$=0.6744888\sigma$, but this will change
%for different distributions. Since this ratio is corrected for, the
%user need only think in the usual multiples of the rms when setting
%\texttt{snrRecon}. See Appendix~\ref{app-madfm} for a derivation of
%this value.

When thresholding the different wavelet scales, the value of the rms
as measured from the wavelet array needs to be scaled to account for
the increased amount of correlation between neighbouring pixels (due
to the convolution). See Appendix~\ref{app-scaling} for details on
this scaling.

\secC{User control of reconstruction parameters}

The most important parameter for the user to select in relation to the
reconstruction is the threshold for each wavelet array. This is set
using the \texttt{snrRecon} parameter, and is given as a multiple of
the rms (estimated by the MADFM) above the mean (which for the wavelet
arrays should be approximately zero). There are several other
parameters that can be altered as well that affect the outcome of the
reconstruction.

By default, the cube is reconstructed in three dimensions, using a
3-dimensional filter and 3-dimensional convolution. This can be
altered, however, using the parameter \texttt{reconDim}. If set to 1,
this means the cube is reconstructed by considering each spectrum
separately, whereas \texttt{reconDim=2} will mean the cube is
reconstructed by doing each channel map separately. The merits of
these choices are discussed in \S\ref{sec-notes}, but it should be
noted that a 2-dimensional reconstruction can be susceptible to edge
effects if the spatial shape of the pixel array is not rectangular.

The user can also select the minimum scale to be used in the
reconstruction. The first scale exhibits the highest frequency
variations, and so ignoring this one can sometimes be beneficial in
removing excess noise. The default is to use all scales
(\texttt{minscale = 1}).

Finally, the filter that is used for the convolution can be selected
by using \texttt{filterCode} and the relevant code number -- the
choices are listed in Appendix~\ref{app-param}. A larger filter will
give a better reconstruction, but take longer and use more memory when
executing. When multi-dimensional reconstruction is selected, this
filter is used to construct a 2- or 3-dimensional equivalent.

\secB{Smoothing the cube}
\label{sec-smoothing}

An alternative to doing the wavelet reconstruction is to smooth the
cube.  This technique can be useful in reducing the noise level
slightly (at the cost of making neighbouring pixels correlated and
blurring any signal present), and is particularly well suited to the
case where a particular signal size (\ie a certain channel width or
spatial size) is believed to be present in the data.

There are two alternative methods that can be used: spectral
smoothing, using the Hanning filter; or spatial smoothing, using a 2D
Gaussian kernel. These alternatives are outlined below. To utilise the
smoothing option, set the parameter \texttt{flagSmooth=true} and set
\texttt{smoothType} to either \texttt{spectral} or \texttt{spatial}.

\secC{Spectral smoothing}

When \texttt{smoothType = spectral} is selected, the cube is smoothed
only in the spectral domain. Each spectrum is independently smoothed
by a Hanning filter, and then put back together to form the smoothed
cube, which is then used by the searching algorithm (see below). Note
that in the case of both the reconstruction and the smoothing options
being requested, the reconstruction will take precedence and the
smoothing will \emph{not} be done.

There is only one parameter necessary to define the degree of
smoothing -- the Hanning width $a$ (given by the user parameter
\texttt{hanningWidth}). The coefficients $c(x)$ of the Hanning filter
are defined by
\[
c(x) = 
  \begin{cases}
   \frac{1}{2}\left(1+\cos(\frac{\pi x}{a})\right) &|x| < (a+1)/2\\
   0                                               &|x| \geq (a+1)/2.
  \end{cases},\ a,x \in \mathbb{Z}
\]
Note that the width specified must be an
odd integer (if the parameter provided is even, it is incremented by
one).

\secC{Spatial smoothing}

When \texttt{smoothType = spatial} is selected, the cube is smoothed
only in the spatial domain. Each channel map is independently smoothed
by a two-dimensional Gaussian kernel, put back together to form the
smoothed cube, and used in the searching algorithm (see below). Again,
reconstruction is always done by preference if both techniques are
requested.

The two-dimensional Gaussian has three parameters to define it,
governed by the elliptical cross-sectional shape of the Gaussian
function: the FWHM (full-width at half-maximum) of the major and minor
axes, and the position angle of the major axis. These are given by the
user parameters \texttt{kernMaj, kernMin} \& \texttt{kernPA}. If a
circular Gaussian is required, the user need only provide the
\texttt{kernMaj} parameter. The \texttt{kernMin} parameter will then
be set to the same value, and \texttt{kernPA} to zero.  If we define
these parameters as $a,b,\theta$ respectively, we can define the
kernel by the function
\[ 
k(x,y) = \exp\left[-0.5 \left(\frac{X^2}{\sigma_X^2} + 
                              \frac{Y^2}{\sigma_Y^2}   \right) \right] 
\]
where $(x,y)$ are the offsets from the central pixel of the gaussian
function, and 
\begin{align*}
X& = x\sin\theta - y\cos\theta&
  Y&= x\cos\theta + y\sin\theta\\
\sigma_X^2& = \frac{(a/2)^2}{2\ln2}&
  \sigma_Y^2& = \frac{(b/2)^2}{2\ln2}\\
\end{align*}

\secB{Input/Output of reconstructed/smoothed arrays}
\label{sec-reconIO}

The smoothing and reconstruction stages can be relatively
time-consuming, particularly for large cubes and reconstructions in
3-D (or even spatial smoothing). To get around this, \duchamp provides
a shortcut to allow users to perform multiple searches (\eg with
different thresholds) on the same reconstruction/smoothing setup
without re-doing the calculations each time.

To save the reconstructed array as a FITS file, set
\texttt{flagOutputRecon = true}. The file will be saved in the same
directory as the input image, so the user needs to have write
permissions for that directory.

The name of the file can given by the \texttt{fileOutputRecon}
parameter, but this can be ignored and \duchamp will present a name
based on the reconstruction parameters. The filename will be derived
from the input filename, with extra information detailing the
reconstruction that has been done. For example, suppose
\texttt{image.fits} has been reconstructed using a 3-dimensional
reconstruction with filter \#2, thresholded at $4\sigma$ using all
scales. The output filename will then be
\texttt{image.RECON-3-2-4-1.fits} (\ie it uses the four parameters
relevant for the \atrous reconstruction as listed in
Appendix~\ref{app-param}). The new FITS file will also have these
parameters as header keywords. If a subsection of the input image has
been used (see \S\ref{sec-input}), the format of the output filename
will be \texttt{image.sub.RECON-3-2-4-1.fits}, and the subsection that
has been used is also stored in the FITS header.

Likewise, the residual image, defined as the difference between the
input and reconstructed arrays, can also be saved in the same manner
by setting \texttt{flagOutputResid = true}. Its filename will be the
same as above, with \texttt{RESID} replacing \texttt{RECON}.

If a reconstructed image has been saved, it can be read in and used
instead of redoing the reconstruction. To do so, the user should set
the parameter \texttt{flagReconExists = true}. The user can indicate
the name of the reconstructed FITS file using the \texttt{reconFile}
parameter, or, if this is not specified, \duchamp searches for the
file with the name as defined above. If the file is not found, the
reconstruction is performed as normal. Note that to do this, the user
needs to set \texttt{flagAtrous = true} (obviously, if this is
\texttt{false}, the reconstruction is not needed).

To save the smoothed array, set \texttt{flagOutputSmooth = true}. As
for the reconstructed/residual arrays, the
name of the file can given by the \texttt{fileOutputSmooth} parameter,
but this can be ignored and \duchamp will present a name based on the
method of smoothing used. It will be either
\texttt{image.SMOOTH-1D-a.fits}, where a is replaced by the Hanning
width used, or \texttt{image.SMOOTH-2D-a-b-c.fits}, where the Gaussian
kernel parameters are a,b,c. Similarly to the reconstruction case, a
saved file can be read in by setting \texttt{flagSmoothExists = true}
and either specifying a file to be read with the \texttt{smoothFile}
parameter or relying on \duchamp to find the file with the name as
given above.


\secB{Searching the image}
\label{sec-detection}

\secC{Technique}

The basic idea behind detection in \duchamp is to locate sets of
contiguous voxels that lie above some threshold. No size or shape
requirement is imposed upon the detections -- that is, \duchamp does
not fit \eg a Gaussian profile to each source. All it does is find
connected groups of bright voxels.

One threshold is calculated for the entire cube, enabling calculation
of signal-to-noise ratios for each source (see
Section~\ref{sec-output} for details). The user can manually specify a
value (using the parameter \texttt{threshold}) for the threshold,
which will override the calculated value. Note that this option
overrides any settings of \texttt{snrCut} or FDR options (see below). 

The cube can be searched in one of two ways, governed by the input
parameter \texttt{searchType}. If \texttt{searchType=spatial}, the
cube is searched one channel map at a time, using the 2-dimensional
raster-scanning algorithm of \citet{lutz80} that connects groups of
neighbouring pixels. Such an algorithm cannot be applied directly to a
3-dimensional case, as it requires that objects are completely nested
in a row (when scanning along a row, if an object finishes and other
starts, you won't get back to the first until the second is completely
finished for the row). Three-dimensional data does not have this
property, hence the need to treat the data on a 2-dimensional basis at
most.

Alternatively, if \texttt{searchType=spectral}, the searching is done
in one dimension on each individual spatial pixel's spectrum. This is
a simpler search, but there are potentially many more of them.

Although there are parameters that govern the minimum number of pixels
in a spatial, spectral and total senses that an object must have
(\texttt{minPix}, \texttt{minChannels} and \texttt{minVoxels}
respectively), these criteria are not applied at this point - see
\S\ref{sec-reject} for details.

Finally, the search only looks for positive features. If one is
interested instead in negative features (such as absorption lines),
set the parameter \texttt{flagNegative = true}. This will invert the
cube (\ie multiply all pixels by $-1$) prior to the search, and then
re-invert the cube (and the fluxes of any detections) after searching
is complete. All outputs are done in the same manner as normal, so
that fluxes of detections will be negative.

\secC{Calculating statistics}
\label{sec-stats}

A crucial part of the detection process (as well as the wavelet
reconstruction: \S\ref{sec-recon}) is estimating the statistics that
define the detection threshold. To determine a threshold, we need to
estimate from the data two parameters: the middle of the noise
distribution (the ``noise level''), and the width of the distribution
(the ``noise spread''). The noise level is estimated by either the
mean or the median, and the noise spread by the rms (or the standard
deviation) or the median absolute deviation from the median
(MADFM). The median and MADFM are robust statistics, in that they are
not biased by the presence of a few pixels much brighter than the
noise.

All four statistics are calculated automatically, but the choice of
parameters that will be used is governed by the input parameter
\texttt{flagRobustStats}. This has the default value \texttt{true},
meaning the underlying mean of the noise distribution is estimated by
the median, and the underlying standard deviation is estimated by the
MADFM. In the latter case, the value is corrected, under the
assumption that the underlying distribution is Normal (Gaussian), by
dividing by 0.6744888 -- see Appendix~\ref{app-madfm} for details. If
\texttt{flagRobustStats=false}, the mean and rms are used instead.

The choice of pixels to be used depend on the analysis method. If the
wavelet reconstruction has been done, the residuals (defined
in the sense of original $-$ reconstruction) are used to estimate the
noise spread of the cube, since the reconstruction should pick out
all significant structure. The noise level (the middle of the
distribution) is taken from the original array.

If smoothing of the cube has been done instead, all noise parameters
are measured from the smoothed array, and detections are made with
these parameters. When the signal-to-noise level is quoted for each
detection (see \S\ref{sec-output}), the noise parameters of the
original array are used, since the smoothing process correlates
neighbouring pixels, reducing the noise level.

If neither reconstruction nor smoothing has been done, then the
statistics are calculated from the original, input array. 

The parameters that are estimated should be representative of the
noise in the cube. For the case of small objects embedded in many
noise pixels (\eg the case of \hi surveys), using the full cube will
provide good estimators. It is possible, however, to use only a
subsection of the cube by setting the parameter \texttt{flagStatSec =
  true} and providing the desired subsection to the \texttt{StatSec}
parameter. This subsection works in exactly the same way as the pixel
subsection discussed in \S\ref{sec-input}. The \texttt{StatSec} will
be trimmed if necessary so that it lies wholly within the image
subsection being used (\ie that given by the \texttt{subsection}
parameter - this governs what pixels are read in and so are able to be
used in the calculations).

Note that \texttt{StatSec} applies only to the statistics used to
determine the threshold. It does not affect the calculation of
statistics in the case of the wavelet reconstruction. Note also that
pixels flagged as BLANK or as part of the ``Milky Way'' range of
channels are ignored in the statistics calculations.

\secC{Determining the threshold}

Once the statistics have been calculated, the threshold is determined
in one of two ways. The first way is a simple sigma-clipping, where a
threshold is set at a fixed number $n$ of standard deviations above
the mean, and pixels above this threshold are flagged as detected. The
value of $n$ is set with the parameter \texttt{snrCut}. The ``mean''
and ``standard deviation'' here are estimated according to
\texttt{flagRobustStats}, as discussed in \S\ref{sec-stats}. In this
first case only, if the user specifies a threshold, using the
\texttt{threshold} parameter, the sigma-clipped value is ignored.

The second method uses the False Discovery Rate (FDR) technique
\citep{miller01,hopkins02}, whose basis we briefly detail here. The
false discovery rate (given by the number of false detections divided
by the total number of detections) is fixed at a certain value
$\alpha$ (\eg $\alpha=0.05$ implies 5\% of detections are false
positives). In practice, an $\alpha$ value is chosen, and the ensemble
average FDR (\ie $\langle FDR \rangle$) when the method is used will
be less than $\alpha$.  One calculates $p$ -- the probability,
assuming the null hypothesis is true, of obtaining a test statistic as
extreme as the pixel value (the observed test statistic) -- for each
pixel, and sorts them in increasing order. One then calculates $d$
where
\[
d = \max_j \left\{ j : P_j < \frac{j\alpha}{c_N N} \right\},
\]
and then rejects all hypotheses whose $p$-values are less than or
equal to $P_d$. (So a $P_i<P_d$ will be rejected even if $P_i \geq
j\alpha/c_N N$.) Note that ``reject hypothesis'' here means ``accept
the pixel as an object pixel'' (\ie we are rejecting the null
hypothesis that the pixel belongs to the background).

The $c_N$ value here is a normalisation constant that depends on the
correlated nature of the pixel values. If all the pixels are
uncorrelated, then $c_N=1$. If $N$ pixels are correlated, then their
tests will be dependent on each other, and so $c_N = \sum_{i=1}^N
i^{-1}$. \citet{hopkins02} consider real radio data, where the pixels
are correlated over the beam. For the calculations done in \duchamp,
$N = B \times C$, where $B$ is the beam area in pixels, calculated
from the FITS header (if the correct keywords -- BMAJ, BMIN -- are not
present, the size of the beam is taken from the input parameters - see
discussion in \S\ref{sec-results}, and if these parameters are not
given, $B=1$), and $C$ is the number of neighbouring channels that can
be considered to be correlated.

The use of the FDR method is governed by the \texttt{flagFDR} flag,
which is \texttt{false} by default. To set the relevant parameters,
use \texttt{alphaFDR} to set the $\alpha$ value, and
\texttt{FDRnumCorChan} to set the $C$ value discussed above. These
have default values of 0.01 and 2 respectively.

The theory behind the FDR method implies a direct connection between
the choice of $\alpha$ and the fraction of detections that will be
false positives. These detections, however, are individual pixels,
which undergo a process of merging and rejection (\S\ref{sec-merger}),
and so the fraction of the final list of detected objects that are
false positives will be much smaller than $\alpha$. See the discussion
in \S\ref{sec-notes}.

%\secC{Storage of detected objects in memory}
%
%It is useful to understand how \duchamp stores the detected objects in
%memory while it is running. This makes use of nested C++ classes, so
%that an object is stored as a class that includes the set of detected
%pixels, plus all the various calculated parameters (fluxes, WCS
%coordinates, pixel centres and extrema, flags,...). The set of pixels
%are stored using another class, that stores 3-dimensional objects as a
%set of channel maps, each consisting of a $z$-value and a
%2-dimensional object (a spatial map if you like). This 2-dimensional
%object is recorded using ``run-length'' encoding, where each row (a
%fixed $y$ value) is stored by the starting $x$-value and the length

\secB{Merging, growing and rejecting detected objects}
\label{sec-merger}

\secC{Merging}

The searches described above are either 1- or 2-dimensional only. They
do not know anything about the third dimension that is likely to be
present. To build up 3D sources, merging of detections must be
done. This is done via an algorithm that matches objects judged to be
``close'', according to one of two criteria.

One criterion is to define two thresholds -- one spatial and one in
velocity -- and say that two objects should be merged if there is at
least one pair of pixels that lie within these threshold distances of
each other. These thresholds are specified by the parameters
\texttt{threshSpatial} and \texttt{threshVelocity} (in units of pixels
and channels respectively).

Alternatively, the spatial requirement can be changed to say that
there must be a pair of pixels that are \emph{adjacent} -- a stricter,
but perhaps more realistic requirement, particularly when the spatial
pixels have a large angular size (as is the case for \hi
surveys). This method can be selected by setting the parameter
\texttt{flagAdjacent=true} in the parameter file. The velocity
thresholding is always done with the \texttt{threshVelocity} test. 


\secC{Stages of merging}

This merging can be done in two stages. The default behaviour is for
each new detection to be compared with those sources already detected,
and for it to be merged with the first one judged to be close. No
other examination of the list is done at this point.

This step can be turned off by setting
\texttt{flagTwoStageMerging=false}, so that new detections are simply
added to the end of the list, leaving all merging to be done in the
second stage.

The second, main stage of merging is more thorough, Once the searching
is completed, the list is iterated through, looking at each pair of
objects, and merging appropriately. The merged objects are then
included in the examination, to see if a merged pair is suitably close
to a third.

\secC{Growing}

Once the detections have been merged, they may be ``grown'' (this is
essentially the process known elsewhere as ``floodfill''). This is a
process of increasing the size of the detection by adding nearby
pixels (according to the \texttt{threshSpatial} and
\texttt{threshVelocity} parameters) that are above some secondary
threshold and not already part of a detected object. This threshold
should be lower than the one used for the initial detection, but above
the noise level, so that faint pixels are only detected when they are
close to a bright pixel. This threshold is specified via one of two
input parameters. It can be given in terms of the noise statistics via
\texttt{growthCut} (which has a default value of $3\sigma$), or it can
be directly given via \texttt{growthThreshold}. Note that if you have
given the detection threshold with the \texttt{threshold} parameter,
the growth threshold \textbf{must} be given with
\texttt{growthThreshold}. If \texttt{growthThreshold} is not provided
in this situation, the growing will not be done.

The use of the growth algorithm is controlled by the
\texttt{flagGrowth} parameter -- the default value of which is
\texttt{false}. If the detections are grown, they are sent through the
merging algorithm a second time, to pick up any detections that should
be merged at the new lower threshold (\ie they have grown into each
other). 

\secC{Rejecting}
\label{sec-reject}

Finally, to be accepted, the detections must satisfy minimum size
criteria, relating to the number of channels, spatial pixels and
voxels occupied by the object. These criteria are set using the
\texttt{minChannels}, \texttt{minPix} and \texttt{minVoxels}
parameters respectively. The channel requirement means a source must
have at least one set of \texttt{minChannels} consecutive channels to
be accepted. The spatial pixels (\texttt{minPix}) requirement refers
to distinct spatial pixels (which are possibly in different channels),
while the voxels requirement refers to the total number of voxels
detected. If the \texttt{minVoxels} parameter is not provided, it
defaults to \texttt{minPix}$+$\texttt{minChannels}-1.

It is possible to do this rejection stage before the main merging and
growing stage. This could be done to remove narrow (hopefully
spurious) sources from the list before growing them, to reduce the
number of false positives in the final list. This mode can be selected
by setting the input parameter \texttt{flagRejectBeforeMerge=true} --
caution is urged if you use this in conjunction with
\texttt{flagTwoStageMerging=false}, as you can throw away parts of
objects that you may otherwise wish to keep.

%%% Local Variables: 
%%% mode: latex
%%% TeX-master: "Guide"
%%% End: 
