\secA{Introduction and getting going quickly}

This document provides a user's guide to \duchamp, an object-finder
for use on spectral-line data cubes. The basic execution of \duchamp
is to read in a FITS data cube, find sources in the cube, and produce
a text file of positions, velocities and fluxes of the detections, as
well as a postscript file of the spectra of each detection.

\secB{What to do}

So, you have a FITS cube, and you want to find the sources in it. What
do you do? First, you need to get \duchamp: there are instructions in
Appendix~\ref{app-install} for obtaining and installing it. Once you
have it running, the first step is to make an input file that contains
the list of parameters. Brief and detailed examples are shown in
Appendix~\ref{app-input}. This file provides the input file name, the
various output files, and defines various parameters that control the
execution.

The standard way to run \duchamp is by the command
\begin{quote}
\texttt{Duchamp -p [parameter file]}
\end{quote}
replacing \texttt{[parameter file]} with the name of the file listing
the parameters. 

An even easier way is to use the default values for all parameters
(these are given in Appendix~\ref{app-param} and in the file
\texttt{InputComplete} included in the distribution directory) and use
the syntax
\begin{quote}
\texttt{Duchamp -f [FITS file]}
\end{quote}
where \texttt{[FITS file]} is the file you wish to search. 

In either case, the program will then work away and give you the list
of detections and their spectra. The program execution is summarised
below, and detailed in \S\ref{sec-flow}. Information on inputs is in
\S\ref{sec-param} and Appendix~\ref{app-param}, and descriptions of
the output is in \S\ref{sec-output}.

\secB{Guide to terminology and conventions}

First, a brief note on the use of terminology in this guide. \duchamp
is designed to work on FITS ``cubes''. These are FITS\footnote{FITS is
the Flexible Image Transport System -- see \citet{hanisch01} or
websites such as
\href{http://fits.cv.nrao.edu/FITS.html}{http://fits.cv.nrao.edu/FITS.html}
for details.} image arrays with (at least) three dimensions. They
are assumed to have the following form: the first two dimensions
(referred to as $x$ and $y$) are spatial directions (that is, relating
to the position on the sky -- often, but not necessarily,
corresponding to Equatorial or Galactic coordinates), while the third
dimension, $z$, is the spectral direction, which can correspond to
frequency, wavelength, or velocity. The three dimensional analogue of
pixels are ``voxels'', or volume cells -- a voxel is defined by a
unique $(x,y,z)$ location and has a unique value of flux, intensity
or brightness (or something equivalent) associated with it.

Note that it is possible for the FITS file to have more than three
dimensions (for instance, there could be a fourth dimension
representing a Stokes parameter). Only the two spatial dimensions and
the spectral dimension are read into the array of pixel values that is
searched for objects. All other dimensions are ignored\footnote{This
actually means that the first pixel only of that axis is used, and the
array is read by the \texttt{fits\_read\_subsetnull} command from the
\textsc{cfitsio} library.}. Herein, we discuss the data in terms of
the three basic dimensions, but you should be aware it is possible for
the FITS file to have more than three. Note that the order of the
dimensions in the FITS file does not matter.

With this setup, each spatial pixel (a given $(x,y)$ coordinate) can
be said to be a single spectrum, while a slice through the cube
perpendicular to the spectral direction at a given $z$-value is a
single channel (the 2-D image is a channel map).

Detection involves locating a contiguous group of voxels with fluxes
above a certain threshold. \duchamp makes no assumptions as to the
size or shape of the detected features, other than having
user-selected minimum size criteria. Features that are detected are
assumed to be positive. The user can choose to search for negative
features by setting an input parameter -- this inverts the cube prior
to the search (see \S\ref{sec-detection} for details).

Finally, note that it is possible to run \duchamp on a
two-dimensional image (\ie one with no frequency or velocity
information), or indeed a one-dimensional array, and many of the
features of the program will work fine. The focus, however, is on
object detection in three dimensions, one of which is a spectral
dimension.

\secB{A summary of the execution steps}

The basic flow of the program is summarised here -- all steps are
discussed in more detail in the following sections.
\begin{enumerate}
\item The necessary parameters are recorded.

  How this is done depends on the way the program is run from the
  command line. If the \texttt{-p} option is used, the parameter file
  given on the command line is read in, and the parameters therein are
  read. All other parameters are given their default values (listed in
  Appendix~\ref{app-param}).

  If the \texttt{-f} option is used, all parameters are assigned their
  default values.

\item The FITS image is located and read in to memory.

  The file given is assumed to be a valid FITS file. As discussed
  above, it can have any number of dimensions, but \duchamp only
  reads in the two spatial and the one spectral dimensions. A subset
  of the FITS array can be given (see \S\ref{sec-input} for details).

\item If requested, a FITS file containing a previously reconstructed
  or smoothed array is read in.

  When a cube is either Hanning-smoothed or reconstructed with the
  \atrous wavelet method, the result can be saved to a FITS file, so
  that subsequent runs of \duchamp can read it in to save having to
  re-do the reconstruction (as it can be relatively time-intensive).

\item \label{step-blank} If requested, BLANK pixels are trimmed from
  the edges, and the baseline of each spectrum is removed.

  When BLANK pixels are present, they can adversely affect the
  reconstruction algorithms, as well as increasing the size in memory
  of the array. This step trims them from the spatial edges, recording
  the amount trimmed so that they can be added back in later.

  A spectral baseline can be removed at this point as well. This may
  be necessary if there is a ripple or other large-scale feature
  present that will hinder detection of faint sources.

\item If the reconstruction method is requested, and the reconstructed
  array has not been read in at Step 3 above, the cube is
  reconstructed using the \atrous wavelet method.

  This step uses the \atrous method to determine the amount of
  structure present at various scales. A simple thresholding technique
  then removes random noise from the cube, leaving the significant
  signal. This process can greatly reduce the noise level in the cube,
  enhancing the detectability of sources.

\item Alternatively (and if requested), the each spectral channel is
  Hanning-smoothed by a desired amount.

  This step considers each spectrum individually, and convolves it
  with a Hanning filter (with width chosen by the user). This can help
  to reduce the amount of noise visible in the cube.

\item A threshold for the cube is then calculated, based on the pixel
  statistics (unless a threshold is manually specified by the user).

  The threshold can either be chosen as a simple $n\sigma$ threshold
  (\ie so many standard deviations above the mean), or calculated via
  the ``False Discovery Rate'' method. Alternatively, the threshold
  can be specified as a simple flux value, without care as to the
  statistical significance (\eg ``I want every source brighter than
  10mJy'').

\item Searching for objects then takes place, using the requested
  thresholding method.

%  The cube is searched in the following manner: each 1-D spectrum is
%  searched, followed by each 2-D image. Any objects detected in each
%  search are added to a master list, or combined with objects already
%  in that list.
  The cube is searched channel-map by channel-map. Detections are
  compared to already detected objects and either combined with a
  neighbouring one or added to the end of the list.

\item The list of objects is condensed by merging neighbouring objects
  and removing those deemed unacceptable.

  While some merging has been done in the previous step, this process
  is a much more rigorous comparison of each object with every other
  one. If a pair of objects lie within requested limits, they are
  combined. 

  After the merging is done, the list is culled (although see comment
  for the next step). There are certain criteria the user can specify
  that objects must meet: minimum numbers of spatial pixels and
  spectral channels, and minimum separations between neighbouring
  objects. Those that do not meet these criteria are deleted
  from the list.

\item If requested, the objects are ``grown'' down to a lower
  threshold, and then the merging step is done a second time.

  In this case, each object has pixels in its neighbourhood examined,
  and if they are above a secondary threshold, they are added to the
  object. The merging process is done a second time in case two
  objects have grown over the top of one another. Note that the
  rejection part of the previous step is not done until the end of the
  second merging process.

\item The baselines and trimmed pixels are replaced prior to output.

  This is just the inverse of step~\#\ref{step-blank}.

\item The details of the detections are written to screen and to the
  requested output file.

  Crucial properties of each detection are provided, showing its
  location, extent, and flux. These are presented in both pixel
  coordinates and world coordinates (\eg sky position and
  velocity). Any warning flags are also printed, showing detections to
  be wary of.

\item Maps showing the spatial location of the detections are written.

  These are 2-dimensional maps, showing where each detection lies on
  the spatial coverage of the cube. This is provided as an aid to the
  user so that a quick idea of the distribution of object positions
  can be gained \eg are all the detections on the edge?

  Two maps are provided: one is a 0th moment map, showing the 0th
  moment of each detection in its appropriate position, while the
  second is a ``detection map'', showing the number of times each
  spatial pixel was detected in the searching routines.

  These maps are written to postscript files, and the 0th moment map
  can also be displayed in a PGPLOT X-window.

\item The integrated or peak spectra of each detection are written to a
  postscript file. 

  The spectral equivalent of the maps -- what is the spectral profile
  of each detection? Also provided here are basic information for each
  object (a summary of the information in the results file), as well
  as a 0th moment map of the detection.

\item If requested, the reconstructed or smoothed array can be written
  to a new FITS file.

  If either of these procedures were done, the resulting array can be
  saved as a FITS file for later use. The FITS header will be the same
  as the input file, with a few additional keywords to identify the
  file. 

\end{enumerate}

