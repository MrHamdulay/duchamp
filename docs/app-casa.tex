% -----------------------------------------------------------------------
% app-casa.tex: Example output region file for CASA
% -----------------------------------------------------------------------
% Copyright (C) 2006, Matthew Whiting, ATNF
%
% This program is free software; you can redistribute it and/or modify it
% under the terms of the GNU General Public License as published by the
% Free Software Foundation; either version 2 of the License, or (at your
% option) any later version.
%
% Duchamp is distributed in the hope that it will be useful, but WITHOUT
% ANY WARRANTY; without even the implied warranty of MERCHANTABILITY or
% FITNESS FOR A PARTICULAR PURPOSE.  See the GNU General Public License
% for more details.
%
% You should have received a copy of the GNU General Public License
% along with Duchamp; if not, write to the Free Software Foundation,
% Inc., 59 Temple Place, Suite 330, Boston, MA 02111-1307, USA
%
% Correspondence concerning Duchamp may be directed to:
%    Internet email: Matthew.Whiting [at] atnf.csiro.au
%    Postal address: Dr. Matthew Whiting
%                    Australia Telescope National Facility, CSIRO
%                    PO Box 76
%                    Epping NSW 1710
%                    AUSTRALIA
% -----------------------------------------------------------------------
\secA{Example CASA Region file output}
\label{app-casa}

This is the format of the CASA region file, showing the locations
of the detected objects. This can be loaded in casapy, and should be
able to be used in the casaviewer image viewer (once that
functionality is made available)

\begin{verbatim}
#CRTFv0
# Duchamp Source Finder v.1.5
# Results for FITS file: /home/mduchamp/fountain.fits
# imageFile              /home/mduchamp/fountain.fits
# flagSubsection         0
# flagStatSec            0
# searchType             spectral
# flagNegative           0
# flagBaseline           0
# flagRobustStats        1
# flagFDR                0
# snrCut                 3.5
# flagGrowth             0
# minPix                 5
# minChannels            3
# minVoxels              7
# flagAdjacent           1
# threshVelocity         7
# flagRejectBeforeMerge  0
# flagTwoStageMerging    0
# pixelCentre            centroid
# flagSmooth             0
# flagATrous             1
# reconDim               1
# scaleMin               1
# scaleMax               8
# snrRecon               4
# reconConvergence       0.005
# filterCode             1
# Detection threshold used = 0.0461842
# Mean of noise background = 0.000122074
# Std. Deviation of noise background = 0.0131606
#  [Using robust methods]
global color=red, coord=J2000 
#
box[[92.842095deg,-22.277967deg], [91.899837deg,-21.755824deg]], label='1'
line[[92.839864deg,-22.144360deg], [92.767849deg,-22.145538deg]]
line[[92.838756deg,-22.077549deg], [92.766775deg,-22.078727deg]]
line[[92.767849deg,-22.145538deg], [92.695836deg,-22.146680deg]]
...
text[[92.378723deg,-21.960721deg], '1']

\end{verbatim}

%%% Local Variables: 
%%% mode: latex
%%% TeX-master: "Guide"
%%% End: 
