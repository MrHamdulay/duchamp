\secA{Obtaining and installing \duchamp}
\label{app-install}

The \duchamp\ web page can be found at the following location:\\
\href{http://www.atnf.csiro.au/people/Matthew.Whiting/Duchamp}%
{http://www.atnf.csiro.au/people/Matthew.Whiting/Duchamp}\\
Here you can find a gzipped tar archive of the source code that can be
downloaded and extracted, as well as this User's Guide in postscript
and hyperlinked PDF formats.

To build \duchamp, you will need three main external libraries:
\textsc{pgplot}, \textsc{cfitsio} and \textsc{wcslib}. If these are not present on your system,
you can download them from the following locations:
\begin{itemize}
\item \textsc{pgplot}:
\href{http://www.astro.caltech.edu/~tjp/pgplot/}%
{http://www.astro.caltech.edu/~tjp/pgplot/}
\item \textsc{cfitsio}:
\href{http://heasarc.gsfc.nasa.gov/docs/software/fitsio/fitsio.html}%
{http://heasarc.gsfc.nasa.gov/docs/software/fitsio/fitsio.html}
\item \textsc{wcslib}:
\href{http://www.atnf.csiro.au/people/Mark.Calabretta/WCS/index.html}%
{http://www.atnf.csiro.au/people/Mark.Calabretta/WCS/index.html}
\end{itemize}

\duchamp\ can be built on Unix/Linux systems by typing (assuming that
the prompt your terminal provides is a \texttt{> } -- don't type this
character!):
\begin{quote}
\texttt{%
> ./configure\\
> make\\
> make clean (optional -- to remove the object files)}
\end{quote}

Run in this manner, \texttt{configure} should find all the necessary
libraries, but if some libraries have been installed in non-standard
locations, it may fail. In this case, you can specify additional
directories to look in by giving extra command-line arguments. There
are separate options for library files (eg. libcpgplot.a) and header
files (eg. cpgplot.h).

For example, suppose \textsc{wcslib} had been locally installed in
\texttt{/home/mduchamp/wcslib}. There will then be two libraries
created that are likely to be in the following subdirectories:
\texttt{C/} and \texttt{pgsbox/}. Each subdirectory needs to be
searched for library and header files, so one could build Duchamp by
typing:
\begin{quote}
\texttt{%
>  ./configure $\backslash$ \\ 
LIBDIRS="/home/mduchamp/wcslib/C /home/mduchamp/wcslib/pgsbox" 
$\backslash$\\
INCDIRS="/home/mduchamp/wcslib/C /home/mduchamp/wcslib/pgsbox"}
\end{quote}
And then just run make in the usual fashion:
\begin{quote}
\texttt{> make}
\end{quote}

This will produce the executable \texttt{Duchamp}. You can verify that
it is running correctly by running the verification shell script:
\begin{quote}
\texttt{> VerifyDuchamp.sh}
\end{quote}
This will use a dummy FITS image in the \texttt{verification/}
directory -- this image has some Gaussian random noise, with five
Gaussian sources present, plus a dummy WCS. The script runs
Duchamp on this image with three different sets of inputs, and
compares to known results, looking for differences and reporting
any. There should be none reported if everything is working correctly.

You can then run \duchamp\ on your own data. This can be done in one
of two ways. The first is:
\begin{quote}
\texttt{> Duchamp -f [FITS file]}
\end{quote}
where \texttt{[FITS file]} is the file you wish to search. This method
simply uses the default values of all parameters.

The second method allows some determination of the parameter values by
the user. Type:
\begin{quote}
\texttt{> Duchamp -p [parameter file]}
\end{quote}
where \texttt{[parameterFile]} is a file with the input parameters,
including the name of the cube you want to search. There are two
example input files included with the distribution. The smaller one,
\texttt{InputExample}, shows the typical parameters one might want to
set. The large one, \texttt{InputComplete}, lists all possible
parameters that can be entered, and a brief description of them. To
get going quickly, just replace the "your-file-here" in
\texttt{InputExample} with your image name, and type
\begin{quote}
\texttt{> Duchamp -p InputExample}
\end{quote}

The following appendices provide details on the individual parameters,
and show examples of the output files that \duchamp\ produces.
