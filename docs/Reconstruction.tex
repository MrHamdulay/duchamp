\documentclass[11pt]{article}
\usepackage[sort]{natbib}
\usepackage{url}
\usepackage{graphicx}
\usepackage{lscape}
\bibpunct[,]{(}{)}{;}{a}{}{,}
\textwidth=161 mm
\textheight=248 mm
\topmargin=-13 mm
\oddsidemargin=0 mm
\parindent=6 mm

\newcommand{\eg}{e.g.\ }
\newcommand{\ie}{i.e.\ }
\newcommand{\hi}{H{\sc i}}
\newcommand{\hipass}{{\sc hipass}}
\newcommand{\progname}{{\tt Duchamp}}
\newcommand{\entrylabel}[1]{\mbox{\textsf{\bf{#1:}}}\hfil}
\newenvironment{entry}
        {\begin{list}{}%
                {\renewcommand{\makelabel}{\entrylabel}%
                        \setlength{\labelwidth}{30mm}%
                        \setlength{\labelsep}{5pt}%
                        \setlength{\itemsep}{2pt}%
                        \setlength{\parsep}{2pt}%
                        \setlength{\leftmargin}{35mm}%
                }%
        }%
{\end{list}}

\title{The ``noiseless reconstruction'' of astronomical data cubes
  using the multi-scale {\it \`a trous} wavelet technique.}

\author{Matthew Whiting\\Australia Telescope National Facility\\CSIRO}

\date{November 2005}

\begin{document}

\maketitle

\begin{abstract}
We describe a technique to reconstruct a three-dimensional FITS data
cube using multi-scale wavelet decomposition. The technique provides a
marked reduction in the noise level of the cube, while retaining
objects, providing an excellent basis for a source-finding algorithm. 
\end{abstract}

\section{Background}

An important step in most astronomical data analysis that involves
multi-dimensional imaging or spectroscopic data is the detection of
sources. Often, astronomical sources (be they stars, galaxies, masers
or otherwise) are faint and of a strength close to the noise or
background of the image. Any procedure that could reduce this
statistical background without removing the real features would be a
great aid in detecting such sources.

This is of great interest for large-scale surveys: large-scale here
meaning both the size of data produced as well as the area of the sky
they cover. The data rate seen in many current and planned surveys
necessitates a largely automated pipeline reduction process, with
minimal input from a user***. An object-detection (and
characterisation) process is the logical next step (particularly with
a view to producing source catalogues and the like), and such a
process will need to be as sensitive as possible. This means beating
the noise level in some way. 

*** MATCHED FILTERS ***

*** SMOOTHING ***

*** WAVELETS ***

\section{Wavelet decomposition}

The technique we describe here relies on the properties of
wavelets. These are localised functions that are described by two
parameters, location (where the wavelet is operating) and scale (what
range of values it operates on). An example of a wavelet is shown in
Fig.~\ref{fig-wavelet}. 

\begin{figure}
\vspace{7.0cm}
\caption{An example of a wavelet function.}
\label{fig-wavelet}
\end{figure}



\section{Implementation}

\subsection{Method}

\subsection{Edge effects}


\section{Results}

\section{Applications of the technique}

\section{Conclusions}


\end{document}