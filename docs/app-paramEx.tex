\secA{Example parameter files}
\label{app-input}

This is what a typical parameter file would look like.

\begin{verbatim}
imageFile       /home/mduchamp/fountain.fits
logFile         logfile.txt
outFile         results.txt
spectraFile     spectra.ps
flagSubsection  false
flagOutputRecon false
flagOutputResid 0
flagTrim        1
flagMW          1
minMW           75
maxMW           112
flagGrowth      1
growthCut       1.5
flagATrous      1
reconDim        1          
scaleMin        1
snrRecon        4
flagFDR         1
alphaFDR        0.1
snrCut          3
threshSpatial   3
threshVelocity  7
\end{verbatim}

Note that, as in this example, the flag parameters can be entered as
strings (true/false) or integers (1/0). Also, note that it is not
necessary to include all these parameters in the file, only those that
need to be changed from the defaults (as listed in
Appendix~\ref{app-param}), which in this case would be very few. A
minimal parameter file might look like:
\begin{verbatim}
imageFile       /home/mduchamp/fountain.fits
flagLog         false
flagATrous      1
snrRecon        3
snrCut          2.5
minChannels     4
\end{verbatim}
This will reconstruct the cube with a lower SNR value than the
default, select objects at a lower threshold,  with a looser minimum
channel requirement, and not keep a log of the intermediate
detections. 

The following page demonstrates how the parameters are presented to
the user, both on the screen at execution time, and in the output and
log files. On each line, there is a description on the parameter, the
relevant parameter name that is used in the input file (if there is
one that the user can enter), and the value of the parameter being
used.

%\begin{sideways}
%Typical presentation of parameters in output and log files:  
%\begin{minipage}{170mm}
{\scriptsize
\begin{verbatim}
---- Parameters ----
Image to be analysed.........................[imageFile]  =  fountain.fits
Reconstructed array exists?................[reconExists]  =  true
FITS file containing reconstruction..........[reconFile]  =  fountain.RECON-1-1-4-1.fits
Intermediate Logfile...........................[logFile]  =  duchamp-Logfile.txt
Final Results file.............................[outFile]  =  duchamp-Results.txt
Spectrum file..............................[spectraFile]  =  duchamp-Spectra.ps
0th Moment Map...............................[momentMap]  =  duchamp-MomentMap.ps
Detection Map.............................[detectionMap]  =  duchamp-DetectionMap.ps
Display a map in a pgplot xwindow?.........[flagXOutput]  =  true
Saving reconstructed cube?.............[flagoutputrecon]  =  false
Saving residuals from reconstruction?..[flagoutputresid]  =  false
------
Blank Pixel Value.......................................  =  -8.00061
Trimming Blank Pixels?........................[flagTrim]  =  true
Searching for Negative features?..........[flagNegative]  =  false
Removing Milky Way channels?....................[flagMW]  =  true
Milky Way Channels.......................[minMW - maxMW]  =  75-112
Beam Size (pixels)......................................  =  10
Removing baselines before search?.........[flagBaseline]  =  false
Smoothing each spectrum first?..............[flagSmooth]  =  false
Using A Trous reconstruction?...............[flagATrous]  =  true
Number of dimensions in reconstruction........[reconDim]  =  1
Minimum scale in reconstruction...............[scaleMin]  =  1
SNR Threshold within reconstruction...........[snrRecon]  =  4
Filter being used for reconstruction........[filterCode]  =  1 (B3 spline function)
Using FDR analysis?............................[flagFDR]  =  false
SNR Threshold (in sigma)........................[snrCut]  =  3
Minimum # Pixels in a detection.................[minPix]  =  2
Minimum # Channels in a detection..........[minChannels]  =  3
Growing objects after detection?............[flagGrowth]  =  false
Using Adjacent-pixel criterion?...........[flagAdjacent]  =  true
Max. velocity separation for merging....[threshVelocity]  =  7
Method of spectral plotting.............[spectralMethod]  =  peak
Type of object centre used in results......[pixelCentre]  =  centroid
--------------------
\end{verbatim}
}
%\end{minipage}
%\end{sideways}
