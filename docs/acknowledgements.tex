% -----------------------------------------------------------------------
% acknowledgements.tex: Acknowledgements & thankyous
% -----------------------------------------------------------------------
% Copyright (C) 2006, Matthew Whiting, ATNF
%
% This program is free software; you can redistribute it and/or modify it
% under the terms of the GNU General Public License as published by the
% Free Software Foundation; either version 2 of the License, or (at your
% option) any later version.
%
% Duchamp is distributed in the hope that it will be useful, but WITHOUT
% ANY WARRANTY; without even the implied warranty of MERCHANTABILITY or
% FITNESS FOR A PARTICULAR PURPOSE.  See the GNU General Public License
% for more details.
%
% You should have received a copy of the GNU General Public License
% along with Duchamp; if not, write to the Free Software Foundation,
% Inc., 59 Temple Place, Suite 330, Boston, MA 02111-1307, USA
%
% Correspondence concerning Duchamp may be directed to:
%    Internet email: Matthew.Whiting [at] atnf.csiro.au
%    Postal address: Dr. Matthew Whiting
%                    Australia Telescope National Facility, CSIRO
%                    PO Box 76
%                    Epping NSW 1710
%                    AUSTRALIA
% -----------------------------------------------------------------------
\secA{Acknowledgements}

Thanks are due to the many people who have provided assistance and
advice during the development and testing of \duchamp, particularly
Ivy Wong, Kathrin Wolfinger, Tobias Westmeier, Mary Putman, Cormac
Purcell, Attila Popping, Tara Murphy, Enno Middelberg, Korinne
McDonnell, Philip Lah, Russell Jurek, Simon Guest, Jose Francisco
Gomez, BiQing For, Luca Cortese, David Barnes and Robbie
Auld. Additionally, Emil Lenc and Anita Richards both provided
valuable comments on the journal paper that have helped the
descriptions therein. I'd like to thank the members of the ASKAP
Working Group 2 (Source Finding) for their interest and feedback -
\duchamp has been considered as part of the development of ASKAP
Survey Science Projects, and this has driven further development of
the core \duchamp software.

The graphics in \duchamp are created using the
\textsc{pgplot}\footnote{http://www.astro.caltech.edu/$\sim$tjp/pgplot/}
library, FITS file access is controlled through the
\textsc{cfitsio}\footnote{http://heasarc.gsfc.nasa.gov/docs/software/fitsio/fitsio.html}
library \citep{pence99}, while the world-coordinate transformations
are performed using the
\textsc{wcslib}\footnote{http://www.atnf.csiro.au/people/mcalabre/WCS/index.html}
library (described in \citet{calabretta02}).

The bulk of this work was conducted as part of a CSIRO Emerging
Science Postdoctoral Fellowship, and \duchamp continues to be
maintained both as a standalone package and as part of the software
development effort for the ASKAP telescope. This work was supported by
the NCI National Facility at the ANU.



%%% Local Variables: 
%%% mode: latex
%%% TeX-master: "Guide"
%%% End: 
