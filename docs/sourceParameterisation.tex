% -----------------------------------------------------------------------
% outputs.tex: Section detailing the different forms of text- and
%              plot-based output.
% -----------------------------------------------------------------------
% Copyright (C) 2006, Matthew Whiting, ATNF
%
% This program is free software; you can redistribute it and/or modify it
% under the terms of the GNU General Public License as published by the
% Free Software Foundation; either version 2 of the License, or (at your
% option) any later version.
%
% Duchamp is distributed in the hope that it will be useful, but WITHOUT
% ANY WARRANTY; without even the implied warranty of MERCHANTABILITY or
% FITNESS FOR A PARTICULAR PURPOSE.  See the GNU General Public License
% for more details.
%
% You should have received a copy of the GNU General Public License
% along with Duchamp; if not, write to the Free Software Foundation,
% Inc., 59 Temple Place, Suite 330, Boston, MA 02111-1307, USA
%
% Correspondence concerning Duchamp may be directed to:
%    Internet email: Matthew.Whiting [at] atnf.csiro.au
%    Postal address: Dr. Matthew Whiting
%                    Australia Telescope National Facility, CSIRO
%                    PO Box 76
%                    Epping NSW 1710
%                    AUSTRALIA
% -----------------------------------------------------------------------
\secA{Source Parameterisation}
\label{sec-sourceparam}

Once sources have been located, numerous measurements are made so that
they can be placed in a catalogue. This section details each of the
source parameters, explaining what they are and how they are
calculated. Each parameter is referred to by the heading of the
relevant column(s) in the output source list (see
\S\ref{sec-output}).

\secB{Object ID, \texttt{Obj\#}}
\label{sec-objectID}

The ID  of the detection is an integer, simply the sequential count for the
list. The default is ordering by increasing spectral coordinate, or channel
number, if the WCS is not good enough to determine the spectral world
coordinate, but this can be changed by the \texttt{sortingParam} input
parameter. See Sec~\ref{sec-results} for details.

\secB{Object Name, \texttt{Name}}

This is the ``IAU''-format name of the detection, derived from the WCS
position if available.  For instance, a source that is centred on the
RA,Dec position 12$^h$53$^m$45$^s$, -36$^\circ$24$'$12$''$ will be
given the name J125345$-$362412, if the epoch is J2000, or the name
B125345$-$362412 if it is B1950. The precision of the RA and Dec
values is determined by the pixel size, such that sufficient precision
is used to uniquely define a position. The RA value will have one
figure greater precision than Dec.

An alternative form is used for Galactic coordinates: a source centred
on the position ($l$,$b$) = (323.1245, 5.4567) will be called
G323.124$+$05.457. 

If the WCS is not valid (\ie is not present or does not have all the
necessary information), the name will instead be of the form
``ObjXXX'', where XXX is replaced with the objectID, padded
sufficiently with zeros.

\secB{Pixel locations}

There are three ways in which the pixel location of the detection is
calculated: 
\begin{itemize}
\item Peak: the pixel value in which the detection has its peak
  flux. Appears in the results file under columns \texttt{X\_peak,
    Y\_peak, Z\_peak}.
\item Average: the average over all detected pixels. Specifically,
  $x_\text{av}=\sum x_i / N$ and similarly for $y_\text{av}$ and
  $z_\text{av}$. Appears in the results file under columns \texttt{X\_av,
    Y\_av, Z\_av}.
\item Centroid: the flux-weighted average over all detected pixels. Specifically,
  $x_\text{cent}=\sum F_i x_i / \sum F_i$ and similarly for $y_\text{cent}$ and
  $z_\text{cent}$. Appears in the results file under columns \texttt{X\_cent,
    Y\_cent, Z\_cent}.
\end{itemize}

All three alternatives are calculated, and written to the results
file, but the choice of the \texttt{pixelCentre} input parameter will
determine which option is used for the reference values \texttt{X, Y,
  Z}.

\secB{Spatial world location, \texttt{RA/GLON, DEC/GLAT}}

These are the conversion of the \texttt{X} and \texttt{Y} pixel
positions to world coordinates (that is, the pixel position determined
by \texttt{pixelCentre}). These will typically be Right Ascension and
Declination, or Galactic Longitude and Galactic Latitude, but the
actual names used in the output file will be taken from the WCS
specification. 

If there is no useful WCS, these are not calculated.

\secB{Spectral world location, \texttt{VEL}}
\label{sec-vel}

The conversion of the \texttt{Z} pixel position to the spectral world
coordinates. This is dictated by the WCS of the FITS file plus the
input parameter \texttt{spectralType}. The name of the output column
will come from the CTYPE of the spectral axis (or
\texttt{spectralType} -- see \S\ref{sec-wcs}), specifically , the
  4-character S-type code) (\ie not necessarily ``VEL'')

The spectral units can be specified by the user, using the input
parameter \texttt{spectralUnits} (enter it as a single string with no
spaces). The default value comes from the FITS header. 

\secB{Spatial size, \texttt{MAJ, MIN, PA}}
\label{sec-shape}

The spatial size of the detection is measured from the moment-0 map
(in the case of three-dimensional data) or the two-dimensional image,
and is parameterised by the FWHM of the major and minor axes, plus the
position angle of the major axis.

The position angle will be measured in the usual astronomical sense,
in degrees East of North. The major and minor axes will be specified
in angular units (assuming the WCS allows this), with the units chosen
such that the numbers are not too small.

The method for calculating these parameters is to form the moment-0
map (if necessary), select all pixels greater than half the maximum
\footnote{In the event of a negative search (see
  \S\ref{sec-searchTechnique}), the moment map is inverted prior to
  this selection.},
then calculate the parameters $a$ (major FWHM), $b$ (minor FWHM) and
$\theta$ (position angle) according to
\begin{eqnarray*}
\frac{1}{2} a^2  &= & S_{xx} + S_{yy} + \sqrt{ (S_{xx} -
  S_{yy})^2 + 4 (S_{xy})^2}\\
\frac{1}{2} b^2  &= & S_{xx} + S_{yy} - \sqrt{ (S_{xx} -
  S_{yy})^2 + 4 (S_{xy})^2}\\
\tan 2\theta &= &\frac{2 S_{xy}}{S_{xx} - S_{yy}}
\end{eqnarray*}
where the sums $S_{xx}$, $S_{yy}$ and $S_{xy}$ are calculated in one
of two ways. First, the algorithm tries to weight each pixel by its
flux:
\begin{eqnarray*}
S_{xx} &= &\sum F_i x_i^2 / \sum F_i\\
S_{yy} &= &\sum F_i y_i^2 / \sum F_i\\
S_{xy} &= &\sum F_i x_i y_i / \sum F_i
\end{eqnarray*}
Mostly, this will work. But there can be situations where the
calculated value of $b^2$ is negative (that is, $S_{xx}+S_{yy} <
\sqrt{(S_{xx}-S_{yy})^2+4S_{xy}^2}$, or $S_{xx}S_{yy}<S_{xy}^2$). These situations are
often where the moment-0 map is very disordered with no clear primary
axis.

In this case, the calculation of the sums is redone without the flux
weighting ($S_{xx} = \sum x_i^2$ etc), and the shape parameters
recalculated. A \textbf{W} flag will be recorded for the detection to
indicate that the weighting failed: see \S\ref{sec-flags} below.

It is possible that this second calculation will fail, particuarly for
sources with only a small number of spatial pixels. In this case, we
revert to using the method of principle axes. 

We first calculate the angle of minimum moment and assign this as the
position angle. This is defined by calculating the net moments:
\begin{eqnarray*}
M_{xx} &= & \sum x_i^2 - \frac{1}{N} \left(\sum x_i\right)^2\\
M_{yy} &= & \sum y_i^2 - \frac{1}{N} \left(\sum y_i\right)^2\\
M_{xy} &= & \sum x_i y_i - \frac{1}{N} \sum x_i \sum y_i\\
\end{eqnarray*}
then
\[
\tan \theta = \frac{(M_{xx} - M_{yy} + \sqrt{(M_{xx}-M_{yy})^2+4M_{xy}})}{2M_{xy}}.
\]
To find the sizes of the principle axes (the major axis aligning with
the angle just calculated, and the minor being perpendicular to it),
we calculate the projection along these two axes of each pixel above
half the peak in the moment-0 map, and take the range between the
maximum and minimum, requiring it to be at least one pixel. Note this
is not sensitive to the flux distribution. If this calculation is
used, a \textbf{w} flag will be recorded for the detection: see
\S\ref{sec-flags} below.


\secB{Spatial extent, \texttt{w\_RA/w\_GLON, w\_DEC}}

The extent of the detection in each of the spatial directions is also
calculated. This is simply the angular width of the detection (in
arcmin), converting the minimum and maximum values of $x$ (usually RA)
and $y$ (Dec) to the world coordinates.

\secB{Spectral width, \texttt{w\_50, w\_20, w\_VEL}}

Several measures of the spectral extent of a detection are
provided. The simplest is the full spectral width, calculated as for
the spatial extents above. This is referred to as \texttt{w\_VEL}, but
need not be velocity. It is obtained by taking the difference in world
coordinates of the minimum and maximum values of $z$. The units are as
described in \S\ref{sec-vel}.

Two other measures of the spectral width are provided, \texttt{w\_50}
and \texttt{w\_20}, being the width at 50\% and 20\% of the peak
flux. These are measured on the integrated spectrum (\ie the spectra
of all detected spatial pixels summed together), and are defined by
starting at the outer spectral extent of the object (the highest and
lowest spectral values) and moving in or out until the required flux
threshold is reached. The widths are then just the difference between
the two values obtained. If the detection threshold is greater than
20\% or 50\% of the peak, then these values will be the same as
\texttt{w\_VEL}.

\secB{Source flux, \texttt{F\_tot, F\_int, F\_peak}}
\label{sec-fluxparams}

%% THE FOLLOWING IS FROM THE MNRAS PAPER
%The flux of the source is given by the peak flux and both the ``total
%flux'', defined above as $F_T$, and the ``integrated flux'' $F_I$, or
%the total flux integrated over the spectral extent and corrected for
%the beam:
%\[
%F_I = \frac{\sum F_i \Delta v_i}{B}
%\]
%where $B=\pi \alpha \beta / 4 \ln(2)$ is the area of a beam of major
%and minor axes $\alpha$ and $\beta$ (in pixels), and $\Delta v_i$ is
%the spectral width of each voxel.

There are two measurements of the total flux of the detection. The
simplest, \texttt{F\_tot}, is just the sum of all detected pixels in
the image: $F_\text{tot}=\sum F_i$. The alternative, \texttt{F\_int},
is the flux integrated over the detected pixels, taking into account
the spectral range. For the case of velocity, the expression is
$F_\text{int} = \sum F_i \Delta v_i$, where $\Delta v_i$ is the
velocity width of the channel containing pixel $i$. The actual units
of the spectral range are as described in \S\ref{sec-vel}.

When the cube brightness units are quoted per beam (\eg Jy/beam), then
the integrated flux \texttt{F\_int} includes a correction for
this. This involves dividing by the integral over the beam. This is
calculated using the BMAJ, BMIN \& BPA header keywords from the FITS
file. BMAJ and BMIN are assumed to be the full-width at half maximum
(FWHM) in the major and minor axis directions of a Gaussian beam. The
integral is calculated as follows: the functional form of a 2D
elliptical Gaussian can be written as
$\exp(-((x^2/2\sigma_x^2)+(y^2/2\sigma_y^2)))$, and the FWHM in a
given direction is then $f=2\sqrt{2\ln2}\sigma$. Then, $F_\text{int} =
C \sum F_i \Delta v_i$, where 
\[
C = \int\exp\left(-\left(\frac{x^2}{2\sigma_x^2}+\frac{y^2}{2\sigma_y^2}\right)\right)
= 2\pi\sigma_x\sigma_y
=\frac{\pi f_x f_y}{4\ln2}
\]
provides the correction factor to go from units of Jy/beam to Jy.

If the FITS file does not have the beam information, the user can
either:
\begin{enumerate}
\item Specify the FWHM of the beam in pixels (assuming a circular
  beam) via the parameter \texttt{beamFWHM}.
\item Specify the area of the beam, again in pixels, via the parameter
  \texttt{beamArea}\footnote{Note that this is equivalent to the old
    parameter \texttt{beamSize}, which was highlighted as being
    ambiguous.}. This should be the value given by the equation above.
\end{enumerate}
If both are given, \texttt{beamFWHM} takes precendence. If neither are
given, and there is no beam information in the header, then no
correction to the integrated flux is made (and so it will stay in
units of Jy/beam or equivalent).

Note that these parameters are measured using \textit{only the
  detected pixels}. The summing of the flux will not include voxels
that fall below the detection (or growth) threshold -- which is in
accord with the definition of the threshold as dividing source and
non-source voxels. If the threshold is not low enough, this will bias
the measurement of the fluxes. This applies to all parameters (with
the exception of the \texttt{w\_50} and \texttt{w\_20} widths, which
are measured from the integrated spectrum, including channels not
necessarily forming part of the detection).

Finally, the peak flux \texttt{F\_peak} is simply the maximum value of
the flux over all the detected pixels.


\secB{Error on total/integrated flux, \texttt{eF\_tot, eF\_int}}

Both \texttt{F\_tot} and \texttt{F\_int} can also have their
associated error calculated (\texttt{eF\_tot} and \texttt{eF\_int}
respectively). This is the random error due to the noise in the image,
and is simply the sum in quadrature of the noise on each of the
voxels, and, in the case of \texttt{F\_int} multiplied by the spectral
width and corrected for the beam if necessary. Since we assume a
constant noise level in the image ($\sigma_i=\sigma\  \forall i$), we
have:
\begin{eqnarray*}
eF_\text{int} &= & \sqrt{\sum\sigma_i^2} \\&= &\sigma \sqrt{N}\\
eF_\text{tot} &= & \sqrt{\sum C^2\sigma_i^2 \Delta v_i^2} \\&= &C \sigma
\sqrt{N} \Delta v \text{ (for the case of $\Delta v_i = \Delta v$)}
\end{eqnarray*}
In the case that a flux threshold is provided, these quantities are
not calculated, since we don't measure the image noise
statistics. Likewise, when the array is smoothed we measure the noise
only in the smoothed image, and this value is not applicable to the
flux measured from the original image, so the errors are not
reported. 


\secB{Peak signal-to-noise, \texttt{S/Nmax}}

This parameter converts the peak flux to a signal-to-noise value,
based on the measured noise level in the image. As for the error
quantities above, if no noise is measured (\ie a flux threshold is
provided by the user), then this is not calculated.

When the array is pre-processed (via smoothing or wavelet
reconstruction), we take the peak flux here to be the peak in the
smoothed or reconstructed array. This is because this is where the
detection is made, and so the \texttt{S/Nmax} value can be directly
compared to the requested signal-to-noise threshold. Note that the
peak flux discussed in \S\ref{sec-fluxparams} is always measured from
the original image array.

\secB{Pixel ranges, \texttt{X1, X2, Y1, Y2, Z1, Z2}}

These quantities give the range of pixel values spanned by the
detection in each of the three axes. \texttt{X1, Y1, Z1} give the
minimum pixel in each direction, while \texttt{X2, Y2, Z2} give the
maximum pixel.

\secB{Size, \texttt{Nvoxel, Nchan, Nspatpix}}

The number of detected pixels that make up the detection. These
quantities show the total number of voxels in the detection
(\texttt{Nvoxel}), the number of distinct spectral channels
(\texttt{Nchan}) and the number of distinct spatial pixels
(\texttt{Nspatpix}). 

Note that \texttt{Nchan} here is different to the quantity tested by
the input parameter \texttt{minChannels} (see \S\ref{sec-reject}) --
this looks at the maximum number of \textit{adjacent} channels, not
the total.

\secB{Warning Flags, \texttt{Flag}}
\label{sec-flags}

The detection can have warning flags recorded, to highlight potential
issues:
\begin{itemize}
\item \textbf{E} -- The detection is next to the spatial edge of the
  image, meaning either the limit of the pixels, or the limit of the
  non-BLANK pixel region.
\item \textbf{S} -- The detection lies at the edge of the spectral
  region.
\item \textbf{F} -- The detection is adjacent to, or overlaps one or
  more flagged channels (see \S\ref{sec-flagging}).
\item \textbf{N} -- The total flux, summed over all the (non-BLANK)
  pixels in the smallest box that completely encloses the detection,
  is negative. Note that this sum is likely to include non-detected
  pixels. It is of use in pointing out detections that lie next to
  strongly negative pixels, such as might arise due to interference --
  the detected pixels might then also be due to the interference, so
  caution is advised.
\item \textbf{W} -- The weighting of fluxes in the shape calculation
  (Sec~\ref{sec-shape}) failed, so the unweighted calculation was
  used. This likely indicates some very disordered shape for the
  moment-0 map.
\item \textbf{w} -- The unweighted calculation also failed, most
  likely because there are too few pixels. In this case, we use the
  alternate method of principle axes to estimate the shape.
\end{itemize}
In the absence of any of these flags, a \textbf{-} will be recorded.


%%% Local Variables: 
%%% mode: latex
%%% TeX-master: "Guide"
%%% End: 
