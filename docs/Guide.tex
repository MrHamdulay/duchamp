\documentclass[12pt]{article}
\usepackage[sort]{natbib}
\usepackage{url}
\usepackage{graphicx}
\usepackage{lscape}
\usepackage{hyperref}
\bibpunct[,]{(}{)}{;}{a}{}{,}
\textwidth=161 mm
\textheight=248 mm
\topmargin=-13 mm
\oddsidemargin=0 mm
\parindent=6 mm

\newcommand{\eg}{e.g.\ }
\newcommand{\ie}{i.e.\ }
\newcommand{\hi}{H{\sc i}}
\newcommand{\hipass}{{\sc hipass}}
\newcommand{\progname}{{\tt Duchamp}}
\newcommand{\diff}{{\rm d}}
\newcommand{\entrylabel}[1]{\mbox{\textsf{\bf{#1:}}}\hfil}
\newenvironment{entry}
        {\begin{list}{}%
                {\renewcommand{\makelabel}{\entrylabel}%
                        \setlength{\labelwidth}{30mm}%
                        \setlength{\labelsep}{5pt}%
                        \setlength{\itemsep}{2pt}%
                        \setlength{\parsep}{2pt}%
                        \setlength{\leftmargin}{35mm}%
                }%
        }%
{\end{list}}

\title{A Guide to the {\it Duchamp} Source Finding Software}
\author{Matthew Whiting\\Australia Telescope National Facility\\CSIRO}
%\date{January 2006}
\date{}

\begin{document}

\maketitle
\tableofcontents

\newpage
\section{Introduction and getting going quickly}

This document gives details on the use of the program Duchamp. This
has been written to provide a source-detection facility for
spectral-line data cubes. The basic execution of Duchamp is to read
in a FITS data cube, find sources in the cube, and produce a text
file of positions, velocities and fluxes of the detections, as well as
a postscript file of the spectra of each detection. 

So, you have a FITS cube, and you want to find the sources in it. What
do you do? The first step is to make an input file that contains the
list of parameters. Brief and detailed examples are shown in
Appendix~\ref{app-input}. This provides the input file name, the various
output files, and defines various parameters that control the
execution.

The program is run by the command
\begin{quote}
{\tt Duchamp -p [parameter file]}
\end{quote}
replacing {\tt [parameter file]} with the name of the file you have
just created/copied. The program will then work away and give you the
list of detections and their spectra. The program execution is
summarise below, and detailed in \S\ref{sec-flow}. Information on
inputs is in \S\ref{sec-param} and Appendix~\ref{app-param}, and
descriptions of the output is in \S\ref{sec-output}.

\subsection{A summary of the execution steps}

The basic flow of the program is summarised here. All these steps are
discussed in more detail in the following sections, so read on if
there are questions!
\begin{enumerate}
\item The parameter file given on the command line is read in, and the
  parameters absorbed.
\item From the parameter file, the FITS image is located and read in
  to memory.
\item If requested, blank pixels are trimmed from the edges, and
  channels corresponding to bright (\eg Galactic) emission are
  excised. 
\item If requested, the baseline of each spectrum is removed.
\item If the reconstruction method is requested, the cube is
  reconstructed using the {\it {\' a} trous} wavelet method.
\item Searching for objects then takes place, using the requested
  thresholding method.
\item The list of objects is trimmed by merging neighbouring objects
  and removing those deemed unacceptable.
\item The baselines and trimmed pixels are replaced prior to output.
\item The details on the detections are written to screen and to the
  requested output file.
\item Maps showing the spatial location of the detections are written.
\item The integrated spectra of each detection are written to a
  postscript file. 
\item If requested, the reconstructed array can be written to a new
  FITS file.
\end{enumerate}

\subsection{Guide to terminology}

First, a brief note on the use of terminology in this guide. The FITS
cube is assumed to have the following form: the first two dimensions
(referred to as $x$ and $y$) are spatial directions (that is, relating
to the position on the sky), while the third dimension, $z$, is the
spectral direction, which can correspond to frequency, wavelength, or
velocity. 

Each spatial pixel (a given $(x,y)$ coordinate) can be said to be a
single spectrum, while a slice through the cube perpendicular to the
spectral direction at a given $z$-value is a single channel (the 2-D
image is a channel map).

\section{User Inputs}
\label{sec-param}

Input to the program is provided by means of a parameter file. Parameters
are listed in the file, followed by the value that should be assigned
to them. The syntax used is {\tt paramName value}. The file is not
case-sensitive, and lines in the input file that start with {\tt \#} are
ignored. If a parameter is listed more than once, the latter value is
used, but otherwise the order in which the parameters are listed in the
input file is arbitrary. 

If a parameter is not listed, the default value is assumed. The
defaults are chosen to provide a good result (using the reconstruction
method), so the user can get away with only specifying a minimal
number of parameters in the file. Note that the image file {\bf must}
be specified! The parameters that can be set are listed in
Appendix~\ref{app-param}, with their default values in parentheses.

The 'flag' parameters are stored as {\tt bool} variables, and so are
either {\tt true = 1} or {\tt false = 0} -- currently the program only
reads them from the file as integers, and so they should be entered in
the file as 0 or 1 (see example file in Appendix~\ref{app-input}).

\section{What the program is doing}
\label{sec-flow}

The execution flow of the program is detailed here, indicating the
main algorithmic steps that are used. The program is written in C/C++
and makes use of the {\sc cfitsio}, {\sc wcslib} and {\sc pgplot}
libraries. 

%\subsection{Parameter input}
%
%The user provides parameters that govern the selection of files and
%the parameters used by the various subroutines in the program. This is
%done via a parameter file, and the parameters are stored in a C++
%class for use throughout the program. The form of the parameter file is
%discussed in \S\ref{sec-param}, and the parameters themselves are
%listed in Appendix~\ref{app-param}.

\subsection{Image input}

The cube is read in using basic {\sc cfitsio} commands, and stored as
an array in a special C++ class structure. This class keeps track of
the list of detected objects, as well as any reconstructed arrays that
are made (see \S\ref{sec-recon}). The WCS information for the cube is
also obtained from the FITS header by {\sc wcslib} functions
\citep{greisen02, calabretta02}, and this information, in the form of
a {\tt wcsprm} structure, is also stored in the same class.

A sub-section of an image can be requested. This is done via the {\tt
subsection} parameter in the parameter file. The generalised form of
the subsection that is used by {\sc cfitsio} is {\tt
[x1:x2:dx,y1:y2:dy,z1:z2:dz]}, such that the x-coordinates run from
{\tt x1} to {\tt x2} (inclusive), with steps of {\tt dx}. The step
value can be omitted (so a subsection of the form {\tt
[2:50,2:50,10:1000]} is still valid). Duchamp does not at this
stage deal with the presence of steps in the subsection string, and
any that are present are removed before the file is opened.

If one wants the full range of a value then replace the range with an
asterisk, \eg {\tt [2:50,2:50,*]}. If one wants to use just a
subsection, one must set {\tt flagSubsection = 1}. A complete
description of the section syntax can be found at the {\sc fitsio} web
site
\footnote{http://heasarc.gsfc.nasa.gov/docs/software/fitsio/c/c\_user/node90.html}.

\subsection{Image modification}
\label{sec-modify}

Several modifications to the cube can be made that improve the
execution and efficiency of Duchamp (these are optional -- their
use is indicated by the relevant flags set in the input parameter
file).

\subsubsection{Milky-Way removal}

First, a single set of contiguous channels can be removed -- these may
exhibit very strong emission, such as that from the Milky Way as seen
in extragalactic \hi\ cubes (hence the references to ``Milky Way'' in
relation to this task -- apologies to Galactic astronomers!). Such
dominant channels will both produce many unnecessary, uninteresting
and large (in size and hence in memory) detections, and will also
affect any reconstruction that is performed (see next section). The
use of this feature is controlled by the {\tt flagMW} parameter, and
the exact channels concerned are able to be set by the user (using
{\tt maxMW} and {\tt minMW}). When employed, the flux in these
channels is set to zero. The information in those channels is not
kept.

\subsubsection{Blank pixel removal}

Second, the cube is trimmed of BLANK pixels that pad the image
out to a rectangular shape. This is also optional, being determined by
the {\tt flagBlankPix} parameter. The value for these pixels is read from
the FITS header (using the BLANK, BSCALE and BZERO keywords), but if
these are not present then the value can be specified by the user in
the parameter file. If these blank pixels are stored as NaNs, then a
normal number will be substituted (allowing these pixels to be
accurately removed without adverse effects).

This stage is particularly important for the reconstruction step, as
lots of BLANK pixels on the edges will smooth out features in the
wavelet calculation stage. The trimming will also reduce the size of
the cube's array, speeding up the execution. The amount of trimming is
recorded, and these pixels are added back in once the source-detection
is completed (so that quoted pixel positions are applicable to the
original cube).

Rows and columns are trimmed one at a time until the first non-BLANK
pixel is reached, so that the image remains rectangular. In practice,
this means that there will be BLANK pixels left in the trimmed image
(if the non-BLANK region is non-rectangular). However, these are
ignored in all further calculations done on the cube.

\subsubsection{Baseline removal}

Finally, the user may request the removal of baselines from the
spectra, via the parameter {\tt flagBaseline}. This may be necessary
if there is a strong baseline ripple present, which can result in
spurious detections on the high points of the ripple. The baseline is
calculated from a wavelet reconstruction procedure (see
\S\ref{sec-recon}) that keeps only the two largest scales. This is
done separately for each spatial pixel (\ie for each spectrum in the
cube), and the baselines are stored and added back in before any
output is done. In this way the quoted fluxes and displayed spectra
are as one would see from the input cube itself -- the detection (and
reconstruction if applicable) is done on the baseline-removed cube
however.

\subsection{Image reconstruction}
\label{sec-recon}

This is an optional step. The user can direct Duchamp to
reconstruct the data cube using the {\it {\`a} trous} wavelet
procedure. A good description of the procedure can be found in
\citet{starck02:book}. This is an effective way of removing a
lot of the noise in the image, but at this stage is relatively time-
and cpu-intensive. The steps in the procedure are as follows:
\begin{enumerate}
\item Set the reconstructed array to 0 everywhere.
\item The cube is discretely convolved with a given filter
  function. This is determined from the parameter file via the {\tt
  filterCode} parameter -- see Appendix~\ref{app-param} for details on
  the filters available.
\item The wavelet coefficients are calculated by taking the difference
  between the convolved array and the input array.
\item If the wavelet coefficients at a given point are above the
  threshold requested (given by {\tt snrRecon} as the number of
  $\sigma$ above the mean and adjusted to the current scale), add
  these to the reconstructed array.
\item The separation of the filter coefficients is doubled.
\item The procedure is repeated from step 2, using the convolved array
  as the input array.
\item Continue until the required maximum number of scales is reached.
\item Add the smoothed array to the reconstructed array. This provides
  the ``DC offset''.
\end{enumerate}


The statistics of the cube are estimated using robust methods, to
avoid corruption by strong outlying points. The mean is actually
estimated by the median, while the median absolute deviation from the
median (MADFM) is calculated and corrected assuming Gaussianity to
estimate the standard deviation $\sigma$. The Gaussianity (or
Normality) assumption is critical, as the MADFM does not give the same
value as the usual rms or standard deviation value -- for a normal
distribution $N(\mu,\sigma)$ we find MADFM$=0.6744888\sigma$. The
difference between the MADFM and $\sigma$ is corrected for, so the
user need only think in the usual multiples of $\sigma$ when setting
{\tt snrRecon}. See Appendix~\ref{app-madfm} for a derivation of this
value.

When thresholding the different wavelet scales, the value of $\sigma$
as measured from the input array needs to be scaled to account for the
increased amount of correlation between neighbouring pixels (due to
the convolution). 

The user can also select the minimum scale to be used in the
reconstruction -- the first scale exhibits the highest frequency
variations, and so ignoring this one can sometimes be beneficial in
removing excess noise. The default, however, is to use all scales
({\tt minscale = 1}).

The reconstruction has at least two iterations. The first iteration
makes a first pass at the wavelet reconstruction (the process outlined
in the 8 stages above), but the residual array will inevitably have
some structure still in it, so the wavelet filtering is done on the
residual, and any significant wavelet terms are added to the final
reconstruction. This step is repeated until the change in the $\sigma$
of the background is less than some fiducial amount.

The user can optionally select to save the reconstructed image as a
FITS file -- at the moment this is just saved in the same directory as
the input file, so it won't work if the user does not have write
permissions on that directory. See Appendix~\ref{app-param} for details
on the naming of the output image. The residual image, which is the
difference between the input image and the reconstructed image, can
also be saved in the same manner. 

Finally, note that any BLANK pixels that are still in the cube
will not be altered by the reconstruction -- they will be left as
BLANK so that the shape of the valid part of the cube is preserved.

\subsection{Searching the image}
\label{sec-detection}

The image is searched for detections in two ways: spectrally (in the
spectrum in each spatial pixel), and spatially (in the spatial image
in each channel). In both cases, the algorithm finds connected pixels
that are above the user-specified threshold. In the case of the
spatial image search, the algorithm of \citet{lutz80} is used to
raster scan through the image and connect groups of pixels on
neighbouring rows.

Note that this algorithm cannot be applied directly to a 3-dimensional
case, as it requires that objects are completely nested in a row: that
is, if you are scanning along a row, and one object finishes and
another starts, you know that you will not get back to the first one
(if at all) until the second is finished for that
row. Three-dimensional data does not have this property, which is why
we break up the searching into 1- and 2-dimensional cases.

The determination of the threshold is done in one of two ways. The
first way is a simple sigma-clipping, where a threshold defined as
$n\sigma$ above the mean is set and pixels above this threshold are
flagged as detected. As before, the value for $\sigma$ is estimated by
the MADFM, and corrected by the ratio derived in
Appendix~\ref{app-madfm}. 

The second method uses the False Discovery Rate (FDR) technique
\citep{miller01,hopkins02}, whose basis we briefly detail here. The
false discovery rate (given by the number of false detections divided
by the total number of detections) is fixed at a certain value
$\alpha$ (\eg $\alpha=0.05$ implies 5\% of detections are false
positives). In practice, an $\alpha$ value is chosen, and the ensemble
average FDR (\ie $<FDR>$) when the method is used will be less than
$\alpha$.  One calculates $p$ -- the probability, assuming the null
hypothesis is true, of obtaining a test statistic as extreme as the
pixel value (the observed test statistic) -- for each pixel, and sorts
them in increasing order. One then calculates $d$ where
\[
d = \max_j \left\{ j : P_j < \frac{j\alpha}{c_N N} \right\}.
\]
And then rejects all hypotheses whose $p$-values are less than or equal
to $P_d$. (So a $P_i<P_d$ will be rejected even if $P_i \geq
j\alpha/c_N N$.) Note that ``reject hypothesis'' here means ``accept
the pixel as an object pixel'' (\ie we are rejecting the null
hypothesis that the pixel belongs to the background). 

The $c_N$ values here are normalisation constants that depend on the
correlated nature of the pixel values. If all the pixels are
uncorrelated, then $c_N=1$. If $N$ pixels are correlated, then their
tests will be dependent on each other, and so $c_N = \sum_{i=1}^N
i^{-1}$. \citet{hopkins02} consider real radio data, where the pixels
are correlated over the beam. In this case the sum is made over the
$N$ pixels that make up the beam. The value of $N$ is calculated from
the FITS header (if the correct keywords are not present, a default
value of 10 pixels is assumed).

If a reconstruction has been made, the residuals (defined as original
$-$ reconstruction) are used to estimate the noise parameters of the
cube. Otherwise they are estimated directly from the cube itself. In
both cases, the median is used as a robust estimator of the mean
value, although the $\sigma$ is estimated by the standard deviation
(of the residual array, in the case of the reconstruction, but of the
original array otherwise).

Once a detection has been made, it is ``grown''. This is a process of
increasing the size of the detection by adding adjacent pixels that
are above some secondary threshold. This threshold is lower than the
one used for the initial detection, but above the noise level, so that
faint pixels are only detected when they are close to a bright
pixel. The value of this threshold is a possible input parameter ({\tt
growthCut}), with a default value of $1.5\sigma$.

\subsection{Merging detected objects}
\label{sec-merger}

The searching step produces a list of detections that will have many
repeated detections of a given object -- for instance, spectral
detections in adjacent pixels of the same object. These are then
combined in an algorithm that matches all objects judged to be
``close''. This determination is made in one of two ways.

One way is to define two thresholds -- one spatial and one in velocity
-- and say that two objects should be merged if there is at least one
pair of pixels that lie within these threshold distances of each
other. These thresholds are specified by the parameters {\tt
threshSpatial} and {\tt threshVelocity} (in units of pixels and
channels respectively).

Alternatively, the spatial requirement can be changed to say that
there must be a pair of pixels that are {\it adjacent} -- a stricter,
but more realistic requirement, particularly when the spatial pixels
have a large angular size (as is the case for \hi\ surveys). This
method can be selected by setting the parameter
{\tt flagAdjacent} to 1 (\ie {\tt true}) in the parameter file. The
velocity thresholding is done in the same way as the first option.

The detections must also span a minimum number of channels (to remove
any spurious single-channel spikes that may be present). This number
is set with the {\tt minChannels} parameter. This requirement means
there must be at least one set of this many consecutive channels in
the source for it to be accepted. 

\section{Outputs}
\label{sec-output}

\subsection{During execution}

Duchamp provides the user with feedback whilst it is running, to
keep the user informed on the progress of the analysis. Most of this
consists of self-explanatory messages about the particular stage the
program is up to. The relevant parameters are printed to the screen at
the start (once the file has been successfully read in), so the user
is able to make a quick check that the setup is correct.

If the cube is being trimmed (\S\ref{sec-modify}), the resulting
dimensions are printed to indicate how much has been trimmed. If a
reconstruction is being done, a continually updating message shows the
current iteration and scale (compared to the maximum scale). 

During the searching algorithms, the progress through the 1D and 2D
searches are shown. When the searches have completed,
the number of objects found in both the 1D and 2D searches are
reported (see \S\ref{sec-detection} for details).

In the merging process (where multiple detections of the same object
are combined -- see \S\ref{sec-merger}), two stages of output
occur. The first is when each object in the list is compared with all
others. The output shows two numbers: the first being how far through
the list we are, and the second being the length of the list. As the
algorithm proceeds, the first number should increase and the second
should decrease (as objects are being combined). When the numbers
meet, the second phase begins, of removing multiply-appearing pixels
in each object and removing objects not meeting the minimum channels
requirement. During this phase, the total number of accepted objects
is shown, which should steadily increase until all have been accepted
or rejected. Note that these steps can be very quick for small numbers
of detections.

Since this continual printing to screen has some overhead of time and
CPU involved, the user can elect to not print this information by
setting the parameter {\tt verbose = 0}. In this case, the user is
still informed as to the steps being undertaken, but the details of
the progress are not shown.

\subsection{Results}

Once the detection list is finalised, the results are printed to the
screen and to the output file, denoted by the {\tt OutFile}
parameter. The results list, an example of which can be seen in
Appendix~\ref{app-output}, contains the following columns:

\begin{entry}
\item[Obj\#] The ID number of the detection (simply the sequential
  count for the list, which is ordered by increasing velocity).
\item[Name] The IAU-format name of the detection (based on the RA \&
Dec).
\item[X] The average X-pixel position.
\item[Y] The average Y-pixel position.
\item[Z] The average Z-pixel position.
\item[RA] The Right Ascension of the centre of the object.
\item[DEC] The Declination of the centre of the object.
\item[w\_RA] The width of the object in Right Ascension [arcmin].
\item[w\_DEC] The width of the object in Declination [arcmin].
\item[VEL] The mean velocity of the object [km/s].
\item[w\_VEL] The full velocity width of the detection (max channel
  $-$ min channel, in velocity units [km/s]).
\item[X1, X2] The minimum and maximum X-pixel coordinates.
\item[Y1, Y2] The minimum and maximum Y-pixel coordinates.
\item[Z1, Z2] The minimum and maximum Z-pixel coordinates.
\item[Npix] The number of pixels \& channels (\ie distinct $(x,y,z)$
  coordinates) in the detection.
\item[F\_tot] The integrated flux over the object, in the units of the
  FITS file. %This is calculated 
\item[F\_peak] The peak flux over the object, in the units of the FITS
  file.
\end{entry}

A VOTable-format XML file can also be produced, containing just the RA,
Dec, Velocity and the corresponding widths of the detections. The user
should set {\tt flagVOT = 1}, and put the desired filename in the
parameter {\tt votFile}.

As the program is running, it also (optionally) records the detections
made in each individual spectrum or channel (see
\S\ref{sec-detection} for details on this process). This is
recorded in the file denoted by the parameter {\tt LogFile}. This file
does not include the columns {\tt Name, RA, DEC, w\_RA, w\_DEC, VEL,
w\_VEL}. This file is designed primarily for diagnostic purposes: \eg
to see if a given set of pixels is detected in, say, one channel
image, but does not survive the merging process. The list of pixels
(and their fluxes) in the final detection list are also printed to
this file, again for diagnostic purposes. This feature can be turned
off by setting {\tt flagLog = false}. (This may be a good idea if you
are not interested in its contents, as it can be a large file.)

As well as the output data file, a postscript file is created that
shows the integrated spectra of each detection, together with a small
cutout image (0th moment) and basic information of the detection. If
the cube was reconstructed, the spectrum from the reconstruction is
shown in red, over the top of the original spectrum. The spectral
extent of the detection is indicated with green lines, and a zoom is
shown in a separate window. The cutout image can optionally include a
border around the spatial pixels that are in the detection (turned on
and off by the parameter {\tt drawBorders}). It also includes a scale bar
in the bottom left corner to indicate size -- it is 30~arcmin long. An
example detection can be seen below in Fig.~\ref{fig-spect}.

Finally, a couple of images are optionally produced: a 1st moment map
of the cube, combining just the detected channels in each object,
showing the integrated flux in grey-scale; and a ``detection image'',
a grey-scale image where the pixel values are the number of channels
that spatial pixel is detected in. In both cases, if {\tt drawBorders =
true}, a border is drawn around the spatial extent of each
detection. An example moment map is shown in Fig.~\ref{fig-moment}.
The production or otherwise of these images is governed by the {\tt
flagMaps} parameter.

The purpose of these images are to provide a visual guide to where the
detections have been made, and, particularly in the case of the moment
map, to provide an indication of the strength of the source. In both
cases, the detections are numbered (in the same way as the output
list), and the spatial borders are marked out as for the cutout images
in the spectra file. Both these images are saved as postscript files
(given by the parameters {\tt momentMap} and {\tt detectionMap}
respectively), with the latter also displayed in a {\sc pgplot}
window (regardless of the state of {\tt flagMaps}).

\begin{figure}[t]
\begin{center}
\includegraphics[width=\textwidth]{example_spectrum.ps}
\end{center}
\caption{An example of the spectrum output. Note several of the
  features discussed in the text: the removal of the Milky Way
  emission around 0 km/s; the red lines indicating the reconstructed
  spectrum and the spatial extent of the detection; the green dashed
  lines indicating the spectral extent of the detection.}
\label{fig-spect}
\end{figure}

\begin{figure}[h]
\begin{center}
\includegraphics[width=\textwidth]{example_moment_map.ps}
\end{center}
\caption{An example of the moment map created by Duchamp. The full
  extent of the cube is covered, and the 1st moment of each object is
  shown (integrated separately over all the detected channels).}
\label{fig-moment}
\end{figure}

\section{Notes and hints on the use of Duchamp}

In using Duchamp, the user has to make a number of decisions about
the way the program runs. This section is designed to give the user
some idea about what the various selections do...

The main choice is whether or not to use the wavelet
reconstruction. The main benefits of this are the marked reduction in
the noise level, leading to regularly-shaped detections, and good
reliability for faint sources. The main drawback with its use is the
long execution time: to reconstruct a $170\times160\times1024$
(\hipass) cube often requires three iterations and takes about 20-25
minutes. The searching part of the procedure is much quicker (although
see the note on merging, below), so if one uses the FDR method on the
un-reconstructed cube, the execution time is only a couple of minutes.

A further drawback with the reconstruction is that it is susceptible
to edge effects. If the valid area in the cube (\ie the part that is
not BLANK) has very curved edges (such as the \hipass\ polar cap cube,
H001, which has a roughly circular shape after gridding), the
convolution can produce artefacts in the reconstruction that mimic
the edges and lead to some spurious sources. Caution is advised with
such data -- the user is advised to check carefully the reconstructed
cube for the presence of such artefacts.

If one chooses the reconstruction method, a further decision is
required on the signal-to-noise cutoff used in determining acceptable
wavelet coefficients. A larger value will remove more noise from the
cube, at the expense of losing fainter sources, while a smaller value
will include more noise, which may produce spurious detections, but
will be more sensitive to faint sources. Values of less than about
$3\sigma$ tend to not reduce the noise a great deal and can lead to
many spurious sources.

The FDR method certainly produces more reliable results than a simple
sigma-clipping (\ie thresholding at some number of $\sigma$ above the
mean). However, at this point it does not seem to be giving the
sensitivity expected for the supplied value of {\tt alpha} (\ie it is
not finding as many sources as expected). Work is
being done to assess this, and to judge whether there is a real
problem (such as with the determination of the statistics), or simply
a result of working in 3 dimensions as opposed to 2.

A further point to bear in mind is that the shape of the detections in
a cube that has been reconstructed will be much more regular and
smooth -- the ragged edges that objects in the raw cube possess are
smoothed by the removal of most of the noise.

\section{Drawbacks of the current program}

The program currently has a few problems/drawbacks/things to be aware
of that will hopefully be fixed in the future:
\begin{itemize}

\item Narrow interference spikes are still getting found, particularly
  if there is no reconstruction, or reconstruction with a relatively
  low {\tt snrRecon} (such as 2 or 3). Increasing the {\tt
  minChannels} parameter is one way to circumvent this, but making the
  algorithm a bit more clever would be preferable.

\item Sources that have strong continuum ripple and/or artefacts often
  generate many spurious detections. This needs some work to avoid
  Duchamp doing this, and until then users are advised to be aware
  of the possibility. Strong continuum ripples may generate many
  sources on the same spatial pixel, and this will be apparent on the
  detection images.

\item Spectra are integrated over every spatial pixel of the
  detection, and this may dilute the actual detection, making it
  harder to see \ie the apparent strength of the line as plotted may
  not give a true indication of how strong it really is.

%\item A caution on the merging part of the procedure. This can be time
%  consuming if there are many detections that do not require merging
%  -- in this case, the time will go like $N^2$ ($N$ = number of
%  detections). If there are plenty of mergers, the size of the list
%  reduces quickly, so the execution time will be less.


\end{itemize}


%\section{Comparison with other software (to be developed further...)}
%
%\subsection{fred, by Matt Howlett}
%
%This is the program used in the \hipass\ analysis. It smoothes the
%data spectrally with a boxcar filter of a size that varies over a
%user-specified range, and then thresholds the data.
%
%Works effectively, but generally doesn't find as many sources as
%Duchamp, particularly when the reconstruction is used. Sensitive to
%faint, broad features that fall below the reconstruction threshold.
%
%Execution takes a long time, depending on the range of filter widths
%that are used.
%
%\subsection{sfind}
%
%Hard to evaluate, as it does not (as far as I can see) output the
%channel number at which detections are made, and does not merge
%detections made at adjacent channels (\ie it just works in 2
%dimensions). 
%

\section{Future Developments}

This is both a list of planned improvements and a wish-list of
features that would be nice to include (but are not planned in the
immediate future):

\begin{itemize}

\item More varied output formats, including a Karma annotation file. {\bf Planned.}

\item Better determination of the noise characteristics of
  spectral-line cubes, including understanding how the noise is
  generated and developing a model for it. {\bf Planned.}
  
\item Include more source analysis. Examples could be: shape
  information; measurements of HI mass; better measurements of
  velocity width and profile... {\bf Some planned.}

\item Provide some indication of the significance of the detection
  (\ie some S/N-like value). {\bf Planned.}

\item Link to lists of possible counterparts (\eg via NED/SIMBAD/other
  VO tools?). {\bf Wishlist.} 

\item Add ability to read in a reconstructed cube that has been
  saved. In this case the residual array will also need to be read
  in. The idea of this will be to avoid the extended time required for
  the reconstruction if the same cube is being analysed multiple
  times. {\bf Wishlist.}
 
\item At this point, the ``Milky Way'' channels are discarded and set
  to zero. It may be that users would like to have those put back in
  the final cube after the source detection is done, so at some point
  this option may be added. {\bf Wishlist -- if needed.}

\end{itemize}

\section{Why ``Duchamp''?}

Well, it's important for a program to have a name, and it certainly
beats the initial version of ``cubefind''. I had planned to call it
``Picasso'' (as in the father of cubism), but sadly this had already
been used before \citep{minchin99}. So I settled on naming it after
Marcel Duchamp, another cubist, but also one of the first artists to
work with ``found objects''.


%\bibliographystyle{mn2e}
%\bibliographystyle{abbrvnat}
%\bibliography{mnrasmnemonic,sourceDetection}
\begin{thebibliography}{}

\bibitem[\protect\citeauthoryear{{Calabretta} \& {Greisen}}{{Calabretta} \&
  {Greisen}}{2002}]{calabretta02}
{Calabretta} M.,  {Greisen} E.,  2002, A\&A, 395, 1077

\bibitem[\protect\citeauthoryear{{Greisen} \& {Calabretta}}{{Greisen} \&
  {Calabretta}}{2002}]{greisen02}
{Greisen} E.,  {Calabretta} M.,  2002, A\&A, 395, 1061

\bibitem[\protect\citeauthoryear{{Hopkins}, {Miller}, {Connolly}, {Genovese},
  {Nichol} \& {Wasserman}}{{Hopkins} et~al.}{2002}]{hopkins02}
{Hopkins} A.,  {Miller} C.,  {Connolly} A.,  {Genovese} C.,  {Nichol} R.,
  {Wasserman} L.,  2002, AJ, 123, 1086

\bibitem[\protect\citeauthoryear{Lutz}{Lutz}{1980}]{lutz80}
Lutz R.,  1980, The Computer Journal, 23, 262

\bibitem[\protect\citeauthoryear{{Meyer} et~al.,}{{Meyer}
  et~al.}{2004}]{meyer04:trunc}
{Meyer} M.,  et~al., 2004, MNRAS, 350, 1195

\bibitem[\protect\citeauthoryear{{Miller}, {Genovese}, {Nichol}, {Wasserman},
  {Connolly}, {Reichart}, {Hopkins}, {Schneider} \& {Moore}}{{Miller}
  et~al.}{2001}]{miller01}
{Miller} C.,  {Genovese} C.,  {Nichol} R.,  {Wasserman} L.,  {Connolly} A.,
  {Reichart} D.,  {Hopkins} A.,  {Schneider} J.,    {Moore} A.,  2001, AJ, 122,
  3492

\bibitem[\protect\citeauthoryear{Minchin}{Minchin}{1999}]{minchin99}
Minchin R.,  1999, PASA, 16, 12

\bibitem[\protect\citeauthoryear{Starck \& Murtagh}{Starck \&
  Murtagh}{2002}]{starck02:book}
Starck J.-L.,  Murtagh F.,  2002, {``Astronomical Image and Data Analysis''}.
Springer

\end{thebibliography}


\appendix
\newpage
\section{Available parameters}
\label{app-param}

The full list of parameters that can be listed in the input file are
given here. If not listed, they take the default value given in
parentheses. Since the order of the parameters in the input file does
not matter, they are grouped here in logical sections.

\subsection*{Input-output related}
\begin{entry}
\item[ImageFile (no default assumed)] The filename of the
  data cube to be analysed.
\item[OutFile {\tt [./duchamp-Results]}] The file where the final
  detections are to be recorded. This also records the list of input
  parameters.
\item[SpectraFile {\tt [./duchamp-Spectra.ps]}] The postscript file
  containing the resulting integrated spectra and images of the
  detections. 
\item[flagLog {\tt [true]}] A flag to indicate whether intermediate
  detections should be logged.
\item[LogFile {\tt [./duchamp-Logfile]}] The file in which intermediate
  detections are logged. These are detections that have not been
  merged. This is primarily for use in debugging and diagnostic
  purposes -- normal use of the program will probably not require
  this.
\item[flagSubsection {\tt [false]}] A flag to indicate whether one
  wants a subsection of the requested image.
\item[Subsection {\tt [ [*,*,*] ]}] The requested subsection, which
  should be specified in the format {\tt [x1:x2,y1:y2,z1:z2]}, where
  the limits are inclusive. If the full range of a dimension is
  required, use a {\tt *}, \eg if you want the full spectral range of
  a subsection of the image, use {\tt [30:140,30:140,*]}.
\item[flagOutputRecon {\tt [false]}] A flag to say whether or not to
  save the reconstructed cube as a FITS file. The filename will be
  derived from the ImageFile -- the reconstruction of {\tt image.fits}
  will be saved as {\tt image.RECON?.fits}, where {\tt ?} stands for
  the value of {\tt snrRecon} (see below).
\item[flagOutputResid {\tt [false]}] As for {\tt flagOutputRecon}, but
  for the residual array -- the difference between the original cube
  and the reconstructed cube. The filename will be {\tt
  image.RESID?.fits}.
\item[flagVOT {\tt [false]}] A flag to say whether to create a VOTable
  file corresponding to the information in {\tt outfile}. This will be
  an XML file in the Virtual Observatory VOTable format.
\item[votFile {\tt [./duchamp-Results.xml]}] The VOTable file with the
  list of final detections. Some input parameters are also recorded.
\item[flagMaps {\tt [true]}] A flag to say whether to save postscript
  files showing the moment map of the whole cube (parameter {\tt
  momentMap}) and the detection image ({\tt detectionMap}).
\item[momentMap {\tt [./latest-moment-map.ps]}] A postscript file
  containing a map of the 1st moment of the detected sources, as well
  as pixel and WCS coordinates.
\item[detectionMap {\tt [./latest-detection-map.ps]}] A postscript
  file showing each of the detected objects, coloured in greyscale by
  the number of channels they span. Also shows pixel and WCS
  coordinates.
\end{entry}

\subsection*{Modifying the cube}
\begin{entry}
\item[flagBlankPix {\tt [true]}] A flag to say whether to remove BLANK
  pixels from the analysis -- these are pixels set to some particular
  value because they fall outside the imaged area.
\item[blankPixValue {\tt [-8.00061]}] The value of the BLANK pixels, if
  this information is not contained in the FITS header (the usual
  procedure is to obtain this value from the header information -- in
  which case the value set by this parameter is ignored).
\item[flagMW {\tt [true]}] A flag to say whether to remove channels
  contaminated by Milky Way (or other) emission -- the flux in these
  channels is currently just set to 0.
\item[maxMW {\tt [112]}] The maximum channel for the Milky Way
  emission. 
\item[minMW {\tt [75]}] The minimum channel for the Milky Way
  emission. Note that the channels specified by {\tt maxMW} and {\tt
  minMW} are assumed to be Milky Way channels (\ie the range is
  inclusive).
\item[flagBaseline {\tt [false]}] A flag to say whether to remove the
  baseline from each spectrum in the cube for the purposes of
  reconstruction and detection.
\end{entry}

\subsection*{Detection related}

\subsubsection*{General detection}
\begin{entry}
\item[snrCut {\tt [3.]}] The cut-off value for thresholding, in terms
  of number of $\sigma$ above the mean.
\item[minPix {\tt [3]}] The minimum number of pixels for a single
  detection to be counted.
\item[flagGrowth {\tt [true]}] A flag indicating whether or not to
  grow the detected objects to a smaller threshold.
\item[growthCut {\tt [1.5]}] The smaller threshold using in growing
  detections. In units of $\sigma$ above the mean.
\end{entry}

\subsubsection*{{\` a} trous reconstruction}
\begin{entry}
\item [flagATrous {\tt [true]}] A flag indicating whether or not to
  reconstruct the cube using the {\it {\`a} trous} wavelet
  reconstruction. Currently does this in 3-dimensions. See
  \S\ref{sec-recon} for details.
\item[scaleMin {\tt [1]}] The minimum wavelet scale to be used in the
  reconstruction. A value of 1 means ``use all scales''.
\item[snrRecon {\tt [4]}] The thresholding cutoff used in the
  reconstruction -- only wavelet coefficients this many $\sigma$ above
  the mean (or greater) are included in the reconstruction. 
\item[filterCode {\tt [2]}] The code number of the filter to use in
  the reconstruction. The options are:
  \begin{itemize}
  \item {\bf 1:} B$_3$-spline filter: coefficients = 
    $(\frac{1}{16}, \frac{1}{4}, \frac{3}{8}, \frac{1}{4}, \frac{1}{16})$
  \item {\bf 2:} Triangle filter: coefficients = $(\frac{1}{4}, \frac{1}{2}, \frac{1}{4})$
  \item {\bf 3:} Haar wavelet: coefficients = $(0, \frac{1}{2}, \frac{1}{2})$
  \end{itemize}
\end{entry}

\subsubsection*{FDR method}
\begin{entry}
\item[flagFDR {\tt [false]}] A flag indicating whether or not to use
  the False Discovery Rate method in thresholding the pixels.
\item[alphaFDR {\tt [0.01]}] The $\alpha$ parameter used in the FDR
analysis. The average number of false detections, as a fraction of the
total number, will be less than $\alpha$ (see \S\ref{sec-detection}).
\end{entry}

\subsubsection*{Merging detections}
\begin{entry}
\item[flagAdjacent {\tt [true]}] A flag indicating whether to use the
  ``adjacent pixel'' criterion to decide whether to merge objects. If
  not, the next two parameters are used to determine whether objects
  are within the necessary thresholds.
\item[minChannels {\tt [3]}] The minimum number of consecutive
  channels that must be present in the detection for it to be accepted
  by the Merging algorithm.
%The minimum number of channels that a
%  detection must span for it to be accepted by the Merging algorithm.
\item[threshSpatial {\tt [3.]}] The maximum allowed minimum spatial
  separation (in pixels) between two detections for them to be merged
  into one. Only used if {\tt flagAdjacent = false}.
\item[threshVelocity {\tt [7.]}] The maximum allowed minimum channel
  separation between two detections for them to be merged into
  one. %Only used if {\tt flagAdjacent = false}.
\end{entry}

\subsubsection*{Other parameters}
\begin{entry}
\item[drawBorders {\tt [true]}] A flag indicating whether borders
  are to be drawn around the detected objects in the moment maps
  included in the output (see for example Fig.~\ref{fig-spect}).
\item[verbose {\tt [true]}] A flag indicating whether to print the
  progress of computationally-intensive algorithms (such as the
  searching and merging) to screen.
\end{entry}


\newpage
\section{Example parameter files}
\label{app-input}

This is what a typical parameter file would look like.

\begin{verbatim}
imageFile       /DATA/SITAR_1/whi550/cubes/H201_abcde_luther_chop.fits
logFile         temp-Logfile
outFile         temp-Results
spectraFile     spectra.ps
flagSubsection  0
flagOutputRecon 0
flagOutputResid 0
flagBlankPix    1
blankPixValue   -8.00061
flagMW          1
minMW           75
maxMW           112
minPix          3
flagGrowth      1
growthCut       1.5
flagATrous      0
scaleMin        1
snrRecon        4
flagFDR         1
alphaFDR        0.1
numPixPSF       20
snrCut          3
threshSpatial   3
threshVelocity  7
minChannels     4
\end{verbatim}

Note that it is not necessary to include all these parameters in the
file, only those that need to be changed from the defaults (as listed
in Appendix~\ref{app-param}), which in this case would be very few. A
minimal parameter file might look like:
\begin{verbatim}
imageFile       /DATA/SITAR_1/whi550/cubes/H201_abcde_luther_chop.fits
flagLog         0
snrRecon        3
snrCut          2.5
minChannels     3
\end{verbatim}
This will reconstruct the cube with a lower SNR value than the
default, select objects at a lower threshold,  with a looser minimum
channel requirement, and not keep a log of the intermediate
detections. 

The following page demonstrates how the parameters are presented to
the user, both on the screen at execution time and in the output and
log files:
\newpage
\begin{landscape}
Presentation of parameters in output and log files:  
\begin{verbatim}
---- Parameters ---
Image to be analysed                    = /DATA/SITAR_1/whi550/cubes/H201_abcde_luther_chop.fits
Intermediate Logfile                    = temp-Logfile
Final Results file                      = temp-Results
Spectrum file                           = spectra.ps
Saving reconstructed cube?              = false
Saving residuals from reconstruction?   = false
-----
Fixing Blank Pixels?                    = true
Blank Pixel Value                       = -8.00061
Removing Milky Way channels?            = true
Milky Way Channels                      = 75-112
Minimum # Pixels in a detection?        = 3
Growing objects after detection?        = true
SNR Threshold for growth                = 1.5
Using A Trous reconstruction?           = true
Minimum scale in reconstruction         = 1
SNR Threshold within reconstruction     = 4.
Using FDR analysis?                     = false
SNR Threshold                           = 3
Using Adjacent-pixel criterion?         = true
Min. # channels for merging             = 4
-------------------
\end{verbatim}

\newpage
\section{Example output file}
\label{app-output}
This the typical content of an output file, after running Duchamp
with the parameters illustrated on the previous page.

{\scriptsize 
  \begin{verbatim}
Results of  the Duchamp source finder: Mon Nov  7 12:21:30 EST 2005
---- Parameters ---

(... omitted for clarity -- see previous page for examples...)

-------------------
Total number of detections = 25
-------------------
Obj#          Name     X     Y      Z           RA          DEC   RAerr  DECerr       VEL   VELerr  X1  X2  Y1  Y2   Z1   Z2 Npix   F_tot  F_peak
----------------------------------------------------------------------------------------------------------------------------------------------
   1    J0618-2535  31.3  86.2  113.3  06:18:11.47 -25:35:44.35   65.52   46.04   195.690   39.574  25  40  81  92  112  115  216 22.5225  0.3505
   2    J0546-2151 140.1 142.3  114.5  05:46:26.96 -21:51:14.69   39.16   37.26   211.162   39.574 135 144 138 146  113  116   97  4.2503  0.0898
   3    J0609-2159  60.0 140.9  114.8  06:09:28.44 -21:59:35.78   52.65   51.38   214.730   65.957  54  66 135 147  113  118  299 16.7887  0.2125
   4    J0617-2637  33.9  70.9  115.5  06:17:32.93 -26:37:15.00   69.19   33.96   224.250   39.574  26  42  67  75  114  117  195 11.3938  0.1173
   5    J0548-2452 132.9  97.3  118.6  05:48:13.00 -24:52:07.42   31.54   20.84   264.383   39.574 129 136  95  99  117  120   60  2.7168  0.0740
   6    J0606-2722  71.5  60.4  121.1  06:06:21.90 -27:22:19.85   76.49   55.45   297.467   52.765  64  82  53  66  119  123  396 18.7983  0.1497
   7    J0611-2142  52.5 144.9  162.4  06:11:36.41 -21:42:54.01   36.41   23.39   842.940  131.913  48  56 142 147  158  168  357 35.1624  0.4103
   8    J0600-2902  89.6  35.5  202.4  06:00:53.79 -29:02:10.54   31.93   32.11  1370.373  197.870  86  93  32  39  195  210  492 24.0909  0.1729
   9    J0559-2642  95.5  70.2  223.0  05:59:10.70 -26:42:55.41   15.89   20.09  1641.573  118.722  94  97  68  72  219  228   82  2.7609  0.0630
  10    J0617-2727  34.8  58.3  226.9  06:17:24.28 -27:27:57.45   33.02   31.07  1693.404  316.592  32  39  55  62  214  238  368 14.3782  0.0929
  11    J0559-2529  95.6  88.6  230.4  05:59:10.00 -25:29:40.27   35.85   28.18  1739.606  250.635  91  99  86  92  220  239  410 17.5353  0.1155
  12    J0601-2146  88.8 144.3  231.6  06:01:12.10 -21:46:38.18   43.97   28.15  1755.172  237.444  84  94 141 147  221  239  636 30.9721  0.1655
  13    J0615-2639  40.1  70.6  232.6  06:15:43.08 -26:39:08.39   20.67   23.47  1769.149   79.148  38  42  68  73  230  236   98  3.5063  0.0685
  14    J0604-2610  75.9  78.4  233.2  06:04:59.84 -26:10:09.94   28.18   31.84  1776.512  237.444  73  79  75  82  224  242  526 25.1401  0.1545
  15    J0601-2344  87.9 114.9  235.8  06:01:27.47 -23:44:31.98   39.97   36.06  1811.078  263.826  83  92 111 119  226  246 1000 70.9797  0.2968
  16    J0615-2239  38.4 130.5  253.6  06:15:46.60 -22:39:13.99   20.49   19.50  2045.387  131.913  36  40 129 133  247  257   71  2.8341  0.0697
  17    J0617-2308  31.4 123.0  257.7  06:17:51.43 -23:08:36.80   16.57   19.53  2099.724  105.531  30  33 121 125  252  260   63  2.3344  0.0624
  18    J0612-2153  49.6 142.1  271.7  06:12:28.63 -21:53:48.46   32.49   27.44  2284.864  501.270  47  54 139 145  256  294  545 23.1440  0.1326
  19    J0616-2137  35.2 145.8  298.8  06:16:34.71 -21:37:46.18   20.22    7.46  2641.616  145.104  33  37 145 146  293  304   42  2.7988  0.1271
  20    J0555-3000 107.3  20.9  367.5  05:55:27.18 -30:00:15.79   27.61   28.41  3548.362   52.765 104 110  18  24  366  370  120  6.1519  0.1692
  21    J0558-2251  99.6 128.1  434.2  05:58:04.62 -22:51:15.44    7.92   12.08  4428.516   52.765  99 100 127 129  432  436    9  0.8672  0.1674
  22    J0616-2651  37.7  67.3  547.7  06:16:26.63 -26:51:59.99   20.60   19.44  5926.045  131.914  36  40  65  69  545  555   63  2.3717  0.0642
  23    J0608-2243  62.0 130.0  723.5  06:08:56.37 -22:43:15.33    4.05    3.95  8244.473   39.574  62  62 130 130  722  725    4  0.2715  0.0813
  24    J0552-2921 116.9  30.3  726.8  05:52:32.33 -29:21:37.75   11.52   24.25  8287.479  303.400 116 118  28  33  716  739  196 27.9587  0.4787


  \end{verbatim}
}

%\end{landscape}

\newpage
\section{Example VOTable output}
\label{app-votable}
This is part of the VOTable, in XML format, corresponding to the
output file in Appendix~\ref{app-output}.

%\begin{landscape}
{\scriptsize
  \begin{verbatim}
<?xml version="1.0"?>
<VOTABLE version="1.1" xmlns:xsi="http://www.w3.org/2001/XMLSchema-instance"
 xsi:noNamespaceSchemaLocation="http://www.ivoa.net/xml/VOTable/VOTable/v1.1">
  <COOSYS ID="J2000" equinox="J2000." epoch="J2000." system="eq_FK5"/>
  <RESOURCE name="Cubefinder Output">
    <TABLE name="Detections">
      <DESCRIPTION>Detected sources and parameters</DESCRIPTION>
      <PARAM name="FITS file" datatype="char" ucd="meta.file;meta.fits" value="/DATA/MULTI_5/LUTHER/hipass_chop/H201_abcde_luther_chop.fits"/>
      <FIELD name="ID" ID="col1" ucd="meta.id" datatype="int" width="4"/>
      <FIELD name="Name" ID="col2" ucd="meta.id;meta.main" datatype="char" arraysize="14"/>
      <FIELD name="RA" ID="col3" ucd="pos.eq.ra;meta.main" ref="J2000" datatype="float" width="10" precision="6" unit="deg"/>
      <FIELD name="Dec" ID="col4" ucd="pos.eq.dec;meta.main" ref="J2000" datatype="float" width="10" precision="6" unit="deg"/>
      <FIELD name="Vel" ID="col4" ucd="src.veloc" datatype="float" width="9" precision="3" unit="km/s"/>
      <FIELD name="e_Vel" ID="col4" ucd="stat.error;src.veloc" datatype="float" width="8" precision="3" unit="km/s"/>
      <DATA>
        <TABLEDATA>
        <TR>
          <TD>   1</TD><TD>    J0618-2535</TD><TD> 94.547798</TD><TD>-25.595652</TD><TD>  195.690</TD><TD>  39.574</TD>
        </TR>
        <TR>
          <TD>   2</TD><TD>    J0546-2151</TD><TD> 86.612350</TD><TD>-21.854080</TD><TD>  211.162</TD><TD>  39.574</TD>
        </TR>
        <TR>
          <TD>   3</TD><TD>    J0609-2159</TD><TD> 92.368500</TD><TD>-21.993271</TD><TD>  214.730</TD><TD>  65.957</TD>
        </TR>
        <TR>
          <TD>   4</TD><TD>    J0617-2637</TD><TD> 94.387222</TD><TD>-26.620832</TD><TD>  224.250</TD><TD>  39.574</TD>
        </TR>
        <TR>
          <TD>   5</TD><TD>    J0548-2452</TD><TD> 87.054153</TD><TD>-24.868727</TD><TD>  264.383</TD><TD>  39.574</TD>
        </TR>
        <TR>
          <TD>   6</TD><TD>    J0606-2722</TD><TD> 91.591263</TD><TD>-27.372181</TD><TD>  297.467</TD><TD>  52.765</TD>
        </TR>

(... table truncated for clarity ...)

        </TABLEDATA>
      </DATA>
    </TABLE>
  </RESOURCE>
</VOTABLE>
  \end{verbatim}
}
\end{landscape}

\section{Robust statistics for a Normal distribution}
\label{app-madfm}

The Normal, or Gaussian, distribution for mean $\mu$ and standard
deviation $\sigma$ can be written as 
\[ 
f(x) = \frac{1}{\sqrt{2\pi\sigma^2}}\ e^{-(x-\mu)^2/2\sigma^2}.
 \]

When one has a purely Gaussian signal, it is straightforward to
estimate $\sigma$ by calculating the standard deviation (or rms) of
the data. However, if there is a small amount of signal present on top
of Gaussian noise, and one wants to estimate the $\sigma$ for the
noise, the presence of the large values from the signal can bias the
estimator to higher values.

An alternative way is to use the median ($m$) and median absolute deviation
from the median ($s$) to estimate $\mu$ and $\sigma$. The median is the
middle of the distribution, defined for a continuous distribution by
\[
\int_{-\infty}^{m} f(x) \diff x = \int_{m}^{\infty} f(x) \diff x.
\]
From symmetry, we quickly see that for the continuous Normal
distribution, $m=\mu$. We consider the case henceforth of $\mu=0$,
without loss of generality.

To find $s$, we find the distribution of the absolute deviation from
the median, and then find the median of that distribution. This
distribution is given by
\begin{eqnarray*}
g(x) &= &{\mbox{\rm distribution of }} |x|\\
     &= &f(x) + f(-x), x\ge0\\
     &= &\sqrt{\frac{2}{\pi\sigma^2}} e^{-x^2/2\sigma^2}, x\ge0.
\end{eqnarray*}
So to find $s$, we find the median of $g(x)$ via
\[
\int_{0}^{s} g(x) \diff x = \int_{s}^{\infty} g(x) \diff x.
\]
Now, $\int_{0}^{\infty}e^{-x^2/2\sigma^2} \diff x = \sqrt{\pi\sigma^2/2}$, and
so $\int_{s}^{\infty} e^{-x^2/2\sigma^2} \diff x =
\sqrt{\pi\sigma^2/2} - \int_{0}^{s} e^{-\frac{x^2}{2\sigma^2}} \diff x
$. Hence, to find $s$ we simply solve the following equation (setting $\sigma=1$ for
simplicity -- equivalent to stating $x$ and $s$ in units of $\sigma$):
\[
\int_{0}^{s}e^{-x^2/2} \diff x - \sqrt{\pi/8} = 0.
\]
This is hard to solve analytically (no nice analytic solution exists
for the finite integral that I'm aware of), but straightforward to
solve numerically, yielding the value of $s=0.6744888$. Thus, to
estimate $\sigma$ for a Normally distributed data set, one can calculate
$s$, then divide by 0.6744888 (or multiply by 1.4826042) to obtain the
correct estimator.

Note that this is different to solutions quoted elsewhere,
specifically in \citet{meyer04:trunc}, where the same robust estimator is
used but a different value for $s/\sigma$ (of
$\sqrt{2/\pi}\approx0.797885$) is quoted. It should thus be noted that
this means the values quoted by \citet{meyer04:trunc} for the cube noise in
the \hipass\ catalogue should be 18\% larger (since 0.1486042 is 18\%
larger than $\sqrt{\pi/2}\approx1.253314$). 

\end{document}