\documentclass[12pt,a4paper]{article}
%\documentclass[12pt,a4paper]{report}

%%%%%% LINE SPACING %%%%%%%%%%%%
\usepackage{setspace}
\singlespacing
%\onehalfspacing
%\doublespacing

%Define a test for doing PDF format -- use different code below
\newif\ifPDF
\ifx\pdfoutput\undefined\PDFfalse 
\else\ifnum\pdfoutput > 0\PDFtrue 
     \else\PDFfalse 
     \fi 
\fi

\textwidth=165 mm
\textheight=245 mm
\topmargin=0 mm
\oddsidemargin=0 mm
\parindent=6 mm

\usepackage[sort]{natbib}
\usepackage{lscape}
\bibpunct[,]{(}{)}{;}{a}{}{,}

%\newcommand{\secA}{\chapter}
%\newcommand{\secB}{\section}
%\newcommand{\secC}{\subsection}

\newcommand{\secA}{\section}
\newcommand{\secB}{\subsection}
\newcommand{\secC}{\subsubsection}


\newcommand{\eg}{e.g.\ }
\newcommand{\ie}{i.e.\ }
\newcommand{\hi}{H\textsc{i}}
\newcommand{\hipass}{\textsc{hipass}}
\newcommand{\duchamp}{\emph{Duchamp}}
\newcommand{\atrous}{\textit{{\`a} trous}}
\newcommand{\Atrous}{\textit{{\`A} trous}}
\newcommand{\diff}{{\rm d}}
\newcommand{\entrylabel}[1]{\mbox{\textsf{\bf{#1:}}}\hfil}
\newenvironment{entry}
        {\begin{list}{}%
                {\renewcommand{\makelabel}{\entrylabel}%
                        \setlength{\labelwidth}{30mm}%
                        \setlength{\labelsep}{5pt}%
                        \setlength{\itemsep}{2pt}%
                        \setlength{\parsep}{2pt}%
                        \setlength{\leftmargin}{35mm}%
                }%
        }%
{\end{list}}

%%%%%%% FANCY HEADINGS %%%%%%%%%%
%\usepackage{fancyheadings}
%\pagestyle{fancyplain}
%\addtolength{\headheight}{1.6pt}
%\addtolength{\headwidth}{\marginparsep}
%\addtolength{\headwidth}{\marginparwidth}
%\renewcommand{\secAmark}[1]
%                {\markboth{#1}{}}
%\renewcommand{\secBmark}[1]
%                {\markright{\thesection\ #1}}
%\lhead[\fancyplain{}{\bfseries\thepage}]
%      {\fancyplain{}{\bfseries\rightmark}}
%\rhead[\fancyplain{}{\bfseries\leftmark}]
%      {\fancyplain{}{\bfseries\thepage}}
%\cfoot{}

% If we are creating a PDF, use different options for graphicx, hyperref.
\ifPDF
  \usepackage[pdftex]{graphicx,color}
  \usepackage[pdftex,bookmarks]{hyperref}
  \hypersetup{colorlinks=true,% 	    
              citecolor=red,%
              filecolor=red,%
              linkcolor=red,%
              urlcolor=red,%
              }
\else
  \usepackage[dvips]{graphicx} 
  \usepackage[dvips]{hyperref} 
\fi

\pagestyle{headings}
\begin{document}

\ifPDF
\pdfbookmark[2]{Title Page}{title}
\fi
\newlength{\centreoff}
\setlength{\centreoff}{-0.5\oddsidemargin}
\addtolength{\centreoff}{0.5\evensidemargin}
\thispagestyle{empty}
\vspace*{\stretch{1}}
\noindent\hspace*{\centreoff}%
\makebox[0pt][l]{\begin{minipage}{\textwidth}
\flushright
{\Huge\bfseries Source Detection with \duchamp\\%
A User's Guide

}
\noindent\rule[-1ex]{\textwidth}{3pt}
\end{minipage}}

\vspace{\stretch{1}}
\noindent\hspace*{\centreoff}%
\makebox[0pt][l]{\begin{minipage}{\textwidth}
\flushright
{\bfseries 
Matthew Whiting\\
Australia Telescope National Facility\\
CSIRO\\[3ex]} 
Duchamp version~1.05\\
September 15, 2006
\end{minipage}}

\begin{figure}[!h]
%\begin{center}
\includegraphics[width=\textwidth]{cover_image}
%\end{center}
\end{figure}

%\newpage
\ifPDF
\pdfbookmark[2]{Contents}{contents}
\fi
\tableofcontents

%\newpage
% -----------------------------------------------------------------------
% intro.tex: Introduction, and guide to what Duchamp is doing.
% -----------------------------------------------------------------------
% Copyright (C) 2006, Matthew Whiting, ATNF
%
% This program is free software; you can redistribute it and/or modify it
% under the terms of the GNU General Public License as published by the
% Free Software Foundation; either version 2 of the License, or (at your
% option) any later version.
%
% Duchamp is distributed in the hope that it will be useful, but WITHOUT
% ANY WARRANTY; without even the implied warranty of MERCHANTABILITY or
% FITNESS FOR A PARTICULAR PURPOSE.  See the GNU General Public License
% for more details.
%
% You should have received a copy of the GNU General Public License
% along with Duchamp; if not, write to the Free Software Foundation,
% Inc., 59 Temple Place, Suite 330, Boston, MA 02111-1307, USA
%
% Correspondence concerning Duchamp may be directed to:
%    Internet email: Matthew.Whiting [at] atnf.csiro.au
%    Postal address: Dr. Matthew Whiting
%                    Australia Telescope National Facility, CSIRO
%                    PO Box 76
%                    Epping NSW 1710
%                    AUSTRALIA
% -----------------------------------------------------------------------
\secA{Introduction and getting going quickly}

\secB{About \duchamp}

This document provides a user's guide to \duchamp, an object-finder
for use on spectral-line data cubes. The basic execution of \duchamp
is to read in a FITS data cube, find sources in the cube, and produce
a text file of positions, velocities and fluxes of the detections, as
well as a postscript file of the spectra of each detection.

\duchamp has been designed to search for objects in particular sorts
of data: those with relatively small, isolated objects in a large
amount of background or noise. Examples of such data are extragalactic
\hi surveys, or maser surveys. \duchamp searches for groups of
connected voxels (or pixels) that are all above some flux
threshold. No assumption is made as to the shape of detections, and
the only size constraints applied are those specified by the user.

\duchamp has been written as a three-dimensional finder, but it is
possible to run it on a two-dimensional image (\ie one with no
frequency or velocity information), or indeed a one-dimensional array,
and many of the features of the program will work fine. The focus,
however, is on object detection in three dimensions, one of which is a
spectral dimension. Note, in particular, that it does not do any
fitting of source profiles, a feature common (and desirable) for many
two-dimensional source finders. This is beyond the current scope of
\duchamp, whose aim is reliable detection of spectral-line objects.

\duchamp provides the ability to pre-process the data prior to
searching. Spectral baselines can be removed, and either smoothing or
multi-resolution wavelet reconstruction can be performed to enhance
the completeness and reliability of the resulting catalogue.

\secB{Acknowledging the use of \duchamp}

\duchamp is provided in the hope that it will be useful for your
research. If you find that it is, I would ask that you acknowledge it
in your publication by using the following:
"This research made use of the Duchamp source finder, produced at
the Australia Telescope National Facility, CSIRO, by M. Whiting."

Additionally, \duchamp has been described in a journal paper
\citep{whiting12}. This paper covers the key algorithms implemented in
the software, and provides some simple completeness and reliability
comparisons of different modes of operation. Users of \duchamp are
encouraged to read the paper in conjunction with this user guide, as
while some things are repeated herein, not everything
is. \citet{whiting12} should be cited when describing the use of
\duchamp in your research.



\secB{What to do}

So, you have a FITS cube, and you want to find the sources in it. What
do you do? First, you need to get \duchamp: there are instructions in
Appendix~\ref{app-install} for obtaining and installing it. Once you
have it running, the first step is to make an input file that contains
the list of parameters. Brief and detailed examples are shown in
Appendix~\ref{app-input}. This file provides the input file name, the
various output files, and defines various parameters that control the
execution.

The standard way to run \duchamp is by the command
\begin{quote}
{\footnotesize
\texttt{> Duchamp -p [parameter file]}
}
\end{quote}
replacing \texttt{[parameter file]} with the name of the file listing
the parameters. 

An even easier way is to use the default values for all parameters
(these are given in Appendix~\ref{app-param} and in the file
\texttt{InputComplete} included in the distribution directory) and use
the syntax
\begin{quote}
{\footnotesize
\texttt{> Duchamp -f [FITS file]}
}
\end{quote}
where \texttt{[FITS file]} is the file you wish to search. 

The default action includes displaying a map of detected objects in a
PGPLOT X-window. This can be disabled by setting the parameter
\texttt{flagXOutput = false} or using the \texttt{-x} command-line
option, as in
\begin{quote}
{\footnotesize
\texttt{> Duchamp -x -p [parameter file]}
}
\end{quote}
and similarly for the \texttt{-f} case.

Once a FITS file and parameters have been set, the program will then
work away and give you the list of detections and their spectra. The
program execution is summarised below, and detailed in
\S\ref{sec-flow}. Information on inputs is in \S\ref{sec-param} and
Appendix~\ref{app-param}, and descriptions of the output is in
\S\ref{sec-output}.

\secB{Guide to terminology and conventions}

First, a brief note on the use of terminology in this guide. \duchamp
is designed to work on FITS ``cubes''. These are FITS\footnote{FITS is
the Flexible Image Transport System -- see \citet{hanisch01} or
websites such as
\href{http://fits.cv.nrao.edu/FITS.html}{http://fits.cv.nrao.edu/FITS.html}
for details.} image arrays with (at least) three dimensions. They
are assumed to have the following form: the first two dimensions
(referred to as $x$ and $y$) are spatial directions (that is, relating
to the position on the sky -- often, but not necessarily,
corresponding to Equatorial or Galactic coordinates), while the third
dimension, $z$, is the spectral direction, which can correspond to
frequency, wavelength, or velocity. The three dimensional analogue of
pixels are ``voxels'', or volume cells -- a voxel is defined by a
unique $(x,y,z)$ location and has a single value of flux, intensity
or brightness (or something equivalent) associated with it.

Sometimes, some pixels in a FITS file are labelled as BLANK -- that
is, they are given a nominal value, defined by FITS header keywords
\textsc{blank} (and potentially \textsc{bscale} and \textsc{bzero}),
that marks them as not having a flux value. These are often used to
pad a cube out so that it has a rectangular spatial shape. \duchamp
has the ability to avoid these: see \S\ref{sec-blank}.

Note that it is possible for the FITS file to have more than three
dimensions (for instance, there could be a fourth dimension
representing a Stokes parameter). Only the two spatial dimensions and
the spectral dimension are read into the array of pixel values that is
searched for objects. All other dimensions are ignored\footnote{This
actually means that the first pixel only of that axis is used, and the
array is read by the \texttt{fits\_read\_subsetnull} command from the
\textsc{cfitsio} library.}. Herein, we discuss the data in terms of
the three basic dimensions, but you should be aware it is possible for
the FITS file to have more than three. Note that the order of the
dimensions in the FITS file does not matter.

With this setup, each spatial pixel (a given $(x,y)$ coordinate) can
be said to be a single spectrum, while a slice through the cube
perpendicular to the spectral direction at a given $z$-value is a
single channel, with the 2-D image in that channel called a channel
map.

Detection involves locating contiguous groups of voxels with fluxes
above a certain threshold. \duchamp makes no assumptions as to the
size or shape of the detected features, other than allowing
user-selected minimum or maximum size criteria. Features that are
detected are assumed to be positive. The user can choose to search for
negative features by setting an input parameter -- which will invert
the cube prior to the search (see \S\ref{sec-searchTechnique} for
details).

\secB{A summary of the execution steps}

The basic flow of the program is summarised here -- all steps are
discussed in more detail in the following sections.
\begin{enumerate}
\item The necessary parameters are recorded.

  How this is done depends on the way the program is run from the
  command line. If the \texttt{-p} option is used, the parameter file
  given on the command line is read in, and the parameters therein are
  read. All other parameters are given their default values (listed in
  Appendix~\ref{app-param}).

  If the \texttt{-f} option is used, all parameters are assigned their
  default values, with the flux threshold able to be set with the
  \texttt{-t} option.

\item The FITS image is located and read in to memory.

  The file given is assumed to be a valid FITS file. As discussed
  above, it can have any number of dimensions, but \duchamp only
  reads in the two spatial and the one spectral dimensions. A subset
  of the FITS array can be given (see \S\ref{sec-input} for details).

\item \label{step-reuse} If requested, a FITS file containing a
  previously reconstructed or smoothed array is read in.

  When a cube is either smoothed or reconstructed with the \atrous
  wavelet method, the result can be saved to a FITS file, so that
  subsequent runs of \duchamp can read it in to save having to re-do
  the calculations.

\item \label{step-blank} If requested, BLANK pixels are trimmed from
  the edges, and the baseline of each spectrum is removed.

  BLANK pixels, while they are ignored by all calculations in
  \duchamp, do increase the size in memory of the array above that
  absolutely needed. This step trims them from the spatial edges,
  keeping a record of the amount trimmed so that they can be added
  back in later.

  A spectral baseline (or bandpass) may optionally be removed at this
  point as well. This may be necessary if there is a ripple or other
  large-scale feature present that will hinder detection of faint
  sources.

\item If the reconstruction method is requested, and the reconstructed
  array has not been read in at Step 3 above, the cube is
  reconstructed using the \atrous wavelet method.

  This step uses the multi-resolution \atrous method to determine the
  amount of structure present at various scales. A simple thresholding
  technique then removes random noise from the cube, leaving the
  significant signal. This process can greatly reduce the noise level
  in the cube, enhancing the reliability of the resulting catalogue.

\item Alternatively (and if requested), the cube is smoothed, either
  spectrally or spatially.

  This step presents two options. The first considers each spectrum
  individually, and convolves it with a Hanning filter (with width
  chosen by the user). The second considers each channel map
  separately, and smoothes it with a Gaussian kernel of size and shape
  chosen by the user. This step can help to reduce the amount of noise
  visible in the cube and enhance fainter sources, increasing the
  completeness and reliability of the output catalogue.

\item A threshold for the cube is then calculated, based on the pixel
  statistics (unless a threshold is manually specified by the user).

  The threshold can either be chosen as a simple $n\sigma$ threshold
  (\ie a certain number of standard deviations above the mean), or
  calculated via the ``False Discovery Rate'' method. Alternatively,
  the threshold can be specified as a simple flux value, without care
  as to the statistical significance (\eg ``I want every source
  brighter than 10mJy'').

  By default, the full cube is used for the statistics calculation,
  although the user can nominate a subsection of the cube to be used
  instead. 

\item Searching for objects then takes place, using the requested
  thresholding method.

  The cube is searched either one channel-map at a time (``spatial''
  search) or one spectrum at a time (``spectral'' search). Detections
  are compared to already detected objects and either combined with a
  neighbouring one or added to the end of the list.

\item The list of objects is condensed by merging neighbouring objects
  and removing those deemed unacceptable.

  While some merging has been done in the previous step, this process
  is a much more rigorous comparison of each object with every other
  one. If a pair of objects lie within requested limits, they are
  combined. 

  After the merging is done, the list is culled (although see comment
  for the next step). There are certain criteria the user can specify
  that objects must meet: minimum (or maximum) numbers of spatial
  pixels and spectral channels, and minimum separations between
  neighbouring objects. Those that do not meet these criteria are
  deleted from the list.

\item If requested, the objects are ``grown'' down to a lower
  threshold, and then the merging step is done a second time.

  In this case, each object has pixels in its neighbourhood examined,
  and if they are above a secondary threshold, they are added to the
  object. The merging process is done a second time in case two
  objects have grown over the top of one another. Note that the
  rejection part of the previous step is not done until the end of the
  second merging process.

\item The baselines and trimmed pixels are replaced prior to output.

  This is just the inverse of step~\#\ref{step-blank}.

\item The details of the detections are written to screen and to the
  requested output file.

  Crucial properties of each detection are provided, showing its
  location, extent, and flux. These are presented in both pixel
  coordinates and world coordinates (\eg sky position and
  velocity). Any warning flags are also printed, showing detections to
  be wary of. Alternative output options are available, such as a
  VOTable or annotation files for image viewers such as kvis, ds9 or
  casaviewer. 

\item Maps showing the spatial location of the detections are written.

  These are 2-dimensional maps, showing where each detection lies on
  the spatial coverage of the cube. This is provided as an aid to the
  user so that a quick idea of the distribution of object positions
  can be gained \eg are all the detections on the edge?

  Two maps are provided: one is a 0th moment map, showing the 0th
  moment (\ie a map of the integrated flux) of each detection in its
  appropriate position, while the second is a ``detection map'',
  showing the number of times each spatial pixel was detected in the
  searching routines (including those pixels rejected at step 9 and so
  not in any of the final detections).

  These maps are written to postscript files, and the 0th moment map
  can also be displayed in a PGPLOT X-window.

\item A pixel mask is written to a FITS file.

  A FITS file of the same size as the input file can be written. Here,
  each pixel has a value indicating whether or note it was detected
  and falls in one of the catalogue sources. Different objects can be
  traced by different non-zero pixel values.

\item The integrated or peak spectra of each detection are written to a
  postscript file. 

  The spectral equivalent of the maps -- what is the spectral profile
  of each detection? Also provided here are basic information for each
  object (a summary of the information in the results file), as well
  as a 0th moment map of the detection.

\item If requested, a text file containing all spectra is written.

  This file will contain the peak or integrated spectra for each
  source, as a function of the appropriate spectral coordinate. The
  file is a multi-column ascii text file, suitable for import into
  other software packages. 

\item If requested, FITS files are written containing the
  reconstructed, smoothed, baseline or mask arrays.

  If one of the preprocessing methods was used, the resulting array
  can be saved as a FITS file for later examination or use (for
  instance, reading in as described at step \#\ref{step-reuse}). The
  FITS header will be the same as the input file, with a few
  additional keywords to identify the file.

\end{enumerate}

\secB{Why ``\duchamp''?}

Well, it's important for a program to have a name, and the initial
working title of \emph{cubefind} was somewhat uninspiring. I wanted to
avoid the classic astronomical approach of designing a cute acronym,
and since it is designed to work on cubes, I looked at naming it after
a cubist. \emph{Picasso}, sadly, was already taken \citep{minchin99},
so I settled on naming it after Marcel Duchamp, another cubist, but
also one of the first artists to work with ``found objects''.


%%% Local Variables: 
%%% mode: latex
%%% TeX-master: "Guide"
%%% End: 
 %Introduction and getting going quickly
\newpage
\secA{User Inputs}
\label{sec-param}

Input to the program is provided by means of a parameter
file. Parameters are listed in the file, followed by the value that
should be assigned to them. The syntax used is \texttt{paramName
value}. Parameter names are not case-sensitive, and lines in the input
file that start with \texttt{\#} are ignored. If a parameter is listed
more than once, the latter value is used, but otherwise the order in
which the parameters are listed in the input file is
arbitrary. Example input files can be seen in
Appendix~\ref{app-input}.

If a parameter is not listed, the default value is assumed. The
defaults are chosen to provide a good result (using the reconstruction
method), so the user doesn't need to specify many new parameters in
the input file. Note that the image file \textbf{must} be specified!
The parameters that can be set are listed in Appendix~\ref{app-param},
with their default values in parentheses.

The parameters with names starting with \texttt{flag} are stored as
\texttt{bool} variables, and so are either \texttt{true = 1} or
\texttt{false = 0}. They can be entered in the file either in text or
integer format -- \duchamp\ will read them correctly in either case.

An example input file is included in the distribution tar file. It is
as follows:

\begin{verbatim}
imageFile       your-file-here
logFile         logfile.txt
outFile         results.txt
spectraFile     spectra.ps
minPix          2
flagATrous      1
snrRecon        5.
snrCut          3.
minChannels     3
flagBaseline    1
\end{verbatim}

You would, of course, replace the ``\texttt{your-file-here}'' with the
FITS file you wanted to search. Further examples are given in
Appendix~\ref{app-input}.
 %User Inputs
\newpage
% -----------------------------------------------------------------------
% executionFlow.tex: Section detailing each of the main algorithms
%                    used by Duchamp.
% -----------------------------------------------------------------------
% Copyright (C) 2006, Matthew Whiting, ATNF
%
% This program is free software; you can redistribute it and/or modify it
% under the terms of the GNU General Public License as published by the
% Free Software Foundation; either version 2 of the License, or (at your
% option) any later version.
%
% Duchamp is distributed in the hope that it will be useful, but WITHOUT
% ANY WARRANTY; without even the implied warranty of MERCHANTABILITY or
% FITNESS FOR A PARTICULAR PURPOSE.  See the GNU General Public License
% for more details.
%
% You should have received a copy of the GNU General Public License
% along with Duchamp; if not, write to the Free Software Foundation,
% Inc., 59 Temple Place, Suite 330, Boston, MA 02111-1307, USA
%
% Correspondence concerning Duchamp may be directed to:
%    Internet email: Matthew.Whiting [at] atnf.csiro.au
%    Postal address: Dr. Matthew Whiting
%                    Australia Telescope National Facility, CSIRO
%                    PO Box 76
%                    Epping NSW 1710
%                    AUSTRALIA
% -----------------------------------------------------------------------
\secA{What \duchamp is doing}
\label{sec-flow}

Each of the steps that \duchamp goes through in the course of its
execution are discussed here in more detail. This should provide
enough background information to fully understand what \duchamp is
doing and what all the output information is. For those interested in
the programming side of things, \duchamp is written in C/C++ and makes
use of the \textsc{cfitsio}, \textsc{wcslib} and \textsc{pgplot}
libraries.

\secB{Image input}
\label{sec-input}

The cube is read in using basic \textsc{cfitsio} commands, and stored
as an array in a special C++ class. This class keeps track of the list
of detected objects, as well as any reconstructed arrays that are made
(see \S\ref{sec-recon}). The World Coordinate System
(WCS)\footnote{This is the information necessary for translating the
pixel locations to quantities such as position on the sky, frequency,
velocity, and so on.} information for the cube is also obtained from
the FITS header by \textsc{wcslib} functions \citep{greisen02,
calabretta02}, and this information, in the form of a \texttt{wcsprm}
structure, is also stored in the same class.

A sub-section of a cube can be requested by defining the subsection
with the \texttt{subsection} parameter and setting
\texttt{flagSubsection = true} -- this can be a good idea if the cube
has very noisy edges, which may produce many spurious detections.

There are two ways of specifying the \texttt{subsection} string. The
first is the generalised form
\texttt{[x1:x2:dx,y1:y2:dy,z1:z2:dz,...]}, as used by the
\textsc{cfitsio} library. This has one set of colon-separated numbers
for each axis in the FITS file. In this manner, the x-coordinates run
from \texttt{x1} to \texttt{x2} (inclusive), with steps of
\texttt{dx}. The step value can be omitted, so a subsection of the
form \texttt{[2:50,2:50,10:1000]} is still valid. In fact, \duchamp
does not make use of any step value present in the subsection string,
and any that are present are removed before the file is opened.

If the entire range of a coordinate is required, one can replace the
range with a single asterisk, \eg \texttt{[2:50,2:50,*]}. Thus, the
subsection string \texttt{[*,*,*]} is simply the entire cube. Note
that the pixel ranges for each axis start at 1, so the full pixel
range of a 100-pixel axis would be expressed as 1:100. A complete
description of this section syntax can be found at the
\textsc{fitsio} web site%
\footnote{%
\href%
{http://heasarc.gsfc.nasa.gov/docs/software/fitsio/c/c\_user/node91.html}%
{http://heasarc.gsfc.nasa.gov/docs/software/fitsio/c/c\_user/node91.html}}.


Making full use of the subsection requires knowledge of the size of
each of the dimensions. If one wants to, for instance, trim a certain
number of pixels off the edges of the cube, without examining the cube
to obtain the actual size, one can use the second form of the
subsection string. This just gives a number for each axis, \eg
\texttt{[5,5,5]} (which would trim 5 pixels from the start \emph{and}
end of each axis).

All types of subsections can be combined \eg \texttt{[5,2:98,*]}. 

Typically, the units of pixel brightness are given by the FITS file's
BUNIT keyword. However, this may often be unwieldy (for instance, the
units are Jy/beam, but the values are around a few mJy/beam). It is
therefore possible to nominate new units, to which the pixel values
will be converted, by using the \texttt{newFluxUnits} input
parameter. The units must be directly translatable from the existing
ones -- for instance, if BUNIT is Jy/beam, you cannot specify mJy, it
must be mJy/beam. If an incompatible unit is given, the BUNIT value is
used instead.

\secB{Image modification}
\label{sec-modify}

Several modifications to the cube can be made that improve the
execution and efficiency of \duchamp (their use is optional, governed
by the relevant flags in the parameter file).

\secC{BLANK pixel removal}
\label{sec-blank}

If the imaged area of a cube is non-rectangular (see the example in
Fig.~\ref{fig-moment}, a cube from the HIPASS survey), BLANK pixels
are used to pad it out to a rectangular shape. The value of these
pixels is given by the FITS header keywords BLANK, BSCALE and
BZERO. While these pixels make the image a nice shape, they will take
up unnecessary space in memory, and so to potentially speed up the
processing we can trim them from the edge. This is done when the
parameter \texttt{flagTrim = true}. If the above keywords are not
present, the trimming will not be done (in this case, a similar effect
can be accomplished, if one knows where the ``blank'' pixels are, by
using the subsection option).

The amount of trimming is recorded, and these pixels are added back in
once the source-detection is completed (so that quoted pixel positions
are applicable to the original cube). Rows and columns are trimmed one
at a time until the first non-BLANK pixel is reached, so that the
image remains rectangular. In practice, this means that there will be
some BLANK pixels left in the trimmed image (if the non-BLANK region
is non-rectangular). However, these are ignored in all further
calculations done on the cube.

\secC{Baseline removal}

Second, the user may request the removal of baselines from the
spectra, via the parameter \texttt{flagBaseline}. This may be
necessary if there is a strong baseline ripple present, which can
result in spurious detections at the high points of the ripple. The
baseline is calculated from a wavelet reconstruction procedure (see
\S\ref{sec-recon}) that keeps only the two largest scales. This is
done separately for each spatial pixel (\ie for each spectrum in the
cube), and the baselines are stored and added back in before any
output is done. In this way the quoted fluxes and displayed spectra
are as one would see from the input cube itself -- even though the
detection (and reconstruction if applicable) is done on the
baseline-removed cube.

The presence of very strong signals (for instance, masers at several
hundred Jy) could affect the determination of the baseline, and would
lead to a large dip centred on the signal in the baseline-subtracted
spectrum. To prevent this, the signal is trimmed prior to the
reconstruction process at some standard threshold (at $8\sigma$ above
the mean). The baseline determined should thus be representative of
the true, signal-free baseline. Note that this trimming is only a
temporary measure which does not affect the source-detection.

\secC{Ignoring bright Milky Way emission}
\label{sec-MW}

Finally, a single set of contiguous channels can be ignored -- these
may exhibit very strong emission, such as that from the Milky Way as
seen in extragalactic \hi cubes (hence the references to ``Milky
Way'' in relation to this task -- apologies to Galactic
astronomers!). Such dominant channels will produce many detections
that are unnecessary, uninteresting (if one is interested in
extragalactic \hi) and large (in size and hence in memory usage), and
so will slow the program down and detract from the interesting
detections. 

The use of this feature is controlled by the \texttt{flagMW}
parameter, and the exact channels concerned are able to be set by the
user (using \texttt{maxMW} and \texttt{minMW} -- these give an
inclusive range of channels). When employed, these channels are
ignored for the searching, and the scaling of the spectral output (see
Fig.~\ref{fig-spect}) will not take them into account. They will be
present in the reconstructed array, however, and so will be included
in the saved FITS file (see \S\ref{sec-reconIO}). When the final
spectra are plotted, the range of channels covered by these parameters
is indicated by a green hashed box.

\secB{Image reconstruction}
\label{sec-recon}

The user can direct \duchamp to reconstruct the data cube using the
\atrous wavelet procedure. A good description of the procedure can be
found in \citet{starck02:book}. The reconstruction is an effective way
of removing a lot of the noise in the image, allowing one to search
reliably to fainter levels, and reducing the number of spurious
detections. This is an optional step, but one that greatly enhances
the source-detection process, with the payoff that it can be
relatively time- and memory-intensive.

\secC{Algorithm}

The steps in the \atrous reconstruction are as follows:
\begin{enumerate}
\item The reconstructed array is set to 0 everywhere.
\item The input array is discretely convolved with a given filter
  function. This is determined from the parameter file via the
  \texttt{filterCode} parameter -- see Appendix~\ref{app-param} for
  details on the filters available.
\item The wavelet coefficients are calculated by taking the difference
  between the convolved array and the input array.
\item If the wavelet coefficients at a given point are above the
  requested threshold (given by \texttt{snrRecon} as the number of
  $\sigma$ above the mean and adjusted to the current scale -- see
  Appendix~\ref{app-scaling}), add these to the reconstructed array.
\item The separation between the filter coefficients is doubled. (Note
  that this step provides the name of the procedure\footnote{\atrous
  means ``with holes'' in French.}, as gaps or holes are created in
  the filter coverage.)
\item The procedure is repeated from step 2, using the convolved array
  as the input array.
\item Continue until the required maximum number of scales is reached.
\item Add the final smoothed (\ie convolved) array to the
  reconstructed array. This provides the ``DC offset'', as each of the
  wavelet coefficient arrays will have zero mean.
\end{enumerate}

The range of scales at which the selection of wavelet coefficients is
made is governed by the \texttt{scaleMin} and \texttt{scaleMax}
parameters. The minimum scale used is given by \texttt{scaleMin},
where the default value is 1 (the first scale). This parameter is
useful if you want to ignore the highest-frequency features
(e.g. high-frequency noise that might be present). Normally the
maximum scale is calculated from the size of the input array, but it
can be specified by using \texttt{scaleMax}. A value $\le0$ will
result in the use of the calculated value, as will a value of
\texttt{scaleMax} greater than the calculated value. Use of these two
parameters can allow searching for features of a particular scale
size, for instance searching for narrow absorption features.

The reconstruction has at least two iterations. The first iteration
makes a first pass at the wavelet reconstruction (the process outlined
in the 8 stages above), but the residual array will likely have some
structure still in it, so the wavelet filtering is done on the
residual, and any significant wavelet terms are added to the final
reconstruction. This step is repeated until the change in the measured
standard deviation of the background (see note below on the evaluation
of this quantity) is less than some fiducial amount.

It is important to note that the \atrous decomposition is an example
of a ``redundant'' transformation. If no thresholding is performed,
the sum of all the wavelet coefficient arrays and the final smoothed
array is identical to the input array. The thresholding thus removes
only the unwanted structure in the array.

Note that any BLANK pixels that are still in the cube will not be
altered by the reconstruction -- they will be left as BLANK so that
the shape of the valid part of the cube is preserved.

\secC{Note on Statistics}

The correct calculation of the reconstructed array needs good
estimators of the underlying mean and standard deviation (or rms) of
the background noise distribution. The methods used to estimate these
quantities are detailed in \S\ref{sec-stats} -- the default behaviour
is to use robust estimators, to avoid biasing due to bright pixels.

%These statistics are estimated using
%robust methods, to avoid corruption by strong outlying points. The
%mean of the distribution is actually estimated by the median, while
%the median absolute deviation from the median (MADFM) is calculated
%and corrected assuming Gaussianity to estimate the underlying standard
%deviation $\sigma$. The Gaussianity (or Normality) assumption is
%critical, as the MADFM does not give the same value as the usual rms
%or standard deviation value -- for a Normal distribution
%$N(\mu,\sigma)$ we find MADFM$=0.6744888\sigma$, but this will change
%for different distributions. Since this ratio is corrected for, the
%user need only think in the usual multiples of the rms when setting
%\texttt{snrRecon}. See Appendix~\ref{app-madfm} for a derivation of
%this value.

When thresholding the different wavelet scales, the value of the rms
as measured from the wavelet array needs to be scaled to account for
the increased amount of correlation between neighbouring pixels (due
to the convolution). See Appendix~\ref{app-scaling} for details on
this scaling.

\secC{User control of reconstruction parameters}

The most important parameter for the user to select in relation to the
reconstruction is the threshold for each wavelet array. This is set
using the \texttt{snrRecon} parameter, and is given as a multiple of
the rms (estimated by the MADFM) above the mean (which for the wavelet
arrays should be approximately zero). There are several other
parameters that can be altered as well that affect the outcome of the
reconstruction.

By default, the cube is reconstructed in three dimensions, using a
3-dimensional filter and 3-dimensional convolution. This can be
altered, however, using the parameter \texttt{reconDim}. If set to 1,
this means the cube is reconstructed by considering each spectrum
separately, whereas \texttt{reconDim=2} will mean the cube is
reconstructed by doing each channel map separately. The merits of
these choices are discussed in \S\ref{sec-notes}, but it should be
noted that a 2-dimensional reconstruction can be susceptible to edge
effects if the spatial shape of the pixel array is not rectangular.

The user can also select the minimum scale to be used in the
reconstruction. The first scale exhibits the highest frequency
variations, and so ignoring this one can sometimes be beneficial in
removing excess noise. The default is to use all scales
(\texttt{minscale = 1}).

Finally, the filter that is used for the convolution can be selected
by using \texttt{filterCode} and the relevant code number -- the
choices are listed in Appendix~\ref{app-param}. A larger filter will
give a better reconstruction, but take longer and use more memory when
executing. When multi-dimensional reconstruction is selected, this
filter is used to construct a 2- or 3-dimensional equivalent.

\secB{Smoothing the cube}
\label{sec-smoothing}

An alternative to doing the wavelet reconstruction is to smooth the
cube.  This technique can be useful in reducing the noise level
slightly (at the cost of making neighbouring pixels correlated and
blurring any signal present), and is particularly well suited to the
case where a particular signal size (\ie a certain channel width or
spatial size) is believed to be present in the data.

There are two alternative methods that can be used: spectral
smoothing, using the Hanning filter; or spatial smoothing, using a 2D
Gaussian kernel. These alternatives are outlined below. To utilise the
smoothing option, set the parameter \texttt{flagSmooth=true} and set
\texttt{smoothType} to either \texttt{spectral} or \texttt{spatial}.

\secC{Spectral smoothing}

When \texttt{smoothType = spectral} is selected, the cube is smoothed
only in the spectral domain. Each spectrum is independently smoothed
by a Hanning filter, and then put back together to form the smoothed
cube, which is then used by the searching algorithm (see below). Note
that in the case of both the reconstruction and the smoothing options
being requested, the reconstruction will take precedence and the
smoothing will \emph{not} be done.

There is only one parameter necessary to define the degree of
smoothing -- the Hanning width $a$ (given by the user parameter
\texttt{hanningWidth}). The coefficients $c(x)$ of the Hanning filter
are defined by
\[
c(x) = 
  \begin{cases}
   \frac{1}{2}\left(1+\cos(\frac{\pi x}{a})\right) &|x| \leq (a+1)/2\\
   0                                               &|x| > (a+1)/2.
  \end{cases},\ a,x \in \mathbb{Z}
\]
Note that the width specified must be an
odd integer (if the parameter provided is even, it is incremented by
one).

\secC{Spatial smoothing}

When \texttt{smoothType = spatial} is selected, the cube is smoothed
only in the spatial domain. Each channel map is independently smoothed
by a two-dimensional Gaussian kernel, put back together to form the
smoothed cube, and used in the searching algorithm (see below). Again,
reconstruction is always done by preference if both techniques are
requested.

The two-dimensional Gaussian has three parameters to define it,
governed by the elliptical cross-sectional shape of the Gaussian
function: the FWHM (full-width at half-maximum) of the major and minor
axes, and the position angle of the major axis. These are given by the
user parameters \texttt{kernMaj, kernMin} \& \texttt{kernPA}. If a
circular Gaussian is required, the user need only provide the
\texttt{kernMaj} parameter. The \texttt{kernMin} parameter will then
be set to the same value, and \texttt{kernPA} to zero.  If we define
these parameters as $a,b,\theta$ respectively, we can define the
kernel by the function
\[ 
k(x,y) = \exp\left[-0.5 \left(\frac{X^2}{\sigma_X^2} + 
                              \frac{Y^2}{\sigma_Y^2}   \right) \right] 
\]
where $(x,y)$ are the offsets from the central pixel of the gaussian
function, and 
\begin{align*}
X& = x\sin\theta - y\cos\theta&
  Y&= x\cos\theta + y\sin\theta\\
\sigma_X^2& = \frac{(a/2)^2}{2\ln2}&
  \sigma_Y^2& = \frac{(b/2)^2}{2\ln2}\\
\end{align*}

\secB{Input/Output of reconstructed/smoothed arrays}
\label{sec-reconIO}

The smoothing and reconstruction stages can be relatively
time-consuming, particularly for large cubes and reconstructions in
3-D (or even spatial smoothing). To get around this, \duchamp provides
a shortcut to allow users to perform multiple searches (\eg with
different thresholds) on the same reconstruction/smoothing setup
without re-doing the calculations each time.

To save the reconstructed array as a FITS file, set
\texttt{flagOutputRecon = true}. The file will be saved in the same
directory as the input image, so the user needs to have write
permissions for that directory.

The name of the file can given by the \texttt{fileOutputRecon}
parameter, but this can be ignored and \duchamp will present a name
based on the reconstruction parameters. The filename will be derived
from the input filename, with extra information detailing the
reconstruction that has been done. For example, suppose
\texttt{image.fits} has been reconstructed using a 3-dimensional
reconstruction with filter \#2, thresholded at $4\sigma$ using all
scales. The output filename will then be
\texttt{image.RECON-3-2-4-1.fits} (\ie it uses the four parameters
relevant for the \atrous reconstruction as listed in
Appendix~\ref{app-param}). The new FITS file will also have these
parameters as header keywords. If a subsection of the input image has
been used (see \S\ref{sec-input}), the format of the output filename
will be \texttt{image.sub.RECON-3-2-4-1.fits}, and the subsection that
has been used is also stored in the FITS header.

Likewise, the residual image, defined as the difference between the
input and reconstructed arrays, can also be saved in the same manner
by setting \texttt{flagOutputResid = true}. Its filename will be the
same as above, with \texttt{RESID} replacing \texttt{RECON}.

If a reconstructed image has been saved, it can be read in and used
instead of redoing the reconstruction. To do so, the user should set
the parameter \texttt{flagReconExists = true}. The user can indicate
the name of the reconstructed FITS file using the \texttt{reconFile}
parameter, or, if this is not specified, \duchamp searches for the
file with the name as defined above. If the file is not found, the
reconstruction is performed as normal. Note that to do this, the user
needs to set \texttt{flagAtrous = true} (obviously, if this is
\texttt{false}, the reconstruction is not needed).

To save the smoothed array, set \texttt{flagOutputSmooth = true}. As
for the reconstructed/residual arrays, the
name of the file can given by the \texttt{fileOutputSmooth} parameter,
but this can be ignored and \duchamp will present a name based on the
method of smoothing used. It will be either
\texttt{image.SMOOTH-1D-a.fits}, where a is replaced by the Hanning
width used, or \texttt{image.SMOOTH-2D-a-b-c.fits}, where the Gaussian
kernel parameters are a,b,c. Similarly to the reconstruction case, a
saved file can be read in by setting \texttt{flagSmoothExists = true}
and either specifying a file to be read with the \texttt{smoothFile}
parameter or relying on \duchamp to find the file with the name as
given above.


\secB{Searching the image}
\label{sec-detection}

\secC{Technique}

The basic idea behind detection in \duchamp is to locate sets of
contiguous voxels that lie above some threshold. No size or shape
requirement is imposed upon the detections -- that is, \duchamp does
not fit \eg a Gaussian profile to each source. All it does is find
connected groups of bright voxels.

One threshold is calculated for the entire cube, enabling calculation
of signal-to-noise ratios for each source (see
Section~\ref{sec-output} for details). The user can manually specify a
value (using the parameter \texttt{threshold}) for the threshold,
which will override the calculated value. Note that this option
overrides any settings of \texttt{snrCut} or FDR options (see below). 

The cube is searched one channel map at a time, using the
2-dimensional raster-scanning algorithm of \citet{lutz80} that
connects groups of neighbouring pixels. Such an algorithm cannot be
applied directly to a 3-dimensional case, as it requires that objects
are completely nested in a row (when scanning along a row, if an
object finishes and other starts, you won't get back to the first
until the second is completely finished for the
row). Three-dimensional data does not have this property, hence the
need to treat the data on a 2-dimensional basis.

Although there are parameters that govern the minimum number of pixels
in a spatial and spectral sense that an object must have
(\texttt{minPix} and \texttt{minChannels} respectively), these
criteria are not applied at this point. It is only after the merging
and growing (see \S\ref{sec-merger}) is done that objects are rejected
for not meeting these criteria.

Finally, the search only looks for positive features. If one is
interested instead in negative features (such as absorption lines),
set the parameter \texttt{flagNegative = true}. This will invert the
cube (\ie multiply all pixels by $-1$) prior to the search, and then
re-invert the cube (and the fluxes of any detections) after searching
is complete. All outputs are done in the same manner as normal, so
that fluxes of detections will be negative.

\secC{Calculating statistics}
\label{sec-stats}

A crucial part of the detection process (as well as the wavelet
reconstruction: \S\ref{sec-recon}) is estimating the statistics that
define the detection threshold. To determine a threshold, we need to
estimate from the data two parameters: the middle of the noise
distribution (the ``noise level''), and the width of the distribution
(the ``noise spread''). The noise level is estimated by either the
mean or the median, and the noise spread by the rms (or the standard
deviation) or the median absolute deviation from the median
(MADFM). The median and MADFM are robust statistics, in that they are
not biased by the presence of a few pixels much brighter than the
noise.

All four statistics are calculated automatically, but the choice of
parameters that will be used is governed by the input parameter
\texttt{flagRobustStats}. This has the default value \texttt{true},
meaning the underlying mean of the noise distribution is estimated by
the median, and the underlying standard deviation is estimated by the
MADFM. In the latter case, the value is corrected, under the
assumption that the underlying distribution is Normal (Gaussian), by
dividing by 0.6744888 -- see Appendix~\ref{app-madfm} for details. If
\texttt{flagRobustStats=false}, the mean and rms are used instead.

The choice of pixels to be used depend on the analysis method. If the
wavelet reconstruction has been done, the residuals (defined
in the sense of original $-$ reconstruction) are used to estimate the
noise spread of the cube, since the reconstruction should pick out
all significant structure. The noise level (the middle of the
distribution) is taken from the original array.

If smoothing of the cube has been done instead, all noise parameters
are measured from the smoothed array, and detections are made with
these parameters. When the signal-to-noise level is quoted for each
detection (see \S\ref{sec-output}), the noise parameters of the
original array are used, since the smoothing process correlates
neighbouring pixels, reducing the noise level.

If neither reconstruction nor smoothing has been done, then the
statistics are calculated from the original, input array. 

The parameters that are estimated should be representative of the
noise in the cube. For the case of small objects embedded in many
noise pixels (\eg the case of \hi surveys), using the full cube will
provide good estimators. It is possible, however, to use only a
subsection of the cube by setting the parameter \texttt{flagStatSec =
  true} and providing the desired subsection to the \texttt{StatSec}
parameter. This subsection works in exactly the same way as the pixel
subsection discussed in \S\ref{sec-input}. Note that this subsection
applies only to the statistics used to determine the threshold. It
does not affect the calculation of statistics in the case of the
wavelet reconstruction. Note also that pixels flagged as BLANK or as
part of the ``Milky Way'' range of channels are ignored in the
statistics calculations.

\secC{Determining the threshold}

Once the statistics have been calculated, the threshold is determined
in one of two ways. The first way is a simple sigma-clipping, where a
threshold is set at a fixed number $n$ of standard deviations above
the mean, and pixels above this threshold are flagged as detected. The
value of $n$ is set with the parameter \texttt{snrCut}. The ``mean''
and ``standard deviation'' here are estimated according to
\texttt{flagRobustStats}, as discussed in \S\ref{sec-stats}. In this
first case only, if the user specifies a threshold, using the
\texttt{threshold} parameter, the sigma-clipped value is ignored.

The second method uses the False Discovery Rate (FDR) technique
\citep{miller01,hopkins02}, whose basis we briefly detail here. The
false discovery rate (given by the number of false detections divided
by the total number of detections) is fixed at a certain value
$\alpha$ (\eg $\alpha=0.05$ implies 5\% of detections are false
positives). In practice, an $\alpha$ value is chosen, and the ensemble
average FDR (\ie $\langle FDR \rangle$) when the method is used will
be less than $\alpha$.  One calculates $p$ -- the probability,
assuming the null hypothesis is true, of obtaining a test statistic as
extreme as the pixel value (the observed test statistic) -- for each
pixel, and sorts them in increasing order. One then calculates $d$
where
\[
d = \max_j \left\{ j : P_j < \frac{j\alpha}{c_N N} \right\},
\]
and then rejects all hypotheses whose $p$-values are less than or
equal to $P_d$. (So a $P_i<P_d$ will be rejected even if $P_i \geq
j\alpha/c_N N$.) Note that ``reject hypothesis'' here means ``accept
the pixel as an object pixel'' (\ie we are rejecting the null
hypothesis that the pixel belongs to the background).

The $c_N$ value here is a normalisation constant that depends on the
correlated nature of the pixel values. If all the pixels are
uncorrelated, then $c_N=1$. If $N$ pixels are correlated, then their
tests will be dependent on each other, and so $c_N = \sum_{i=1}^N
i^{-1}$. \citet{hopkins02} consider real radio data, where the pixels
are correlated over the beam. For the calculations done in \duchamp,
$N = B \times C$, where $B$ is the beam size in pixels, calculated
from the FITS header (if the correct keywords -- BMAJ, BMIN -- are not
present, the size of the beam is taken from the parameter
\texttt{beamSize}), and $C$ is the number of neighbouring channels
that can be considered to be correlated.

The use of the FDR method is governed by the \texttt{flagFDR} flag,
which is \texttt{false} by default. To set the relevant parameters,
use \texttt{alphaFDR} to set the $\alpha$ value, and
\texttt{FDRnumCorChan} to set the $C$ value discussed above. These
have default values of 0.01 and 2 respectively.

The theory behind the FDR method implies a direct connection between
the choice of $\alpha$ and the fraction of detections that will be
false positives. These detections, however, are individual pixels,
which undergo a process of merging and rejection (\S\ref{sec-merger}),
and so the fraction of the final list of detected objects that are
false positives will be much smaller than $\alpha$. See the discussion
in \S\ref{sec-notes}.

%\secC{Storage of detected objects in memory}
%
%It is useful to understand how \duchamp stores the detected objects in
%memory while it is running. This makes use of nested C++ classes, so
%that an object is stored as a class that includes the set of detected
%pixels, plus all the various calculated parameters (fluxes, WCS
%coordinates, pixel centres and extrema, flags,...). The set of pixels
%are stored using another class, that stores 3-dimensional objects as a
%set of channel maps, each consisting of a $z$-value and a
%2-dimensional object (a spatial map if you like). This 2-dimensional
%object is recorded using ``run-length'' encoding, where each row (a
%fixed $y$ value) is stored by the starting $x$-value and the length

\secB{Merging and growing detected objects}
\label{sec-merger}

The searching step produces a list of detected objects that will have
many repeated detections of a given object -- for instance, spectral
detections in adjacent pixels of the same object and/or spatial
detections in neighbouring channels. These are then combined in an
algorithm that matches all objects judged to be ``close'', according
to one of two criteria.

One criterion is to define two thresholds -- one spatial and one in
velocity -- and say that two objects should be merged if there is at
least one pair of pixels that lie within these threshold distances of
each other. These thresholds are specified by the parameters
\texttt{threshSpatial} and \texttt{threshVelocity} (in units of pixels
and channels respectively).

Alternatively, the spatial requirement can be changed to say that
there must be a pair of pixels that are \emph{adjacent} -- a stricter,
but perhaps more realistic requirement, particularly when the spatial
pixels have a large angular size (as is the case for 
\hi surveys). This 
method can be selected by setting the parameter
\texttt{flagAdjacent} to 1 (\ie \texttt{true}) in the parameter
file. The velocity thresholding is done in the same way as the first
option.

Once the detections have been merged, they may be ``grown''. This is a
process of increasing the size of the detection by adding nearby
pixels (according to the \texttt{threshSpatial} and
\texttt{threshVelocity} parameters) that are above some secondary
threshold. This threshold should be lower than the one used for the
initial detection, but above the noise level, so that faint pixels are
only detected when they are close to a bright pixel. This
threshold is specified via one of two input parameters. It can be
given in terms of the noise statistics via \texttt{growthCut} (which
has a default value of $3\sigma$), or it can be directly given via
\texttt{growthThreshold}. Note that if you have given the detection
threshold with the \texttt{threshold} parameter, the growth threshold
\textbf{must} be given with \texttt{growthThreshold}.

The use of the growth algorithm is controlled by the
\texttt{flagGrowth} parameter -- the default value of which is
\texttt{false}. If the detections are grown, they are sent through the
merging algorithm a second time, to pick up any detections that now
overlap or have grown over each other.

Finally, to be accepted, the detections must span \emph{both} a
minimum number of channels (enabling the removal of any spurious
single-channel spikes that may be present), and a minimum number of
spatial pixels. These numbers, as for the original detection step, are
set with the \texttt{minChannels} and \texttt{minPix} parameters. The
channel requirement means a source must have at least one set of
\texttt{minChannels} consecutive channels to be
accepted.

 %What \duchamp\ is doing
\newpage
% -----------------------------------------------------------------------
% outputs.tex: Section detailing the different forms of text- and
%              plot-based output.
% -----------------------------------------------------------------------
% Copyright (C) 2006, Matthew Whiting, ATNF
%
% This program is free software; you can redistribute it and/or modify it
% under the terms of the GNU General Public License as published by the
% Free Software Foundation; either version 2 of the License, or (at your
% option) any later version.
%
% Duchamp is distributed in the hope that it will be useful, but WITHOUT
% ANY WARRANTY; without even the implied warranty of MERCHANTABILITY or
% FITNESS FOR A PARTICULAR PURPOSE.  See the GNU General Public License
% for more details.
%
% You should have received a copy of the GNU General Public License
% along with Duchamp; if not, write to the Free Software Foundation,
% Inc., 59 Temple Place, Suite 330, Boston, MA 02111-1307, USA
%
% Correspondence concerning Duchamp may be directed to:
%    Internet email: Matthew.Whiting [at] atnf.csiro.au
%    Postal address: Dr. Matthew Whiting
%                    Australia Telescope National Facility, CSIRO
%                    PO Box 76
%                    Epping NSW 1710
%                    AUSTRALIA
% -----------------------------------------------------------------------
\secA{Outputs}
\label{sec-output}

\secB{During execution}

\duchamp provides the user with feedback whilst it is running, to
keep the user informed on the progress of the analysis. Most of this
consists of self-explanatory messages about the particular stage the
program is up to. The relevant parameters are printed to the screen at
the start (once the file has been successfully read in), so the user
is able to make a quick check that the setup is correct (see
Appendix~{app-input} for an example).

If the cube is being trimmed (\S\ref{sec-modify}), the resulting
dimensions are printed to indicate how much has been trimmed. If a
reconstruction is being done, a continually updating message shows
either the current iteration and scale, compared to the maximum scale
(when \texttt{reconDim = 3}), or a progress bar showing the amount of
the cube that has been reconstructed (for smaller values of
\texttt{reconDim}).

During the searching algorithms, the progress through the search is
shown. When completed, the number of objects found is reported (this
is the total number found, before any merging is done).

In the merging process (where multiple detections of the same object
are combined -- see \S\ref{sec-merger}), two stages of output
occur. The first is when each object in the list is compared with all
others. The output shows two numbers: the first being how far through
the list the current object is, and the second being the length of the
list. As the algorithm proceeds, the first number should increase and
the second should decrease (as objects are combined). When the numbers
meet, the whole list has been compared. If the objects are being
grown, a similar output is shown, indicating the progress through the
list. In the rejection stage, in which objects not meeting the minimum
pixels/channels requirements are removed, the total number of objects
remaining in the list is shown, which should steadily decrease with
each rejection until all have been examined. Note that these steps can
be very quick for small numbers of detections.

Since this continual printing to screen has some overhead of time and
CPU involved, the user can elect to not print this information by
setting the parameter \texttt{verbose = false}. In this case, the user
is still informed as to the steps being undertaken, but the details of
the progress are not shown.

There may also be Warning or Error messages printed to screen. The
Warning messages occur when something happens that is unexpected (for
instance, a desired keyword is not present in the FITS header), but
not detrimental to the execution. An Error message is something more
serious, and indicates some part of the program was not able to
complete its task. The message will also indicate which function or
subroutine generated it -- this is primarily a tool for debugging, but
can be useful in determining what went wrong.

\secB{Text-based output files}

\secC{Table of results}
\label{sec-results}

Finally, we get to the results -- the reason for running \duchamp in
the first place. Once the detection list is finalised, it is sorted by
the mean velocity of the detections (or, if there is no good WCS
associated with the cube, by the mean $z$-pixel position). The results
are then printed to the screen and to the output file, given by the
\texttt{OutFile} parameter. 

The output consists of two sections. First, a list of the parameters
are printed to the output file, for future reference. Next, the
detection threshold that was used is given, so comparison can be made
with other searches. The statistics estimating the noise parameters
are given (see \S\ref{sec-stats}). Thirdly, the number of detections
are reported.

All this information, known as the ``header'', can either be written
to the start of the output file (denoted by the parameter
\texttt{OutFile}), or written to a separate file from the list of
detections. This second option is activated by the parameter
\texttt{flagSeparateHeader}, and the information is written to the
file given by \texttt{HeaderFile}.

The most interesting part, however, is the list of detected
objects. This list, an example of which can be seen in
Appendix~\ref{app-output}, contains the following columns (note that
the title of the columns depending on WCS information will depend on
the details of the WCS projection: they are shown below for the
Equatorial and Galactic projection cases).

\begin{Lentry}
\item[{Obj\#}] The ID number of the detection (simply the
  sequential count for the list, which is ordered by increasing
  velocity, or channel number, if the WCS is not good enough to find
  the velocity).
\item[{Name}] The ``IAU''-format name of the detection (derived from the
  WCS position -- see below for a description of the format).
\item[{X,Y,Z}] The ``centre'' pixel position, determined by the input
  parameter \texttt{pixelCentre}.
\item[{RA/GLON}] The Right Ascension or Galactic Longitude of the centre
  of the object.
\item[{DEC/GLAT}] The Declination or Galactic Latitude of the centre of
  the object.
\item[{VEL}] The mean velocity of the object [units given by the
  \texttt{spectralUnits} parameter].
\item[{w\_RA/w\_GLON}] The width of the object in Right Ascension or
  Galactic Longitude (depending on FITS coordinates) [arcmin].
\item[{w\_DEC/w\_GLAT}] The width of the object in Declination Galactic
  Latitude [arcmin].
\item[{w\_50}] The velocity width of the detection at 50\% of the peak
  flux (the measured full-width at half-maximum, FWHM), in velocity
  units [see note below].
\item[{F\_int}] The integrated flux over the object, in the units of
  flux times velocity, corrected for the beam if necessary.
\item[{F\_peak}] The peak flux over the object, in the units of flux.
\item[{S/Nmax}] The signal-to-noise ratio at the peak pixel.
\item[{X1, X2}] The minimum and maximum X-pixel coordinates.
\item[{Y1, Y2}] The minimum and maximum Y-pixel coordinates.
\item[{Z1, Z2}] The minimum and maximum Z-pixel coordinates.
\item[{Npix}] The number of voxels (\ie distinct $(x,y,z)$ coordinates)
  in the detection.
\item[{Flag}] Whether the detection has any warning flags (see below).
\end{Lentry}

These parameters are written to the screen during execution. There are
alternative ways of calculating the total flux, the position and
velocity width, however, and so additional parameters are written to
the output file:
\begin{Lentry}
\item[{w\_20}] The velocity width of the detection at 20\% of the peak
  flux, in velocity units [see note below].
\item[{w\_VEL}] The full velocity width of the detection (max channel
  $-$ min channel, in velocity units).
\item[{F\_tot}] The sum of the flux values of all detected voxels.
\item[{X\_av, Y\_av, Z\_av}] The average pixel value in each
  axis direction \ie X\_av is the average of the $x$-values of all
  pixels in the detection.
\item[{X\_cent, Y\_cent, Z\_cent}] The centroid position, being
  the flux-weighted average of the pixels.
\item[{X\_peak, Y\_peak, Z\_peak}] The location of the pixel
  containing the peak flux value.
\end{Lentry}
The velocity width of the detection is calculated at 50\% and 20\% of
the peak flux, as well as the full detected width (if the detection
threshold is greater than 20\% or 50\% of the peak, then these values
will be the same as \texttt{w\_VEL}. The type of position value given
in the \texttt{X, Y, Z} columns in the screen output is determined by
the \texttt{pixelCentre} parameter. All three alternatives are shown
in the output file.

The user can specify the precision used to display the flux, velocity
and S/Nmax values, by using the input parameters \texttt{precFlux},
\texttt{precVel} and \texttt{precSNR} respectively. These values apply
to the tables written to the screen and to the output file, as well as
for the VOTable (see below).

The \texttt{Name} is derived from the WCS position. For instance, a
source that is centred on the RA,Dec position 12$^h$53$^m$45$^s$,
-36$^\circ$24$'$12$''$ will be given the name J125345$-$362412, if the
epoch is J2000, or the name B125345$-$362412 if it is B1950. An
alternative form is used for Galactic coordinates: a source centred on
the position ($l$,$b$) = (323.1245, 5.4567) will be called
G323.124$+$05.457. If the WCS is not valid (\ie is not present or does
not have all the necessary information), the \texttt{Name, RA, DEC,
VEL} and related columns are not printed, but the pixel coordinates
are still provided.

The velocity units can be specified by the user, using the parameter
\texttt{spectralUnits} (enter it as a single string with no
spaces). The default value is km/s, which should be suitable for most
users. These units are also used to give the units of integrated
flux. Note that if there is no rest frequency specified in the FITS
header, the \duchamp output will instead default to using Frequency,
with units of MHz.

If the WCS is absent or not sufficiently specified, then all columns
from \texttt{RA/GLON} to \texttt{w\_VEL} will be left blank. Also,
\texttt{F\_int} will be replaced with the more simple \texttt{F\_tot}.

The \texttt{Flag} column contains any warning flags, such as:
\begin{itemize}
\item \textbf{E} -- The detection is next to the spatial edge of the image,
meaning either the limit of the pixels, or the limit of the non-BLANK
pixel region.
\item \textbf{S} -- The detection lies at the edge of the spectral region. 
\item \textbf{N} -- The total flux, summed over all the (non-BLANK)
pixels in the smallest box that completely encloses the detection, is
negative. Note that this sum is likely to include non-detected
pixels. It is of use in pointing out detections that lie next to
strongly negative pixels, such as might arise due to interference --
the detected pixels might then also be due to the interference, so
caution is advised.
\end{itemize}
In the absence of any of these flags, a \textbf{-} will be recorded in
this column.

\secC{Other results lists}

Three additional results files can also be requested. One option is a
VOTable-format XML file, containing just the RA, Dec, Velocity and the
corresponding widths of the detections, as well as the fluxes. The
user should set \texttt{flagVOT = true}, and put the desired filename
in the parameter \texttt{votFile} -- note that the default is for it
not to be produced. This file should be compatible with all Virtual
Observatory tools (such as Aladin%
\footnote{%Aladin can be found on the web at
\href{http://aladin.u-strasbg.fr/}{http://aladin.u-strasbg.fr/}}
or TOPCAT\footnote{%Tool for OPerations on Catalogues And Tables:
\href{http://www.star.bristol.ac.uk/~mbt/topcat/}%
{http://www.star.bristol.ac.uk/~mbt/topcat/}}). 

A second option is an annotation file for use with the Karma toolkit
of visualisation tools (in particular, with \texttt{kvis}). There are
two options on how objects are represented, governed by the
\texttt{annotationType} parameter. Setting this to \texttt{borders}
results in a border being drawn around the spatial pixels of the
object, in a manner similar to that seen in Fig.~\ref{fig-spect}. Note
that Karma/\texttt{kvis} does not always do this perfectly, so the
lines may not be directly lined up with pixel borders. The other
option is \texttt{annotationType = circles}. This will draw a circle
at the position of each detection, scaled by the spatial size of the
detection, and number it according to the Obj\# given above. To make
use of this option, the user should set \texttt{flagKarma = true}, and
put the desired filename in the parameter \texttt{karmaFile} -- again,
the default is for it not to be produced.

The final optional results file produced is a simple text file that
contains the spectra for each detected object. The format of the file
is as follows: the first column has the spectral coordinate, over the
full range of values; the remaining columns represent the flux values
for each object at the corresponding spectral position. The flux value
used is the same as that plotted in the spectral plot detailed below,
and governed by the \texttt{spectralMethod} parameter. An example of
what a spectral text file might look like is given below:

\begin{quote}
  {\footnotesize
    \begin{tabular}{lllll}
      1405.00219727  &0.01323344  &0.23648241  &0.04202826  &-0.00506790  \\
      1405.06469727  &0.01302835  &0.27393046  &0.04686056  &-0.00471084  \\
      1405.12719727  &0.01583361  &0.27760920  &0.04114933  &-0.01168737  \\
      1405.18969727  &0.01271889  &0.31489247  &0.03307962  &-0.00300790  \\
      1405.25219727  &0.01597644  &0.30401203  &0.05356426  &-0.00551653  \\
      1405.31469727  &0.00773827  &0.30031312  &0.04074831  &-0.00570147  \\
      1405.37719727  &0.00738304  &0.27921870  &0.05272378  &-0.00504959  \\
      1405.43969727  &0.01353923  &0.26132512  &0.03667958  &-0.00151006  \\  
      1405.50219727  &0.01119724  &0.28987029  &0.03497849  &-0.00645589  \\  
      1405.56469727  &0.00813379  &0.29839963  &0.04711142  &0.00536576   \\  
      1405.62719727  &0.00774377  &0.26530230  &0.04620502  &0.00724631   \\  
      1405.68969727  &0.00576067  &0.23215000  &0.04995513  &0.00290841   \\ 
      1405.75219727  &0.00452834  &0.16484940  &0.04261605  &-0.00612812  \\  
      1405.81469727  &0.01406293  &0.15989439  &0.03817926  &-0.00758385  \\ 
      1405.87719727  &0.01116611  &0.11890682  &0.05499069  &-0.00626362  \\  
      1405.93969727  &0.00687582  &0.10620256  &0.04743370  &0.00055177   \\
      $\vdots$       &$\vdots$    &$\vdots$    &$\vdots$    &$\vdots$     \\
    \end{tabular}
  }
\end{quote}

In addition to these three files, a log file can also be produced. As
the program is running, it also (optionally) records the detections
made in each individual spectrum or channel (see \S\ref{sec-detection}
for details on this process). This is recorded in the file given by
the parameter \texttt{LogFile}. This file does not include the columns
\texttt{Name, RA, DEC, w\_RA, w\_DEC, VEL, w\_VEL}. This file is
designed primarily for diagnostic purposes: \eg to see if a given set
of pixels is detected in, say, one channel image, but does not survive
the merging process. The list of pixels (and their fluxes) in the
final detection list are also printed to this file, again for
diagnostic purposes. The file also records the execution time, as well
as the command-line statement used to run \duchamp. The creation of
this log file can be prevented by setting \texttt{flagLog = false}
(which is the default).

\secB{Graphical output}

\begin{figure}[t]
  \begin{center}
    \includegraphics[width=\textwidth]{example_spectrum}
  \end{center}
  \caption{\footnotesize An example of the spectral output. Note several
    of the features discussed in the text: the red lines indicating the
    reconstructed spectrum; the blue dashed lines indicating the
    spectral extent of the detection; the green hashed area indicating
    the Milky Way channels that are ignored by the searching algorithm;
    the blue border showing its spatial extent on the 0th moment map;
    and the 15~arcmin-long scale bar.}
  \label{fig-spect}
\end{figure}

\begin{figure}[!t]
  \begin{center}
    \includegraphics[width=\textwidth]{example_moment_map}
  \end{center}
  \caption{\footnotesize An example of the moment map created by
    \duchamp. The full extent of the cube is covered, and the 0th moment
    of each object is shown (integrated individually over all the
    detected channels). The purple line indicates the limit of the
    non-BLANK pixels.}
  \label{fig-moment}
\end{figure}

\secC{Mask image}
\label{sec-maskOut}

It is possible to create a FITS file containing a mask array. This
array is designed to indicate the location of detected objects. The
value of the detected pixels is determined by the
\texttt{flagMaskWithObjectNum} parameter: if \texttt{true}, the value
of the pixels is given by the corresponding object ID number; if
\texttt{false}, they take the value 1 for all objects. Pixels not in a
detected object have the value 0. To create this FITS file, set the
input parameter \texttt{flagOutputMask=true}. The file will be given
the name \texttt{image.MASK.fits} (where the input image is called
\texttt{image.fits}).

\secC{Spectral plots}

As well as the output data file, a postscript file (with the filename
given by the \texttt{spectralFile} parameter) is created that shows
the spectrum for each detection, together with a small cutout image
(the 0th moment) and basic information about the detection (note that
any flags are printed after the name of the detection, in the format
\texttt{[E]}). If the cube was reconstructed, the spectrum from the
reconstruction is shown in red, over the top of the original
spectrum. The spectral extent of the detected object is indicated by
two dashed blue lines, and the region covered by the ``Milky Way''
channels is shown by a green hashed box. An example detection can be
seen in Fig.~\ref{fig-spect}.

The spectrum that is plotted is governed by the
\texttt{spectralMethod} parameter. It can be either \texttt{peak} (the
default), where the spectrum is from the spatial pixel containing the
detection's peak flux; or \texttt{sum}, where the spectrum is summed
over all spatial pixels, and then corrected for the beam size.  The
spectral extent of the detection is indicated with blue lines, and a
zoom is shown in a separate window.

The cutout image can optionally include a border around the spatial
pixels that are in the detection (turned on and off by the
\texttt{drawBorders} parameter -- the default is \texttt{true}). It
includes a scale bar in the bottom left corner to indicate size -- its
length is indicated next to it (the choice of length depends on the
size of the image).

There may also be one or two extra lines on the image. A yellow line
shows the limits of the cube's spatial region: when this is shown, the
detected object will lie close to the edge, and the image box will
extend outside the region covered by the data. A purple line, however,
shows the dividing line between BLANK and non-BLANK pixels. The BLANK
pixels will always be shown in black. The first type of line is always
drawn, while the second is governed by the parameter
\texttt{drawBlankEdges} (whose default is \texttt{true}), and
obviously whether there are any BLANK pixel present.

\secC{Output for 2-dimensional images}

When the input image is two-dimensional, with no spectral dimension,
this spectral plot would not make much sense. Instead, \duchamp
creates a similar postscript file that simply includes the text
headers as well as the 0th-moment map of the detection. As for the
normal spectral case, this file will be written to the filename given
by the \texttt{spectralFile} parameter.

\secC{Spatial maps}

Finally, a couple of images are optionally produced: a 0th moment map
of the cube, combining just the detected channels in each object,
showing the integrated flux in grey-scale; and a ``detection image'',
a grey-scale image where the pixel values are the number of channels
that spatial pixel is detected in. In both cases, if
\texttt{drawBorders = true}, a border is drawn around the spatial
extent of each detection, and if \texttt{drawBlankEdges = true}, the
purple line dividing BLANK and non-BLANK pixels (as described above)
is drawn. An example moment map is shown in Fig.~\ref{fig-moment}.
The production or otherwise of these images is governed by the
\texttt{flagMaps} parameter.

The moment map is also displayed in a PGPlot XWindow (with the
\texttt{/xs} display option). This feature can be turned off by
setting \texttt{flagXOutput = false} -- this might be useful if
running \duchamp on a terminal with no window display capability, or
if you have set up a script to run it in a batch mode.

The purpose of these images are to provide a visual guide to where the
detections have been made, and, particularly in the case of the moment
map, to provide an indication of the strength of the source. In both
cases, the detections are numbered (in the same sense as the output
list and as the spectral plots), and the spatial borders are marked
out as for the cutout images in the spectra file. Both these images
are saved as postscript files (given by the parameters
\texttt{momentMap} and \texttt{detectionMap} respectively), with the
latter also displayed in a \textsc{pgplot} window (regardless of the
state of \texttt{flagMaps}).



\secB{Re-using previous detections}
\label{sec-reuse}

It may be the case that, once you have run \duchamp with a set of
parameters, you are unsatisfied with the output spectra -- perhaps you
would have preferred integrated rather than peak flux to be
plotted. However, the searching might have taken a while to run, and
the thought of doing it again just for different plots may be a bit
off-putting.

Well, provided you have made a log file when running \duchamp (with
the \texttt{flagLog=true} setting), it is possible to do this easily
without having to go through the process of detecting your sources a
second time. By using the same input file, with the additional
parameter \texttt{usePrevious=true}, the log file that was created
with a previous \duchamp run can be read to extract each of the
individual detections. The output stage is then run again, with the
parameters (in particular \texttt{pixelCentre} and
\texttt{spectralMethod}) as given in the input file. 

Perhaps you would also like to extract a single source's
spectral plot (\eg for use in a journal paper). The use-previous
method allows you to specify particular sources to re-plot. Only these
sources will be plotted in the \texttt{SpectraFile} file, and
individual files will be created for each of the listed sources. Their
filenames will follow the format of \texttt{SpectraFile}: if,
\texttt{SpectraFile=file.ps}, source \#3 will appear in
\texttt{file-03.ps}. To give a list of sources, use the
\texttt{objectList} parameter, and provide a string with individual
object numbers or object ranges: \eg 1,2,4-7,8,11. %Outputs
\newpage
\secA{Notes and hints on the use of \duchamp}
\label{sec-notes}

In using \duchamp, the user has to make a number of decisions about
the way the program runs. This section is designed to give the user
some idea about what to choose.

The main choice is whether or not to use the wavelet
reconstruction. The main benefits of this are the marked reduction in
the noise level, leading to regularly-shaped detections, and good
reliability for faint sources. The main drawback with its use is the
long execution time: to reconstruct a $170\times160\times1024$
(\hipass) cube often requires three iterations and takes about 20-25
minutes to run completely. Note that this is for the more complete
three-dimensional reconstruction: using \texttt{reconDim=1} makes the
reconstruction quicker (the full program then takes about 6 minutes),
but it is still the largest part of the time.

The searching part of the procedure is much quicker: searching an
un-reconstructed cube leads to execution times of only a couple of
minutes. Alternatively, using the ability to read in previously-saved
reconstructed arrays makes running the reconstruction more than once a
more feasible prospect.

On the positive side, the shape of the detections in a cube that has
been reconstructed will be much more regular and smooth -- the ragged
edges that objects in the raw cube possess are smoothed by the removal
of most of the noise. This enables better determination of the shapes
and characteristics of objects.

A further point to consider when using the reconstruction is that if
the two-dimensional reconstruction is chosen (\texttt{reconDim=2}), it
can be susceptible to edge effects. If the valid area in the cube (\ie
the part that is not BLANK) has non-rectangular edges, the convolution
can produce artefacts in the reconstruction that mimic the edges and
can lead (depending on the selection threshold) to some spurious
sources. Caution is advised with such data -- the user is advised to
check carefully the reconstructed cube for the presence of such
artefacts. Note, however, that the 1- and 3-dimensional
reconstructions are \emph{not} susceptible in the same way, since the
spectral direction does not generally exhibit these BLANK edges, and
so we recommend the use of either of these.

If one chooses the reconstruction method, a further decision is
required on the signal-to-noise cutoff used in determining acceptable
wavelet coefficients. A larger value will remove more noise from the
cube, at the expense of losing fainter sources, while a smaller value
will include more noise, which may produce spurious detections, but
will be more sensitive to faint sources. Values of less than about
$3\sigma$ tend to not reduce the noise a great deal and can lead to
many spurious sources (this depends, of course on the cube itself).

When it comes to searching, the FDR method produces more reliable
results than simple sigma-clipping, particularly in the absence of
reconstruction.  However, it does not work in exactly the way one
would expect for a given value of \texttt{alpha}. For instance,
setting fairly liberal values of \texttt{alpha} (say, 0.1) will often
lead to a much smaller fraction of false detections (\ie much less
than 10\%). This is the effect of the merging algorithms, that combine
the sources after the detection stage, and reject detections not
meeting the minimum pixel or channel requirements.  It is thus better
to aim for larger \texttt{alpha} values than those derived from a
straight conversion of the desired false detection rate.

Finally, as \duchamp\ is still undergoing development, there are some
elements that are not fully developed. In particular, it is not as
clever as I would like at avoiding interference. The ability to place
requirements on the minimum number of channels and pixels partially
circumvents this problem, but work is being done to make \duchamp\
smarter at rejecting signals that are clearly (to a human eye at
least) interference. See the following section for further
improvements that are planned.
 %Notes and hints on the use of \duchamp
\newpage
\secA{Future developments}

Here are lists of planned improvements and a wish-list of
features that would be nice to include (but are not planned in the
immediate future). Let me know if there are items not on these lists,
or items on the list you would like prioritised.

Planned developments:
\begin{itemize}
\item Parallelisation of the code, to improve speed particularly on
multi-core machines.

\item Better determination of the noise characteristics of
  spectral-line cubes, including understanding how the noise is
  generated and developing a model for it. 
  
\item Include more source analysis. Examples could be: shape
  information; measurements of HI mass; more variety of measurements
  of velocity width and profile. 

\item Improved ability to reject interference, possibly on the
  spectral shape of features.

\item Ability to separate (de-blend) distinct sources that have been
  merged.
\end{itemize}

Wish-list:
\begin{itemize}
\item Incorporation of Swinburne's S2PLOT
\footnote{\href{http://astronomy.swin.edu.au/s2plot/}
{http://astronomy.swin.edu.au/s2plot/}} code for improved
visualisation. 
\item Link to lists of possible counterparts (\eg via NED/SIMBAD/other
  VO tools?). 

\item On-line web service interface, so a user can upload a cube and
  get back a source-list.

\item Embed \duchamp\ in a GUI, to move away from the text-based
  interaction.
\end{itemize}

 %Future developments


\bibliographystyle{mn2e}
\bibliography{guide}

\appendix
\newpage
\secA{Obtaining and installing \duchamp}
\label{app-install}

\secB{Installing}
The \duchamp\ web page can be found at the following location:\\
\href{http://www.atnf.csiro.au/people/Matthew.Whiting/Duchamp}%
{http://www.atnf.csiro.au/people/Matthew.Whiting/Duchamp}\\
Here you can find a gzipped tar archive of the source code that can be
downloaded and extracted, as well as this User's Guide in postscript
and hyperlinked PDF formats.

To build \duchamp, you will need three main external libraries:
\textsc{pgplot}, \textsc{cfitsio} (this needs to be version 2.5 or
greater -- version 3+ is better) and \textsc{wcslib}. If these are not
present on your system, you can download them from the following
locations:
\begin{itemize}
\item \textsc{pgplot}:
\href{http://www.astro.caltech.edu/~tjp/pgplot/}%
{\footnotesize http://www.astro.caltech.edu/~tjp/pgplot/}
\item \textsc{cfitsio}:
\href{http://heasarc.gsfc.nasa.gov/docs/software/fitsio/fitsio.html}%
{\footnotesize http://heasarc.gsfc.nasa.gov/docs/software/fitsio/fitsio.html}
\item \textsc{wcslib}:
\href{http://www.atnf.csiro.au/people/Mark.Calabretta/WCS/index.html}%
{\footnotesize http://www.atnf.csiro.au/people/Mark.Calabretta/WCS/index.html}
\end{itemize}

\duchamp\ can be built on Unix/Linux systems by typing (assuming that
the prompt your terminal provides is a \texttt{> } -- don't type this
character!):
\begin{quote}
{\footnotesize
\texttt{%
> ./configure\\
> make\\
> make clean (optional -- to remove the object files)}
}
\end{quote}

Run in this manner, \texttt{configure} should find all the necessary
libraries, but if some libraries have been installed in non-standard
locations, it may fail. In this case, you can specify additional
directories to look in by giving extra command-line arguments. There
are separate options for library files (eg. libcpgplot.a) and header
files (eg. cpgplot.h).

For example, suppose \textsc{wcslib} had been locally installed in the
directory \texttt{/home/mduchamp/wcslib}. There will then be two
libraries created that are likely to be in the following
subdirectories: \texttt{C/} and \texttt{pgsbox/}. Each subdirectory
needs to be searched for library and header files, so one could build
Duchamp by typing:
\begin{quote}
{\footnotesize
\texttt{%
>  ./configure $\backslash$ \\ 
LIBDIRS="/home/mduchamp/wcslib/C /home/mduchamp/wcslib/pgsbox" 
$\backslash$\\
INCDIRS="/home/mduchamp/wcslib/C /home/mduchamp/wcslib/pgsbox"}
}
\end{quote}
And then just run make in the usual fashion:
\begin{quote}
{\footnotesize
\texttt{> make}
}
\end{quote}

This will produce the executable \texttt{Duchamp}. You can verify that
it is running correctly by running the verification shell script:
\begin{quote}
{\footnotesize
\texttt{> VerifyDuchamp.sh}
}
\end{quote}
This will use a dummy FITS image in the \texttt{verification/}
directory -- this image has some Gaussian random noise, with five
Gaussian sources present, plus a dummy WCS. The script runs
Duchamp on this image with three different sets of inputs, and
compares to known results, looking for differences and reporting
any. There should be none reported if everything is working correctly.

\secB{Running \duchamp}
You can then run \duchamp\ on your own data. This can be done in one
of two ways. The first is:
\begin{quote}
{\footnotesize
\texttt{> Duchamp -f [FITS file]}
}
\end{quote}
where \texttt{[FITS file]} is the file you wish to search. This method
simply uses the default values of all parameters.

The second method allows some determination of the parameter values by
the user. Type:
\begin{quote}
{\footnotesize
\texttt{> Duchamp -p [parameter file]}
}
\end{quote}
where \texttt{[parameterFile]} is a file with the input parameters,
including the name of the cube you want to search. There are two
example input files included with the distribution. The smaller one,
\texttt{InputExample}, shows the typical parameters one might want to
set. The large one, \texttt{InputComplete}, lists all possible
parameters that can be entered, and a brief description of them. To
get going quickly, just replace the \texttt{"your-file-here"} in the
\texttt{InputExample} file with your image name, and type
\begin{quote}
{\footnotesize
\texttt{> Duchamp -p InputExample}
}
\end{quote}

The following appendices provide details on the individual parameters,
and show examples of the output files that \duchamp\ produces.

\secB{Feedback}
It may happen that you discover bugs or problems with \duchamp, or you
have suggestions for improvements or additional features to be
included in future releases. You can submit a ``ticket'' (a trackable
bug report) at the \duchamp\ Trac wiki at the following location:\\
\href{http://sourcecode.atnf.csiro.au/cgi-bin/trac\_duchamp.cgi/newticket}%
{\footnotesize 
http://sourcecode.atnf.csiro.au/cgi-bin/trac\_duchamp.cgi/newticket}
\\(there is a link to this page from the Duchamp website).

There is also an email exploder, duchamp-user\textbf{[at]}atnf.csiro.au,
that users can subscribe to keep up to date with changes, updates, and
other news about \duchamp. To subscribe, send an email (from the
account you wish to subscribe to the list) to
duchamp-user-request\textbf{[at]}atnf.csiro.au with the single word
``subscribe'' in the body of the message. To be removed from this
list, send a message with ``unsubscribe'' in its body to the same
address.

 %Obtaining and installing \duchamp
\newpage
\secA{Available parameters}
\label{app-param}

The full list of parameters that can be listed in the input file are
given here. If not listed, they take the default value given in
parentheses. Since the order of the parameters in the input file does
not matter, they are grouped here in logical sections.

\secB*{Input related}
\begin{entry}
\item[ImageFile (no default assumed)] The filename of the
  data cube to be analysed.
\item[flagSubsection \texttt{[false]}] A flag to indicate whether one
  wants a subsection of the requested image.
\item[Subsection \texttt{[ [*,*,*] ]}] The requested subsection
%, which should be specified in the format \texttt{[x1:x2,y1:y2,z1:z2]},
%  where the limits are inclusive
 -- see \S\ref{sec-input} for details on the subsection format. 
 If the full range of a dimension is required, use a \texttt{*}
%, \eg if you want the full spectral range of a subsection of the
%  image, use \texttt{[30:140,30:140,*]} 
  (thus  the default is the full cube).
\item[flagReconExists \texttt{[false]}] A flag to indicate whether the
  reconstructed array has been saved by a previous run of \duchamp. If
  set true, the reconstructed array will be read from the file given
  by \texttt{reconFile}, rather than calculated directly.
\item[reconFile (no default assumed)] The FITS file that contains the
  reconstructed array. If \texttt{flagReconExists} is true and this
  parameter is not defined, the default file that is looked for will
  be determined by the \atrous\ parameters (see \S\ref{sec-recon}).
\item[flagSmoothExists \texttt{[false]}] A flag to indicate whether the
  Hanning-smoothed array has been saved by a previous run of \duchamp. If
  set true, the smoothed array will be read from the file given
  by \texttt{smoothFile}, rather than calculated directly.
\item[smoothFile (no default assumed)] The FITS file that contains the
  Hanning-smoothed array. If \texttt{flagSmoothExists} is true and
  this parameter is not defined, the default file that is looked for
  will be determined by the Hanning width parameter (see
  \S\ref{sec-smoothing}).
\end{entry}

\secB*{Output related}
\begin{entry}
\item[OutFile \texttt{[duchamp-Results.txt]}] The file containing the
  final list of detections. This also records the list of input
  parameters.
\item[SpectraFile \texttt{[duchamp-Spectra.ps]}] The postscript file
  containing the resulting integrated spectra and images of the
  detections. 
\item[flagLog \texttt{[true]}] A flag to indicate whether the
  details of intermediate detections should be logged.
\item[LogFile \texttt{[duchamp-Logfile.txt]}] The file in which
  intermediate detections are logged. These are detections that have
  not been merged. This is primarily for use in debugging and
  diagnostic purposes -- normal use of the program will probably not
  require this.
\item[flagOutputRecon \texttt{[false]}] A flag to say whether or not
  to save the reconstructed cube as a FITS file. The filename will be
  derived according to the naming scheme detailed in
  Section~\ref{sec-reconIO}.
\item[flagOutputResid \texttt{[false]}] As for
  \texttt{flagOutputRecon}, but for the residual array -- the
  difference between the original cube and the reconstructed cube. The
  filename will be derived according to the naming scheme detailed in
  Section~\ref{sec-reconIO}.
\item[flagOutputSmooth \texttt{[false]}] A flag to say whether or not
  to save the smoothed cube as a FITS file. The filename will be
  derived according to the naming scheme detailed in
  Section~\ref{sec-smoothing}.
\item[flagVOT \texttt{[false]}] A flag to say whether to create a
  VOTable file corresponding to the information in
  \texttt{outfile}. This will be an XML file in the Virtual
  Observatory VOTable format.
\item[votFile \texttt{[duchamp-Results.xml]}] The VOTable file with
  the list of final detections. Some input parameters are also
  recorded.
\item[flagKarma \texttt{[false]}] A flag to say whether to create a
  Karma annotation file corresponding to the information in
  \texttt{outfile}. This can be used as an overlay for the Karma
  programs such as \texttt{kvis}.
\item[karmaFile \texttt{[duchamp-Results.ann]}] The Karma annotation
  file showing the list of final detections. 
\item[flagMaps \texttt{[true]}] A flag to say whether to save
  postscript files showing the 0th moment map of the whole cube
  (parameter \texttt{momentMap}) and the detection image
  (\texttt{detectionMap}).
\item[momentMap \texttt{[duchamp-MomentMap.ps]}] A postscript file
  containing a map of the 0th moment of the detected sources, as well
  as pixel and WCS coordinates.
\item[detectionMap \texttt{[duchamp-DetectionMap.ps]}] A postscript
  file showing each of the detected objects, coloured in greyscale by
  the number of channels spanned by each pixel. Also shows pixel and
  WCS coordinates.
\item[flagXOutput \texttt{[true]}] A flag to say whether to display a
  0th moment map in a PGPlot Xwindow. This will be in addition to any
  that are saved to a file.
\end{entry}

\secB*{Modifying the cube}
\begin{entry}
\item[flagBlankPix \texttt{[true]}] A flag to say whether to remove
  BLANK pixels from the analysis -- these are pixels set to some
  particular value because they fall outside the imaged area.
\item[blankPixValue \texttt{[-8.00061]}] The value of the BLANK
  pixels, if this information is not contained in the FITS header (the
  usual procedure is to obtain this value from the header information
  -- in which case the value set by this parameter is ignored).
\item[flagMW \texttt{[false]}] A flag to say whether to ignore
  channels contaminated by Milky Way (or other) emission -- the
  searching algorithms will not look at these channels.
\item[maxMW \texttt{[112]}] The maximum channel number that contains
  ``Milky Way'' emission.
\item[minMW \texttt{[75]}] The minimum channel number that contains
  ``Milky Way'' emission. Note that the range specified by
  \texttt{maxMW} and \texttt{minMW} is inclusive.
\item[flagBaseline \texttt{[false]}] A flag to say whether to remove
  the baseline from each spectrum in the cube for the purposes of
  reconstruction and detection.
\end{entry}

\secB*{Detection related}

\secC*{General detection}
\begin{entry}
\item[flagNegative \texttt{[false]}] A flag indicating that the
  features being searched for are negative. The cube will be inverted
  prior to searching.
\item[snrCut \texttt{[3.]}] The cut-off value for thresholding, in
  terms of number of $\sigma$ above the mean.
\item[threshold (no default assumed)] The actual value of the
  threshold. Normally the threshold is calculated from the cube's
  statistics, but the user can manually specify a value to be used
  that overrides the calculated value. If this is not specified, the
  calculated value is used. Also, when the FDR method is requested
  (see below), the value of the \texttt{threshold} parameter is
  ignored. 
\item[flagGrowth \texttt{[false]}] A flag indicating whether or not to
  grow the detected objects to a smaller threshold.
\item[growthCut \texttt{[2.]}] The smaller threshold using in growing
  detections. In units of $\sigma$ above the mean.
\item[beamSize \texttt{[10.]}] The size of the beam in pixels. If the
  header keywords BMAJ and BMIN are present, then these will be used
  to calculate the beam size, and this parameter will be ignored. 
\end{entry}

\secC*{\Atrous\ reconstruction}
\begin{entry}
\item [flagATrous \texttt{[true]}] A flag indicating whether or not to
  reconstruct the cube using the \atrous\ wavelet
  reconstruction. See \S\ref{sec-recon} for details.
\item[reconDim \texttt{[3]}] The number of dimensions to use in the
  reconstruction. 1 means reconstruct each spectrum separately, 2
  means each channel map is done separately, and 3 means do the whole
  cube in one go.
\item[scaleMin \texttt{[1]}] The minimum wavelet scale to be used in the
  reconstruction. A value of 1 means ``use all scales''.
\item[snrRecon \texttt{[4]}] The thresholding cutoff used in the
  reconstruction -- only wavelet coefficients this many $\sigma$ above
  the mean (or greater) are included in the reconstruction. 
\item[filterCode \texttt{[1]}] The code number of the filter to use in
  the reconstruction. The options are:
  \begin{itemize}
  \item \textbf{1:} B$_3$-spline filter: coefficients = 
    $(\frac{1}{16}, \frac{1}{4}, \frac{3}{8}, \frac{1}{4}, \frac{1}{16})$
  \item \textbf{2:} Triangle filter: coefficients = 
    $(\frac{1}{4}, \frac{1}{2}, \frac{1}{4})$
  \item \textbf{3:} Haar wavelet: coefficients = 
    $(0, \frac{1}{2}, \frac{1}{2})$
  \end{itemize}
\end{entry}

\secC*{Smoothing the cube}
\begin{entry}
\item [flagSmooth \texttt{[false]}] A flag indicating whether to
  Hanning-smooth the cube. See \S\ref{sec-smoothing} for details.
\item [hanningWidth \texttt{[5]}] The width of the Hanning smoothing
kernel. 
\end{entry}

\secC*{FDR method}
\begin{entry}
\item[flagFDR \texttt{[false]}] A flag indicating whether or not to use
  the False Discovery Rate method in thresholding the pixels.
\item[alphaFDR \texttt{[0.01]}] The $\alpha$ parameter used in the FDR
analysis. The average number of false detections, as a fraction of the
total number, will be less than $\alpha$ (see \S\ref{sec-detection}).
\end{entry}

\secC*{Merging detections}
\begin{entry}
\item[minPix \texttt{[2]}] The minimum number of spatial pixels for a
  single detection to be counted.
\item[minChannels \texttt{[3]}] The minimum number of consecutive
  channels that must be present in a detection.
\item[flagAdjacent \texttt{[true]}] A flag indicating whether to use
  the ``adjacent pixel'' criterion to decide whether to merge
  objects. If not, the next two parameters are used to determine
  whether objects are within the necessary thresholds.
\item[threshSpatial \texttt{[3.]}] The maximum allowed minimum spatial
  separation (in pixels) between two detections for them to be merged
  into one. Only used if \texttt{flagAdjacent = false}.
\item[threshVelocity \texttt{[7.]}] The maximum allowed minimum channel
  separation between two detections for them to be merged into
  one. 
\end{entry}

\secC*{Other parameters}
\begin{entry}
\item[spectralMethod \texttt{[peak]}] This indicates which method is used
  to plot the output spectra: \texttt{peak} means plot the spectrum
  containing the detection's peak pixel; \texttt{sum} means sum the
  spectra of each detected spatial pixel, and correct for the beam
  size. Any other choice defaults to \texttt{peak}.
\item[spectralUnits \texttt{[km/s]}] The user can specify the units of
  the spectral axis. Assuming the WCS of the FITS file is valid, the
  spectral axis is transformed into velocity, and put into these units
  for all output and for calculations such as the integrated flux of a
  detection.
\item[drawBorders \texttt{[true]}] A flag indicating whether borders
  are to be drawn around the detected objects in the moment maps
  included in the output (see for example Fig.~\ref{fig-spect}).
\item[drawBlankEdges \texttt{[true]}] A flag indicating whether to
 draw the dividing line between BLANK and non-BLANK pixels on the
 2-dimensional images (see for example Fig.~\ref{fig-moment}).
\item[verbose \texttt{[true]}] A flag indicating whether to print the
  progress of any computationally-intensive algorithms (\eg
  reconstruction, searching or merging algorithms) to the screen.
\end{entry}

 %Available parameters
\newpage
% -----------------------------------------------------------------------
% app-paramEx.tex: Example input parameter files, and how the
%                  parameters are listed in the output.
% -----------------------------------------------------------------------
% Copyright (C) 2006, Matthew Whiting, ATNF
%
% This program is free software; you can redistribute it and/or modify it
% under the terms of the GNU General Public License as published by the
% Free Software Foundation; either version 2 of the License, or (at your
% option) any later version.
%
% Duchamp is distributed in the hope that it will be useful, but WITHOUT
% ANY WARRANTY; without even the implied warranty of MERCHANTABILITY or
% FITNESS FOR A PARTICULAR PURPOSE.  See the GNU General Public License
% for more details.
%
% You should have received a copy of the GNU General Public License
% along with Duchamp; if not, write to the Free Software Foundation,
% Inc., 59 Temple Place, Suite 330, Boston, MA 02111-1307, USA
%
% Correspondence concerning Duchamp may be directed to:
%    Internet email: Matthew.Whiting [at] atnf.csiro.au
%    Postal address: Dr. Matthew Whiting
%                    Australia Telescope National Facility, CSIRO
%                    PO Box 76
%                    Epping NSW 1710
%                    AUSTRALIA
% -----------------------------------------------------------------------
\secA{Example parameter files}
\label{app-input}

This is what a typical parameter file would look like.

\begin{verbatim}
imageFile       /home/mduchamp/fountain.fits
logFile         logfile.txt
outFile         results.txt
spectraFile     spectra.ps
flagSubsection  false
flagOutputRecon false
flagOutputResid 0
flagTrim        1
flaggedChannels 75-112
flagGrowth      1
growthCut       1.5
flagATrous      1
reconDim        1          
scaleMin        1
snrRecon        4
flagFDR         1
alphaFDR        0.1
snrCut          3
threshSpatial   3
threshVelocity  7
\end{verbatim}

Note that, as in this example, the flag parameters can be entered as
strings (\texttt{true}/\texttt{false}) or integers
(\texttt{1}/\texttt{0}). Also, note that it is not necessary to
include all these parameters in the file, only those that need to be
changed from the defaults (as listed in Appendix~\ref{app-param}),
which in this case would be very few. A minimal parameter file might
look like:
\begin{verbatim}
imageFile       /home/mduchamp/fountain.fits
flagLog         false
flagATrous      1
snrRecon        3
snrCut          2.5
minChannels     4
\end{verbatim}
This will reconstruct the cube with a lower SNR value than the
default, select objects at a lower threshold,  with a looser minimum
channel requirement, and not keep a log of the intermediate
detections. 

The following page demonstrates how the parameters are presented to
the user, both on the screen at execution time, and in the output and
log files. On each line, there is a description on the parameter, the
relevant parameter name that is used in the input file (if there is
one that the user can enter), and the value of the parameter being
used.

\newpage
{\scriptsize
\begin{verbatim}
# ---- Parameters ----
# Image to be analysed.............................[imageFile]  =  fountain.fits
# Intermediate Logfile...............................[logFile]  =  duchamp-Logfile.txt
# Final Results file.................................[outFile]  =  duchamp-Results.txt
# Header for results file.........................[headerFile]  =  duchamp-Results.hdr
# Spectrum file..................................[spectraFile]  =  duchamp-Spectra.ps
# Text file with ascii spectral data.........[spectraTextFile]  =  duchamp-Spectra.txt
# VOTable file.......................................[votFile]  =  duchamp-Results.xml
# Karma annotation file............................[karmaFile]  =  duchamp-Results.ann
# DS9 annotation file................................[ds9File]  =  duchamp-Results.reg
# CASA annotation file..............................[casaFile]  =  duchamp-Results.crf
# 0th Moment Map...................................[momentMap]  =  duchamp-MomentMap.ps
# Detection Map.................................[detectionMap]  =  duchamp-DetectionMap.ps
# Display a map in a pgplot xwindow?.............[flagXOutput]  =  true
# Saving reconstructed cube?.................[flagOutputRecon]  =  true --> fountain.RECON-1-1-4-1-8-0.005.fits
# Saving residuals from reconstruction?......[flagOutputResid]  =  true --> latestResid.fits
# Saving mask cube?...........................[flagOutputMask]  =  true --> latestmask2.fits
# Saving 0th moment to FITS file?........[flagOutputMomentMap]  =  true --> latestmom0.fits
# Saving 0th moment mask to FITS file?..[flagOutputMomentMask]  =  true --> latestmom0mask.fits
# Saving baseline values to FITS file?....[flagOutputBaseline]  =  false
# ------
# Type of searching performed.....................[searchType]  =  spectral
# Blank Pixel Value...........................................  =  -8.00061
# Trimming Blank Pixels?............................[flagTrim]  =  false
# Searching for Negative features?..............[flagNegative]  =  false
# Channels flagged by user...................[flaggedChannels]  =  75-112
# Area of Beam (pixels).......................................  =  14.6848   (beam: 3.6 x 3.6 pixels)
# Removing baselines before search?.............[flagBaseline]  =  false
# Smoothing data prior to searching?..............[flagSmooth]  =  false
# Using A Trous reconstruction?...................[flagATrous]  =  true
# Number of dimensions in reconstruction............[reconDim]  =  1
# Scales used in reconstruction............[scaleMin-scaleMax]  =  1-8
# SNR Threshold within reconstruction...............[snrRecon]  =  4
# Residual convergence criterion............[reconConvergence]  =  0.005
# Filter being used for reconstruction............[filterCode]  =  1 (B3 spline function)
# Using Robust statistics?...................[flagRobustStats]  =  true
# Using FDR analysis?................................[flagFDR]  =  false
# SNR Threshold (in sigma)............................[snrCut]  =  3.5
# Minimum # Pixels in a detection.....................[minPix]  =  5
# Minimum # Channels in a detection..............[minChannels]  =  3
# Minimum # Voxels in a detection..................[minVoxels]  =  7
# Growing objects after detection?................[flagGrowth]  =  false
# Using Adjacent-pixel criterion?...............[flagAdjacent]  =  true
# Max. velocity separation for merging........[threshVelocity]  =  7
# Reject objects before merging?.......[flagRejectBeforeMerge]  =  false
# Merge objects in two stages?...........[flagTwoStageMerging]  =  false
# Method of spectral plotting.................[spectralMethod]  =  peak
# Type of object centre used in results..........[pixelCentre]  =  centroid
# --------------------
\end{verbatim}
}
%\end{minipage}
%\end{sideways}

%%% Local Variables: 
%%% mode: latex
%%% TeX-master: "Guide"
%%% End: 
 %Example parameter files

\begin{landscape}
\secA{Example results file}
\label{app-output}
This the typical content of an output file, after running \duchamp\
with the parameters illustrated on the previous page. 

{\tiny 
  \begin{verbatim}
Results of the Duchamp source finder: Tue May 23 14:51:38 2006
---- Parameters ----
      (... omitted for clarity -- see previous page for examples...)
--------------------
Detection threshold = 0.0373519 Jy/beam
  Noise level = 0.000122074, Noise spread = 0.0124099
Total number of detections = 22
--------------------
------------------------------------------------------------------------------------------------------------------------------------------------------------------
 Obj#           Name     X     Y     Z           RA          DEC      VEL     w_RA    w_DEC    w_VEL     F_int    F_peak S/Nmax  X1  X2  Y1  Y2  Z1  Z2  Npix Flag
                                                                   [km/s] [arcmin] [arcmin]   [km/s] [Jy km/s] [Jy/beam]                                [pix]     
------------------------------------------------------------------------------------------------------------------------------------------------------------------
    1 J060919-215700  59.5 140.5 114.5  06:09:19.25 -21:57:00.84  223.839    44.66    51.51   39.574    16.969     0.213  17.12  55  65 133 145 113 116   170     
    2 J054555-214432 141.0 142.9 114.7  05:45:55.07 -21:44:32.92  226.599    23.52    20.78   26.383     2.877     0.090   7.23 138 143 141 145 114 116    35     
    3 J061722-263336  33.5  70.8 115.5  06:17:22.66 -26:33:36.04  237.560    64.92    26.11   26.383    11.182     0.117   9.44  26  41  68  74 115 117   120    E
    4 J060142-250018  86.0  94.9 117.9  06:01:42.69 -25:00:18.58  269.025    27.99    24.02   26.383     4.404     0.124   9.98  83  89  92  97 117 119    51     
    5 J060218-254650  84.0  83.3 117.9  06:02:18.24 -25:46:50.45  269.277    20.01    23.99   26.383     3.295     0.118   9.49  82  86  81  86 117 119    38     
    6 J060610-271934  71.1  60.1 121.3  06:06:10.94 -27:19:34.16  313.788    52.36    39.59   26.383    14.040     0.150  12.05  65  77  55  64 120 122   149     
    7 J061119-213726  52.5 145.3 162.6  06:11:19.99 -21:37:26.56  858.208    32.39    23.49  118.722    42.903     0.410  33.05  49  56 142 147 158 167   256    E
    8 J060034-285855  89.7  35.3 202.3  06:00:34.64 -28:58:55.56 1382.114    23.93    24.10  171.487    23.577     0.173  13.92  87  92  33  38 196 209   255     
    9 J055853-263852  95.4  70.2 223.2  05:58:53.52 -26:38:52.62 1658.705     7.95     8.05   92.339     0.752     0.063   5.07  95  96  70  71 220 227    12     
   10 J061707-272344  34.7  58.3 227.4  06:17:07.40 -27:23:44.93 1712.894    16.64    19.53  290.209     7.237     0.093   7.48  33  36  56  60 215 237    95     
   11 J055849-252525  95.8  88.6 231.8  05:58:49.15 -25:25:25.60 1771.131    23.88    20.14  237.444    11.712     0.115   9.30  93  98  87  91 221 239   154     
   12 J061524-263409  40.0  70.9 232.4  06:15:24.79 -26:34:09.27 1779.710    12.44    15.69   52.765     1.803     0.068   5.51  39  41  69  72 231 235    26     
   13 J060054-214152  88.8 144.5 232.5  06:00:54.01 -21:41:52.68 1781.012    27.96    24.13  224.252    29.538     0.166  13.33  86  92 142 147 222 239   322    E
   14 J060441-260613  75.9  78.4 232.9  06:04:41.33 -26:06:13.16 1785.878    20.12    23.90  211.061    22.397     0.155  12.44  74  78  76  81 225 241   256     
   15 J060108-234023  87.9 114.9 235.8  06:01:08.76 -23:40:23.49 1823.639    27.96    28.07  237.444    81.622     0.297  23.90  85  91 112 118 227 245   628     
   16 J061733-230557  31.4 122.7 258.2  06:17:33.33 -23:05:57.88 2119.133     8.33    11.77   26.383     0.848     0.062   5.02  31  32 122 124 257 259    13     
   17 J061210-214929  49.6 142.2 270.1  06:12:10.85 -21:49:29.47 2276.626    16.27    15.73  395.740    13.315     0.101   8.12  48  51 141 144 257 287   172     
   18 J061616-213319  35.3 145.9 299.4  06:16:16.08 -21:33:19.50 2663.021    20.22     7.47  211.061     2.877     0.127  10.23  33  37 145 146 294 310    26    E
   19 J055508-295540 107.3  21.0 367.5  05:55:08.50 -29:55:40.89 3561.149    19.71    20.30   39.574     5.486     0.169  13.62 105 109  19  23 366 369    49     
   20 J055743-224642  99.8 128.2 434.0  05:57:43.77 -22:46:42.95 4438.776    11.88    16.12  105.531     1.703     0.167  13.48  99 101 127 130 430 438    17    N
   21 J061603-264802  38.0  67.4 546.6  06:16:03.03 -26:48:02.76 5924.685    12.35    11.67   26.383     1.017     0.064   5.16  37  39  66  68 546 548    14     
   22 J055213-291656 116.9  30.5 726.8  05:52:13.81 -29:16:57.00 8300.595    11.59    20.25  290.209    35.296     0.479  38.56 116 118  28  32 716 738   110     
  \end{verbatim}
}

Note that the width of the table can make it hard to read. A good
trick for those using UNIX/Linux is to make use of the \texttt{a2ps}
command. The following works well, producing a postscript file
\texttt{duchamp-Results.ps}:
\\\verb|a2ps -1 -r -f7 -o duchamp-Results.ps duchamp-Results.txt|
\end{landscape}
 %Example results files

\begin{landscape}
\secA{Example VOTable output}
\label{app-votable}
This is part of the VOTable, in XML format, corresponding to a basic
FDR search.

%\begin{sideways}
%\begin{minipage}[b]{180mm}
{\tiny
  \begin{verbatim}
<?xml version="1.0"?>
<VOTABLE version="1.1" xmlns:xsi="http://www.w3.org/2001/XMLSchema-instance"
 xsi:noNamespaceSchemaLocation="http://www.ivoa.net/xml/VOTable/VOTable/v1.1">
  <COOSYS ID="J2000" equinox="J2000." epoch="J2000." system="eq_FK5"/>
  <RESOURCE name="Duchamp Output">
    <TABLE name="Detections">
      <DESCRIPTION>Detected sources and parameters from running the Duchamp source finder.</DESCRIPTION>
      <PARAM name="FITS file" datatype="char" ucd="meta.file;meta.fits" value="/DATA/SITAR_1/whi550/ObsData/cubes/H201_abcde_luther_chop.fits" arraysize="62"/>
      <PARAM name="FDR Significance" datatype="float" ucd="stat.param" value="0.1"/>
      <FIELD name="ID" ID="col01" ucd="meta.id" datatype="int" width="5" unit="--"/>
      <FIELD name="Name" ID="col02" ucd="meta.id;meta.main" datatype="char" arraysize="15" unit="--"/>
      <FIELD name="RA" ID="col03" ucd="pos.eq.ra;meta.main" ref="J2000" datatype="float" width="13" precision="6" unit="deg"/>
      <FIELD name="DEC" ID="col04" ucd="pos.galactic.lon;meta.main" ref="J2000" datatype="float" width="13" precision="6" unit="deg"/>
      <FIELD name="w_RA" ID="col05" ucd="phys.angSize;pos.eq.ra" ref="J2000" datatype="float" width="9" precision="2" unit="arcmin"/>
      <FIELD name="w_DEC" ID="col06" ucd="phys.angSize;pos.eq.ra" ref="J2000" datatype="float" width="9" precision="2" unit="arcmin"/>
      <FIELD name="Vel" ID="col07" ucd="phys.veloc;src.dopplerVeloc" datatype="float" width="9" precision="3" unit="km/s"/>
      <FIELD name="w_Vel" ID="col08" ucd="phys.veloc;src.dopplerVeloc;spect.line.width" datatype="float" width="9" precision="3" unit="km/s"/>
      <FIELD name="Integrated_Flux" ID="col09" ucd="phot.flux;spect.line.intensity" datatype="float" width="10" precision="3" unit="Jy.km/s"/>
      <FIELD name="Peak_Flux" ID="col10" ucd="phot.flux;spect.line.intensity" datatype="float" width="10" precision="3" unit="Jy/beam"/>
      <FIELD name="Flag" ID="col11" ucd="meta.code.qual" datatype="char" arraysize="3" unit="--"/>
      <FIELD name="X_Centroid" ID="col12" ucd="pos.cartesian.x" datatype="float" width="7" precision="1" unit=""/>
      <FIELD name="Y_Centroid" ID="col13" ucd="pos.cartesian.y" datatype="float" width="7" precision="1" unit=""/>
      <FIELD name="Z_Centroid" ID="col14" ucd="pos.cartesian.z" datatype="float" width="7" precision="1" unit=""/>
      <FIELD name="X_Av" ID="col15" ucd="pos.cartesian.x" datatype="float" width="6" precision="1" unit=""/>
      <FIELD name="Y_Av" ID="col16" ucd="pos.cartesian.y" datatype="float" width="6" precision="1" unit=""/>
      <FIELD name="Z_Av" ID="col17" ucd="pos.cartesian.z" datatype="float" width="6" precision="1" unit=""/>
      <FIELD name="X_Peak" ID="col18" ucd="pos.cartesian.x" datatype="int" width="7" precision="1" unit=""/>
      <FIELD name="Y_Peak" ID="col19" ucd="pos.cartesian.y" datatype="int" width="7" precision="1" unit=""/>
      <FIELD name="Z_Peak" ID="col20" ucd="pos.cartesian.z" datatype="int" width="7" precision="1" unit=""/>
      <DATA>
        <TABLEDATA>
        <TR>
          <TD>    1</TD><TD> J060925-215712</TD><TD>    92.356613</TD><TD>   -21.953545</TD><TD>    44.51</TD><TD>    39.49</TD><TD>  223.351</TD><TD>   52.765</TD><TD>    14.564</TD>
<TD>     0.213</TD><TD>     </TD><TD> 59.175</TD><TD>140.463</TD><TD>114.439</TD><TD>59.554</TD><TD>140.580</TD><TD>114.536</TD><TD>     59</TD><TD>    140</TD><TD>    114</TD>
        </TR>
... truncated ...
        </TABLEDATA>
      </DATA>
    </TABLE>
  </RESOURCE>
</VOTABLE>
  \end{verbatim}
}
%\end{minipage}
%\end{sideways}
\end{landscape} %Example VOTable output

% -----------------------------------------------------------------------
% app-Karma.tex: Example output annotation file for Karma software.
% -----------------------------------------------------------------------
% Copyright (C) 2006, Matthew Whiting, ATNF
%
% This program is free software; you can redistribute it and/or modify it
% under the terms of the GNU General Public License as published by the
% Free Software Foundation; either version 2 of the License, or (at your
% option) any later version.
%
% Duchamp is distributed in the hope that it will be useful, but WITHOUT
% ANY WARRANTY; without even the implied warranty of MERCHANTABILITY or
% FITNESS FOR A PARTICULAR PURPOSE.  See the GNU General Public License
% for more details.
%
% You should have received a copy of the GNU General Public License
% along with Duchamp; if not, write to the Free Software Foundation,
% Inc., 59 Temple Place, Suite 330, Boston, MA 02111-1307, USA
%
% Correspondence concerning Duchamp may be directed to:
%    Internet email: Matthew.Whiting [at] atnf.csiro.au
%    Postal address: Dr. Matthew Whiting
%                    Australia Telescope National Facility, CSIRO
%                    PO Box 76
%                    Epping NSW 1710
%                    AUSTRALIA
% -----------------------------------------------------------------------
\secA{Example Karma Annotation file output}
\label{app-karma}

This is the format of the Karma Annotation file, showing the locations
of the detected objects. This can be loaded by the plotting tools of
the Karma package (for instance, \texttt{kvis}) as an overlay on the FITS
file.

\begin{verbatim}
# Duchamp Source Finder v.1.3.1
# Results for FITS file: /home/mduchamp/fountain.fits
# imageFile              /home/mduchamp/fountain.fits
# flagSubsection         0
# flagStatSec            0
# searchType             spatial
# flagNegative           0
# flagBaseline           0
# flagRobustStats        1
# flagFDR                0
# snrCut                 4
# flagGrowth             0
# minVoxels              4
# minPix                 2
# minChannels            2
# flagAdjacent           1
# threshVelocity         3
# flagRejectBeforeMerge  0
# flagTwoStageMerging    1
# pixelCentre            centroid
# flagSmooth             0
# flagATrous             0
# Detection threshold used = 4.05187
# Mean of noise background = 0.0148019
# Std. Deviation of noise background = 1.00927
#  [Using robust methods]
COLOR RED
COORD W
LINE 92.796776 -27.682375 92.721457 -27.683528
LINE 92.795375 -27.615691 92.720102 -27.616844
LINE 92.722816 -27.750214 92.647453 -27.751330
LINE 92.714730 -27.350128 92.639640 -27.351239
...
TEXT 92.301572 -27.583707 1
\end{verbatim}

%%% Local Variables: 
%%% mode: latex
%%% TeX-master: "Guide"
%%% End: 
 %Example Karma annotation file output
\newpage
% -----------------------------------------------------------------------
% app-stats.tex: Section on how the robust statistics are calculated
%                for a Normal distribution.
% -----------------------------------------------------------------------
% Copyright (C) 2006, Matthew Whiting, ATNF
%
% This program is free software; you can redistribute it and/or modify it
% under the terms of the GNU General Public License as published by the
% Free Software Foundation; either version 2 of the License, or (at your
% option) any later version.
%
% Duchamp is distributed in the hope that it will be useful, but WITHOUT
% ANY WARRANTY; without even the implied warranty of MERCHANTABILITY or
% FITNESS FOR A PARTICULAR PURPOSE.  See the GNU General Public License
% for more details.
%
% You should have received a copy of the GNU General Public License
% along with Duchamp; if not, write to the Free Software Foundation,
% Inc., 59 Temple Place, Suite 330, Boston, MA 02111-1307, USA
%
% Correspondence concerning Duchamp may be directed to:
%    Internet email: Matthew.Whiting [at] atnf.csiro.au
%    Postal address: Dr. Matthew Whiting
%                    Australia Telescope National Facility, CSIRO
%                    PO Box 76
%                    Epping NSW 1710
%                    AUSTRALIA
% -----------------------------------------------------------------------
\secA{Robust statistics for a Normal distribution}
\label{app-madfm}

The Normal, or Gaussian, distribution for mean $\mu$ and standard
deviation $\sigma$ can be written as 
\[ 
f(x) = \frac{1}{\sqrt{2\pi\sigma^2}}\ e^{-(x-\mu)^2/2\sigma^2}.
 \]

When one has a purely Gaussian signal, it is straightforward to
estimate $\sigma$ by calculating the standard deviation (or rms) of
the data. However, if there is a small amount of signal present on top
of Gaussian noise, and one wants to estimate the $\sigma$ for the
noise, the presence of the large values from the signal can bias the
estimator to higher values.

An alternative way is to use the median ($m$) and median absolute
deviation from the median ($s$) to estimate $\mu$ and $\sigma$. The
median is the middle of the distribution, defined for a continuous
distribution by
\[
\int_{-\infty}^{m} f(x) dx = \int_{m}^{\infty} f(x) dx.
\]
From symmetry, we quickly see that for the continuous Normal
distribution, $m=\mu$. We consider the case henceforth of $\mu=0$,
without loss of generality.

To find $s$, we find the distribution of the absolute deviation from
the median, and then find the median of that distribution. This
distribution is given by
\begin{eqnarray*}
g(x) &= &{\text{distribution of }} |x|\\
     &= &f(x) + f(-x),\ x\ge0\\
     &= &\sqrt{\frac{2}{\pi\sigma^2}}\, e^{-x^2/2\sigma^2},\ x\ge0.
\end{eqnarray*}
So, the median absolute deviation from the median, $s$, is given by
\[
\int_{0}^{s} g(x) dx = \int_{s}^{\infty} g(x) dx.
\]
If we use the identity
\[
\int_{0}^{\infty}e^{-x^2/2\sigma^2} dx = \sqrt{\pi\sigma^2/2}
\] 
we find that 
\[
\int_{s}^{\infty} e^{-x^2/2\sigma^2} dx =
\sqrt{\pi\sigma^2/2}-\int_{0}^{s} e^{-x^2/2\sigma^2}dx.
\]
Hence, to find $s$ we simply solve the following equation (setting
$\sigma=1$ for simplicity -- equivalent to stating $x$ and $s$ in
units of $\sigma$):
\[
\int_{0}^{s}e^{-x^2/2} dx - \sqrt{\pi/8} = 0.
\]
This is hard to solve analytically (no nice analytic solution exists
for the finite integral that I'm aware of), but straightforward to
solve numerically, yielding the value of $s=0.6744888$. Thus, to
estimate $\sigma$ for a Normally distributed data set, one can
calculate $s$, then divide by 0.6744888 (or multiply by 1.4826042) to
obtain the correct estimator.

Note that this is different to solutions quoted elsewhere,
specifically in \citet{meyer04-alt}, where the same robust estimator
is used but with an incorrect conversion to standard deviation -- they
assume $\sigma = s\sqrt{\pi/2}$. This, in fact, is the conversion used
to convert the \emph{mean} absolute deviation from the mean to the
standard deviation. This means that the cube noise in the \hipass
catalogue (their parameter Rms$_{\rm cube}$) should be 18\% larger
than quoted.
 %Robust statistics for a Normal distribution

\secA{How Gaussian noise changes with wavelet scale}
\label{app-scaling}

The key element in the wavelet reconstruction of an array is the
thresholding of the individual wavelet coefficient arrays. This is
usually done by choosing a level to be some number of standard
deviations above the mean value.

However, since the wavelet arrays are produced by convolving the input
array by an increasingly large filter, the pixels in the coefficient
arrays become increasingly correlated as the scale of the filter
increases. This results in the measured standard deviation from a
given coefficient array decreasing with increasing scale. To calculate
this, we need to take into account how many other pixels each pixel in
the convolved array depends on.

To demonstrate, suppose we have a 1-D array with $N$ pixel values
given by $F_i,\ i=1,...,N$, and we convolve it with the B$_3$-spline
filter, defined by the set of coefficients
$\{1/16,1/4,3/8,1/4,1/16\}$. The flux of the $i$th pixel in the
convolved array will be
\[
F'_i = \frac{1}{16}F_{i-2} + \frac{1}{4}F_{i-1} + \frac{3}{8}F_{i}
+ \frac{1}{4}F_{i+1} + \frac{1}{16}F_{i+2}
\]
and the flux of the corresponding pixel in the wavelet array will be 
\[
W'_i = F_i - F'_i = \frac{-1}{16}F_{i-2} - \frac{1}{4}F_{i-1} 
+ \frac{5}{8}F_{i} - \frac{1}{4}F_{i+1} - \frac{1}{16}F_{i+2}
\]
Now, assuming each pixel has the same standard deviation
$\sigma_i=\sigma$, we can work out the standard deviation for the
wavelet array:
\[
\sigma'_i = \sigma \sqrt{\left(\frac{1}{16}\right)^2 
  + \left(\frac{1}{4}\right)^2 + \left(\frac{5}{8}\right)^2 
  + \left(\frac{1}{4}\right)^2 + \left(\frac{1}{16}\right)^2}
          = 0.72349\ \sigma
\]
Thus, the first scale wavelet coefficient array will have a standard
deviation of 72.3\% of the input array. This procedure can be followed
to calculate the necessary values for all scales, dimensions and
filters used by \duchamp.

Calculating these values is clearly a critical step in performing the
reconstruction. \citet{starck02:book} did so by simulating data sets
with Gaussian noise, taking the wavelet transform, and measuring the
value of $\sigma$ for each scale. We take a different approach, by
calculating the scaling factors directly from the filter coefficients
by taking the wavelet transform of an array made up of a 1 in the
central pixel and 0s everywhere else. The scaling value is then
derived by taking the square root of the sum (in quadrature) of all
the wavelet coefficient values at each scale. We give the scaling
factors for the three filters available to \duchamp\ on the following
page. These values are hard-coded into \duchamp, so no on-the-fly
calculation of them is necessary.

Memory limitations prevent us from calculating factors for large
scales, particularly for the three-dimensional case (hence the --
symbols in the tables). To calculate factors for higher scales than
those available, we note the following relationships apply for large
scales to a sufficient level of precision:
\begin{itemize}
\item 1-D: factor(scale $i$) = factor(scale $i-1$)$/\sqrt{2}$.
\item 2-D: factor(scale $i$) = factor(scale $i-1$)$/2$.
\item 1-D: factor(scale $i$) = factor(scale $i-1$)$/\sqrt{8}$.
\end{itemize}

\newpage
\begin{itemize}
\item \textbf{B$_3$-Spline Function:} $\{1/16,1/4,3/8,1/4,1/16\}$

\begin{tabular}{llll}
Scale & 1 dimension      & 2 dimension     & 3 dimension\\ \hline
1     & 0.723489806      & 0.890796310     & 0.956543592\\
2     & 0.285450405	 & 0.200663851	   & 0.120336499\\
3     & 0.177947535	 & 0.0855075048	   & 0.0349500154\\
4     & 0.122223156	 & 0.0412474444	   & 0.0118164242\\
5     & 0.0858113122	 & 0.0204249666	   & 0.00413233507\\
6     & 0.0605703043	 & 0.0101897592	   & 0.00145703714\\
7     & 0.0428107206	 & 0.00509204670   & 0.000514791120\\
8     & 0.0302684024	 & 0.00254566946   & --\\
9     & 0.0214024008	 & 0.00127279050   & --\\
10    & 0.0151336781	 & 0.000636389722  & --\\
11    & 0.0107011079	 & 0.000318194170  & --\\
12    & 0.00756682272	 & --		   & --\\
13    & 0.00535055108	 & --		   & --\\
%14    & 0.00378341085	 & --		   & --\\
%15    & 0.00267527545	 & --		   & --\\
%16    & 0.00189170541	 & --		   & --\\
%17    & 0.00133763772	 & --		   & --\\
%18    & 0.000945852704   & --		   & --
\end{tabular}

\item \textbf{Triangle Function:} $\{1/4,1/2,1/4\}$

\begin{tabular}{llll}
Scale & 1 dimension      & 2 dimension     & 3 dimension\\ \hline
1     & 0.612372436      & 0.800390530     & 0.895954449  \\
2     & 0.330718914	 & 0.272878894     & 0.192033014\\
3     & 0.211947812	 & 0.119779282     & 0.0576484078\\
4     & 0.145740298	 & 0.0577664785    & 0.0194912393\\
5     & 0.102310944	 & 0.0286163283    & 0.00681278387\\
6     & 0.0722128185	 & 0.0142747506    & 0.00240175885\\
7     & 0.0510388224	 & 0.00713319703   & 0.000848538128 \\
8     & 0.0360857673	 & 0.00356607618   & 0.000299949455 \\
9     & 0.0255157615	 & 0.00178297280   & -- \\
10    & 0.0180422389	 & 0.000891478237  & --  \\
11    & 0.0127577667	 & 0.000445738098  & --  \\
12    & 0.00902109930	 & 0.000222868922  & --  \\
13    & 0.00637887978	 & --		   & -- \\
%14   & 0.00451054902	 & --		   & -- \\
%15   & 0.00318942978	 & --		   & -- \\
%16   & 0.00225527449	 & --		   & -- \\
%17   & 0.00159471988	 & --		   & -- \\
%18   & 0.000112763724	 & --		   & -- 

\end{tabular}

\item \textbf{Haar Wavelet:} $\{0,1/2,1/2\}$

\begin{tabular}{llll}
Scale & 1 dimension      & 2 dimension     & 3 dimension\\ \hline
1     & 0.707167810      & 0.433012702     & 0.935414347 \\
2     & 0.500000000	 & 0.216506351     & 0.330718914\\
3     & 0.353553391	 & 0.108253175     & 0.116926793\\
4     & 0.250000000	 & 0.0541265877    & 0.0413398642\\
5     & 0.176776695	 & 0.0270632939    & 0.0146158492\\
6     & 0.125000000	 & 0.0135316469    & 0.00516748303

\end{tabular}


\end{itemize}
 %How Gaussian noise changes with wavelet scale


\end{document}
