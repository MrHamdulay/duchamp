% -----------------------------------------------------------------------
% app-param.tex: Section listing all the possible input parameters and
%                their defaults.
% -----------------------------------------------------------------------
% Copyright (C) 2006, Matthew Whiting, ATNF
%
% This program is free software; you can redistribute it and/or modify it
% under the terms of the GNU General Public License as published by the
% Free Software Foundation; either version 2 of the License, or (at your
% option) any later version.
%
% Duchamp is distributed in the hope that it will be useful, but WITHOUT
% ANY WARRANTY; without even the implied warranty of MERCHANTABILITY or
% FITNESS FOR A PARTICULAR PURPOSE.  See the GNU General Public License
% for more details.
%
% You should have received a copy of the GNU General Public License
% along with Duchamp; if not, write to the Free Software Foundation,
% Inc., 59 Temple Place, Suite 330, Boston, MA 02111-1307, USA
%
% Correspondence concerning Duchamp may be directed to:
%    Internet email: Matthew.Whiting [at] atnf.csiro.au
%    Postal address: Dr. Matthew Whiting
%                    Australia Telescope National Facility, CSIRO
%                    PO Box 76
%                    Epping NSW 1710
%                    AUSTRALIA
% -----------------------------------------------------------------------
\secA{Available parameters}
\label{app-param}

The full list of parameters that can be listed in the input file are
given here. If not listed, they take the default value given in
parentheses. Since the order of the parameters in the input file does
not matter, they are grouped here in logical sections.

\secB*{Input related}
\begin{entry}
\item[ImageFile (no default assumed)] The filename of the
  data cube to be analysed.
\item[flagSubsection \texttt{[false]}] A flag to indicate whether one
  wants a subsection of the requested image.
\item[Subsection \texttt{[ [*,*,*] ]}] The requested subsection
 -- see \S\ref{sec-input} for details on the subsection format.  If
 the full range of a dimension is required, use a \texttt{*} (thus the
 default is the full cube).
\item[flagReconExists \texttt{[false]}] A flag to indicate whether the
  reconstructed array has been saved by a previous run of \duchamp. If
  set true, the reconstructed array will be read from the file given
  by \texttt{reconFile}, rather than calculated directly.
\item[reconFile (no default assumed)] The FITS file that contains the
  reconstructed array. If \texttt{flagReconExists} is true and this
  parameter is not defined, the default file that is looked for will
  be determined by the \atrous parameters (see \S\ref{sec-recon}).
\item[flagSmoothExists \texttt{[false]}] A flag to indicate whether the
  Hanning-smoothed array has been saved by a previous run of \duchamp. If
  set true, the smoothed array will be read from the file given
  by \texttt{smoothFile}, rather than calculated directly.
\item[smoothFile (no default assumed)] The FITS file that contains the
  Hanning-smoothed array. If \texttt{flagSmoothExists} is true and
  this parameter is not defined, the default file that is looked for
  will be determined by the Hanning width parameter (see
  \S\ref{sec-smoothing}).
\end{entry}

\secB*{Output related}
\begin{entry}
\item[OutFile \texttt{[duchamp-Results.txt]}] The file containing the
  final list of detections. This also records the list of input
  parameters.
\item[SpectraFile \texttt{[duchamp-Spectra.ps]}] The postscript file
  containing the resulting integrated spectra and images of the
  detections. 
\item[flagLog \texttt{[false]}] A flag to indicate whether the
  details of intermediate detections should be logged.
\item[LogFile \texttt{[duchamp-Logfile.txt]}] The file in which
  intermediate detections are logged. These are detections that have
  not been merged. This is primarily for use in debugging and
  diagnostic purposes -- normal use of the program will probably not
  require this.
\item[flagOutputRecon \texttt{[false]}] A flag to say whether or not
  to save the reconstructed cube as a FITS file. The filename will be
  derived according to the naming scheme detailed in
  Section~\ref{sec-reconIO}.
\item[flagOutputResid \texttt{[false]}] As for
  \texttt{flagOutputRecon}, but for the residual array -- the
  difference between the original cube and the reconstructed cube. The
  filename will be derived according to the naming scheme detailed in
  Section~\ref{sec-reconIO}.
\item[flagOutputSmooth \texttt{[false]}] A flag to say whether or not
  to save the smoothed cube as a FITS file. The filename will be
  derived according to the naming scheme detailed in
  Section~\ref{sec-smoothing}.
\item[flagVOT \texttt{[false]}] A flag to say whether to create a
  VOTable file corresponding to the information in
  \texttt{outfile}. This will be an XML file in the Virtual
  Observatory VOTable format.
\item[votFile \texttt{[duchamp-Results.xml]}] The VOTable file with
  the list of final detections. Some input parameters are also
  recorded.
\item[flagKarma \texttt{[false]}] A flag to say whether to create a
  Karma annotation file corresponding to the information in
  \texttt{outfile}. This can be used as an overlay for the Karma
  programs such as \texttt{kvis}.
\item[karmaFile \texttt{[duchamp-Results.ann]}] The Karma annotation
  file showing the list of final detections. 
\item[flagMaps \texttt{[true]}] A flag to say whether to save
  postscript files showing the 0th moment map of the whole cube
  (parameter \texttt{momentMap}) and the detection image
  (\texttt{detectionMap}).
\item[momentMap \texttt{[duchamp-MomentMap.ps]}] A postscript file
  containing a map of the 0th moment of the detected sources, as well
  as pixel and WCS coordinates.
\item[detectionMap \texttt{[duchamp-DetectionMap.ps]}] A postscript
  file showing each of the detected objects, coloured in greyscale by
  the number of channels spanned by each pixel. Also shows pixel and
  WCS coordinates.
\item[flagXOutput \texttt{[true]}] A flag to say whether to display a
  0th moment map in a PGPlot X-window. This will be in addition to any
  that are saved to a file. This parameter can be overridden by the
  use of the \texttt{-x} command-line option, which disables the
  X-windows output.
\end{entry}

\secB*{Modifying the cube}
\begin{entry}
\item[flagTrim \texttt{[false]}] A flag to say whether to trim
  BLANK pixels from the edges of the cube -- these are typically
  pixels set to some particular value because they fall outside the
  imaged area, and trimming them can help speed up the execution.
\item[flagMW \texttt{[false]}] A flag to say whether to ignore
  channels contaminated by Milky Way (or other) emission -- the
  searching algorithms will not look at these channels.
\item[maxMW \texttt{[112]}] The maximum channel number that contains
  ``Milky Way'' emission.
\item[minMW \texttt{[75]}] The minimum channel number that contains
  ``Milky Way'' emission. Note that the range specified by
  \texttt{maxMW} and \texttt{minMW} is inclusive.
\item[flagBaseline \texttt{[false]}] A flag to say whether to remove
  the baseline from each spectrum in the cube for the purposes of
  reconstruction and detection.
\end{entry}

\secB*{Detection related}

\secC*{General detection}
\begin{entry}
\item[flagStatSec \texttt{[false]}] A flag indicating whether the
  statistics should be calculated on a subsection of the cube, rather
  than the full cube. Note that this only applies to the statistics
  used to determine the threshold, and not for other statistical
  calculations (such as those in the reconstruction phase).
\item[StatSec \texttt{[ [*,*,*] ]}] The subsection of the cube used
  for calculating statistics -- see \S\ref{sec-input} for details on
 the subsection format. Only used if \texttt{flagStatSec=true}.
\item[flagNegative \texttt{[false]}] A flag indicating that the
  features being searched for are negative. The cube will be inverted
  prior to searching.
\item[snrCut \texttt{[3.]}] The cut-off value for thresholding, in
  terms of number of $\sigma$ above the mean.
\item[threshold (no default assumed)] The actual value of the
  threshold. Normally the threshold is calculated from the cube's
  statistics, but the user can manually specify a value to be used
  that overrides the calculated value. If this is not specified, the
  calculated value is used. Also, when the FDR method is requested
  (see below), the value of the \texttt{threshold} parameter is
  ignored. 
\item[flagGrowth \texttt{[false]}] A flag indicating whether or not to
  grow the detected objects to a smaller threshold.
\item[growthCut \texttt{[2.]}] The smaller threshold using in growing
  detections. In units of $\sigma$ above the mean.
\item[beamSize \texttt{[10.]}] The size of the beam in pixels. If the
  header keywords BMAJ and BMIN are present, then these will be used
  to calculate the beam size, and this parameter will be ignored. 
\end{entry}

\secC*{\Atrous reconstruction}
\begin{entry}
\item [flagATrous \texttt{[false]}] A flag indicating whether or not to
  reconstruct the cube using the \atrous wavelet
  reconstruction. See \S\ref{sec-recon} for details.
\item[reconDim \texttt{[1]}] The number of dimensions to use in the
  reconstruction. 1 means reconstruct each spectrum separately, 2
  means each channel map is done separately, and 3 means do the whole
  cube in one go.
\item[scaleMin \texttt{[1]}] The minimum wavelet scale to be used in the
  reconstruction. A value of 1 means ``use all scales''.
\item[snrRecon \texttt{[4]}] The thresholding cutoff used in the
  reconstruction -- only wavelet coefficients this many $\sigma$ above
  the mean (or greater) are included in the reconstruction. 
\item[filterCode \texttt{[1]}] The code number of the filter to use in
  the reconstruction. The options are:
  \begin{itemize}
  \item \textbf{1:} B$_3$-spline filter: coefficients = 
    $(\frac{1}{16}, \frac{1}{4}, \frac{3}{8}, \frac{1}{4}, \frac{1}{16})$
  \item \textbf{2:} Triangle filter: coefficients = 
    $(\frac{1}{4}, \frac{1}{2}, \frac{1}{4})$
  \item \textbf{3:} Haar wavelet: coefficients = 
    $(0, \frac{1}{2}, \frac{1}{2})$
  \end{itemize}
\end{entry}

\secC*{Smoothing the cube}
\begin{entry}
\item [flagSmooth \texttt{[false]}] A flag indicating whether to
  smooth the cube. See \S\ref{sec-smoothing} for details. 
\item [smoothType \texttt{[spectral]}] The type of smoothing to be
done: either ``spectral'' (1D Hanning smoothing) or ``spatial'' (2D
Gaussian smoothing).
\item [hanningWidth \texttt{[5]}] The width of the Hanning smoothing
kernel. 
\item [kernMaj \texttt{[3]}] The full-width-half-maximum (FWHM) of the
major axis of the 2D Gaussian smoothing kernel.
\item [kernMin \texttt{[3]}] The FWHM of the minor axis of the 2D
Gaussian smoothing kernel.
\item [kernPA \texttt{[0]}] The position angle, in degrees,
anticlockwise from vertical (\ie usually East of North). 
\end{entry}

\secC*{FDR method}
\begin{entry}
\item[flagFDR \texttt{[false]}] A flag indicating whether or not to use
  the False Discovery Rate method in thresholding the pixels.
\item[alphaFDR \texttt{[0.01]}] The $\alpha$ parameter used in the FDR
analysis. The average number of false detections, as a fraction of the
total number, will be less than $\alpha$ (see \S\ref{sec-detection}).
\end{entry}

\secC*{Merging detections}
\begin{entry}
\item[minPix \texttt{[2]}] The minimum number of spatial pixels for a
  single detection to be counted.
\item[minChannels \texttt{[3]}] The minimum number of consecutive
  channels that must be present in a detection.
\item[flagAdjacent \texttt{[true]}] A flag indicating whether to use
  the ``adjacent pixel'' criterion to decide whether to merge
  objects. If not, the next two parameters are used to determine
  whether objects are within the necessary thresholds.
\item[threshSpatial \texttt{[3.]}] The maximum allowed minimum spatial
  separation (in pixels) between two detections for them to be merged
  into one. Only used if \texttt{flagAdjacent = false}.
\item[threshVelocity \texttt{[7.]}] The maximum allowed minimum channel
  separation between two detections for them to be merged into
  one. 
\end{entry}

\secC*{Other parameters}
\begin{entry}
\item[spectralMethod \texttt{[peak]}] This indicates which method is used
  to plot the output spectra: \texttt{peak} means plot the spectrum
  containing the detection's peak pixel; \texttt{sum} means sum the
  spectra of each detected spatial pixel, and correct for the beam
  size. Any other choice defaults to \texttt{peak}.
\item[spectralUnits \texttt{[km/s]}] The user can specify the units of
  the spectral axis. Assuming the WCS of the FITS file is valid, the
  spectral axis is transformed into velocity, and put into these units
  for all output and for calculations such as the integrated flux of a
  detection.
\item[pixelCentre \texttt{[centroid]}] Which of the three ways of
  expressing the ``centre'' of a detection (see \S\ref{sec-results}
  for a description of the options) to use for the X, Y, \& Z
  columns in the output list. Alternatives are: \texttt{centroid, peak,
  average}. 
\item[drawBorders \texttt{[true]}] A flag indicating whether borders
  are to be drawn around the detected objects in the moment maps
  included in the output (see for example Fig.~\ref{fig-spect}).
\item[drawBlankEdges \texttt{[true]}] A flag indicating whether to
 draw the dividing line between BLANK and non-BLANK pixels on the
 2-dimensional images (see for example Fig.~\ref{fig-moment}).
\item[verbose \texttt{[true]}] A flag indicating whether to print the
  progress of any computationally-intensive algorithms (\eg
  reconstruction, searching or merging algorithms) to the screen.
\end{entry}

