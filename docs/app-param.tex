% -----------------------------------------------------------------------
% app-param.tex: Section listing all the possible input parameters and
%                their defaults.
% -----------------------------------------------------------------------
% Copyright (C) 2006, Matthew Whiting, ATNF
%
% This program is free software; you can redistribute it and/or modify it
% under the terms of the GNU General Public License as published by the
% Free Software Foundation; either version 2 of the License, or (at your
% option) any later version.
%
% Duchamp is distributed in the hope that it will be useful, but WITHOUT
% ANY WARRANTY; without even the implied warranty of MERCHANTABILITY or
% FITNESS FOR A PARTICULAR PURPOSE.  See the GNU General Public License
% for more details.
%
% You should have received a copy of the GNU General Public License
% along with Duchamp; if not, write to the Free Software Foundation,
% Inc., 59 Temple Place, Suite 330, Boston, MA 02111-1307, USA
%
% Correspondence concerning Duchamp may be directed to:
%    Internet email: Matthew.Whiting [at] atnf.csiro.au
%    Postal address: Dr. Matthew Whiting
%                    Australia Telescope National Facility, CSIRO
%                    PO Box 76
%                    Epping NSW 1710
%                    AUSTRALIA
% -----------------------------------------------------------------------
\secA{Available parameters}
\label{app-param}

The full list of parameters that can be listed in the input file are
given here. If not listed, they take the default value given in
parentheses. Since the order of the parameters in the input file does
not matter, they are grouped here in logical sections.

\secB*{Input related}
\begin{Lentry}
\item[{ImageFile [no default]}] The filename of the
  data cube to be analysed. 
\item[{flagSubsection [false]}] A flag to indicate whether one
  wants a subsection of the requested image.
\item[{Subsection [ [*,*,*] ]}] The requested subsection
 -- see \S\ref{sec-input} for details on the subsection format.  If
 the full range of a dimension is required, use a \texttt{*} (thus the
 default is the full cube).
\item[{flagReconExists [false]}] A flag to indicate whether the
  reconstructed array has been saved by a previous run of \duchamp. If
  set true, the reconstructed array will be read from the file given
  by \texttt{reconFile}, rather than calculated directly.
\item[{reconFile [no default]}] The FITS file that contains the
  reconstructed array. If \texttt{flagReconExists} is true and this
  parameter is not defined, the default file that is looked for will
  be determined by the \atrous parameters (see \S\ref{sec-recon}).
\item[{flagSmoothExists [false]}] A flag to indicate whether the
  Hanning-smoothed array has been saved by a previous run of \duchamp. If
  set true, the smoothed array will be read from the file given
  by \texttt{smoothFile}, rather than calculated directly.
\item[{smoothFile [no default]}] The FITS file that has
  a previously smoothed array. If \texttt{flagSmoothExists}
  is true and this parameter is not defined, the default file that is
  looked for will be determined by the smoothing parameters (see
  \S\ref{sec-smoothing}).
\item[{usePrevious [false]}] A flag to indicate that \duchamp should
  read the list of objects from a previously-created log file, rather
  than doing the searching itself. The set of outputs will be created
  according to the flags in the following section.
\item[{objectList [no default]}] When \texttt{usePrevious=true}, this
  list is used to output individual spectral plots, as well as a
  postscript file for all spectral plots as given by
  \texttt{SpectraFile}. The filenames of the plots will be the same as
  \texttt{SpectraFile}, but with -XX at the end, where XX is the
  object number (\eg \texttt{duchamp-Spectra-07.ps}). The format of
  the parameter value should be a string listing individual objects or
  object ranges: \eg 1,2,4-7,8,14.
\end{Lentry}

\secB*{Output related}
\begin{Lentry}
\item[{OutFile [duchamp-\\Results.txt]}] The file containing the
  final list of detections. This also records the list of input
  parameters.
\item[{flagSeparateHeader [false]}] A flag to indicate that the header
  information that would normally be printed at the start of the
  results file (containing information on the parameters, image
  statistics and number of detections) should instead be written to a
  separate file.
\item[{HeaderFile [duchamp-\\Results.hdr]}] The file to which the
  header information should be written when
  \texttt{flagSeparateHeader=true}.
\item[{flagPlotSpectra [true]}] Whether to produce a postscript file
  containing spectra of all detected objects.
\item[{SpectraFile [duchamp-\\Spectra.ps]}] The postscript file
  containing the resulting integrated spectra and images of the
  detections.
\item[{flagTextSpectra [false]}] A flag to say whether the spectra
  should be saved in text form in a single file. See below for a
  description. 
\item[{spectraTextFile [duchamp-\\Spectra.txt]}] The file containing
  the spectra of each object in ascii format. This file will have a
  column showing the spectral coordinates, and one column for each of
  the detected objects, showing the flux values as plotted in the
  graphical output of \texttt{spectraFile}.
\item[{flagLog [false]}] A flag to indicate whether the
  details of intermediate detections should be logged.
\item[{LogFile [duchamp-\\Logfile.txt]}] The file in which
  intermediate detections are logged. These are detections that have
  not been merged. This is primarily for use in debugging and
  diagnostic purposes: normal use of the program will probably not
  require it.
\item[{flagOutputMomentMap [false]}] A flag to say whether or not to save a
  FITS file containing the moment-0 map. 
\item[fileOutputMomentMap{ [see text]}] The file to which the moment-0 array is
  written. If left blank (the default), the naming scheme detailed in
  Section~\ref{sec-maskOut} is used.
\item[{flagOutputMask [false]}] A flag to say whether or not to save a
  FITS file containing a mask array, with values 1 where there is a
  detected object and 0 elsewhere. 
\item[fileOutputMask{ [see text]}] The file to which the mask array is
  written. If left blank (the default), the naming scheme detailed in
  Section~\ref{sec-maskOut} is used.
\item[{flagMaskWithObjectNum [false]}] If this flag is true, the
  detected pixels in the mask image have the corresponding object ID
  as their value. If false, they have the value 1. All non-detected
  pixels have the value 0.
\item[{flagOutputRecon [false]}] A flag to say whether or not
  to save the reconstructed cube as a FITS file. 
\item[fileOutputRecon{ [see text]}] The file to which the reconstructed array
  is written. If left blank (the default), the naming scheme detailed
  in Section~\ref{sec-reconIO} is used.
\item[{flagOutputResid [false]}] As for
  \texttt{flagOutputRecon}, but for the residual array -- the
  difference between the original cube and the reconstructed cube. 
\item[fileOutputResid{ [see text]}] The file to which the residual array
  is written. If left blank (the default), the naming scheme detailed
  in Section~\ref{sec-reconIO} is used.
\item[{flagOutputSmooth [false]}] A flag to say whether or not
  to save the smoothed cube as a FITS file. 
\item[fileOutputSmooth{ [see text]}] The file to which the smoothed array
  is written. If left blank (the default), the naming scheme detailed
  in Section~\ref{sec-reconIO} is used.
\item[{flagVOT [false]}] A flag to say whether to create a
  VOTable file with the detection information. This will be an XML
  file in the Virtual Observatory VOTable format.
\item[{votFile [duchamp-\\Results.xml]}] The VOTable file with
  the list of final detections. Some input parameters are also
  recorded. 
\item[{flagKarma [false]}] A flag to say whether to create a
  Karma annotation file corresponding to the information in
  \texttt{outfile}. This can be used as an overlay in Karma
  programs such as \texttt{kvis}.
\item[{karmaFile [duchamp-\\Results.ann]}] The Karma annotation
  file showing the list of final detections.
\item[{annotationType [borders]}] Which type of annotation plot to
  use. Specifying ``borders'' gives an outline around the detected
  spatial pixels, while ``circles'' gives a circle centred on the
  centre of the object with radius large enough to encompass all
  spatial pixels.
\item[{flagMaps [true]}] A flag to say whether to save
  postscript files showing the 0th moment map of the whole cube
  (parameter \texttt{momentMap}) and the detection image
  (\texttt{detectionMap}).
\item[{momentMap [duchamp-\\MomentMap.ps]}] A postscript file
  containing a map of the 0th moment of the detected sources, as well
  as pixel and WCS coordinates.
\item[{detectionMap [duchamp-\\DetectionMap.ps]}] A postscript
  file with a map showing each of the detected objects, coloured in
  greyscale by the number of detected channels in each spatial
  pixel. Also shows pixel and WCS coordinates.
\item[{flagXOutput [true]}] A flag to say whether to display a
  0th moment map in a PGPlot X-window. This will be in addition to any
  that are saved to a file. This parameter can be overridden by the
  use of the \texttt{-x} command-line option, which disables the
  X-windows output.
\item[{newFluxUnits [no default]}] Flux units that the pixel values
  should be converted into. These should be directly compatible with
  the existing units, given by the BUNIT keyword.
\item[{precFlux [3]}] The desired precision (\ie number of decimal
  places) for flux values given in the output files and tables.
\item[{precVel [3]}] The desired precision (\ie number of decimal
  places) for velocity/frequency values given in the output files and
  tables.
\item[{precSNR [2]}] The desired precision (\ie number of decimal
  places) for the peak SNR value given in the output files and tables.
\end{Lentry}

\secB*{Modifying the cube}
\begin{Lentry}
\item[{flagTrim [false]}] A flag to say whether to trim
  BLANK pixels from the edges of the cube -- these are typically
  pixels set to some particular value because they fall outside the
  imaged area, and trimming them can help speed up the execution.
\item[{flagMW [false]}] A flag to say whether to ignore
  channels contaminated by Milky Way (or other) emission -- the
  searching algorithms will not look at these channels.
\item[{maxMW [112]}] The maximum channel number that contains
  ``Milky Way'' emission. This is the channel number in the original
  cube, before any subsection is applied.
\item[{minMW [75]}] The minimum channel number that contains ``Milky
  Way'' emission. This is the channel number in the original cube,
  before any subsection is applied.  Note that the range specified by
  \texttt{maxMW} and \texttt{minMW} is inclusive.
\item[{flagBaseline [false]}] A flag to say whether to remove
  the baseline from each spectrum in the cube for the purposes of
  reconstruction and detection.
\end{Lentry}

\secB*{Detection related}

\secC*{General detection}
\begin{Lentry}
\item[{searchType [spatial]}] How the searches are done. A value of
  ``spatial'' means each 2D channel map is searched, whereas
  ``spectral'' means each 1D spectrum is searched.
\item[{flagStatSec [false]}] A flag indicating whether the
  statistics should be calculated on a subsection of the cube, rather
  than the full cube. Note that this only applies to the statistics
  used to determine the threshold, and not for other statistical
  calculations (such as those in the reconstruction phase).
\item[{StatSec [ [*,*,*] ]}] The subsection of the cube used
  for calculating statistics -- see \S\ref{sec-input} for details on
 the subsection format. Only used if \texttt{flagStatSec=true}.
\item[{flagRobustStats [true]}] A flag indicating whether to use the
  robust statistics (median and MADFM) to estimate the noise
  parameters, rather than the mean and rms. See \S\ref{sec-stats} for
  details. 
\item[{flagNegative [false]}] A flag indicating that the
  features of interest are negative. The cube is inverted prior to
  searching.
\item[{snrCut [3.]}] The threshold, in multiples of $\sigma$ above
  the mean.
\item[{threshold [no default]}] The actual value of the
  threshold. Normally the threshold is calculated from the cube's
  statistics, but the user can manually specify a value to be used
  that overrides the calculated value. If this is not specified, the
  calculated value is used, but this value will take precedence over
  other means of calculating the threshold (\ie via \texttt{snrCut} or
  the FDR method).
\item[{flagGrowth [false]}] A flag indicating whether or not to
  grow the detected objects to a smaller threshold.
\item[{growthCut [3.]}] The smaller threshold using in growing
  detections. In units of $\sigma$ above the mean.
\item[{growthThreshold [no default]}] Alternatively, the threshold to
  which detections are grown can be specified in flux units (in the
  same manner as the \texttt{threshold} parameter). When the
  \texttt{threshold} parameter is given, this option \textbf{must} be
  used instead of \texttt{growthCut}.
\item[{beamFWHM [0.]}] The full-width at half maximum of the beam, in
  pixels.  If the header keywords BMAJ and BMIN are present, then
  these will be used to calculate the beam area, and this parameter
  will be ignored. This will take precedence over \texttt{beamArea}
  (but is ignored if not specified).
\item[{beamArea [0.]}] The \textbf{area} of the beam in pixels (\ie
  how many pixel does the beam cover?). As above, if the header
  keywords BMAJ and BMIN are present, then these will be used to
  calculate the beam area, and this parameter will be ignored. 
\end{Lentry}

\secC*{\Atrous reconstruction}
\begin{Lentry}
\item[{flagATrous [false]}] A flag indicating whether or not to
  reconstruct the cube using the \atrous wavelet
  reconstruction. See \S\ref{sec-recon} for details.
\item[{reconDim [1]}] The number of dimensions to use in the
  reconstruction. 1 means reconstruct each spectrum separately, 2
  means each channel map is done separately, and 3 means do the whole
  cube in one go.
\item[{scaleMin [1]}] The minimum wavelet scale to be used in the
  reconstruction. A value of 1 means ``use all scales''.
\item[{scaleMax [0]}] The maximum wavelet scale to be used in the
  reconstruction. If the value is $\le0$ then the maximum scale is
  calculated from the size of the input array. Similarly, if the value
  given is larger than this calculated value, the calculated value is
  used instead.
\item[{snrRecon [4]}] The thresholding cutoff used in the
  reconstruction -- only wavelet coefficients this many $\sigma$ above
  the mean (or greater) are included in the reconstruction. 
\item[{filterCode [1]}] The code number of the filter to use in
  the reconstruction. The options are:
  \begin{itemize}
  \item \textbf{1:} B$_3$-spline filter: coefficients = 
    $(\frac{1}{16}, \frac{1}{4}, \frac{3}{8}, \frac{1}{4}, \frac{1}{16})$
  \item \textbf{2:} Triangle filter: coefficients = 
    $(\frac{1}{4}, \frac{1}{2}, \frac{1}{4})$
  \item \textbf{3:} Haar wavelet: coefficients = 
    $(0, \frac{1}{2}, \frac{1}{2})$
  \end{itemize}
\end{Lentry}

\secC*{Smoothing the cube}
\begin{Lentry}
\item[{flagSmooth [false]}] A flag indicating whether to
  smooth the cube. See \S\ref{sec-smoothing} for details. 
\item[{smoothType [spectral]}] The smoothing method used: either
  ``spectral'' (with a 1D Hanning filter) or ``spatial'' (with a 2D
  Gaussian filter).  
\item[{hanningWidth [5]}] The width of the Hanning smoothing
  kernel.
\item[{kernMaj [3]}] The full-width-half-maximum (FWHM) of the
  2D Gaussian smoothing kernel's major axis.
\item[{kernMin [3]}] The FWHM of the 2D Gaussian smoothing kernel's
  minor axis.
\item[{kernPA [0]}] The position angle, in degrees,
  anticlockwise from vertical (\ie usually East of North). 
\end{Lentry}

\secC*{FDR method}
\begin{Lentry}
\item[{flagFDR [false]}] A flag indicating whether or not to use
  the False Discovery Rate method in thresholding the pixels.
\item[{alphaFDR [0.01]}] The $\alpha$ parameter used in the FDR
  analysis. The average number of false detections, as a fraction of
  the total number, will be less than $\alpha$ (see
  \S\ref{sec-detection}).
\item[{FDRnumCorChan [2]}] The number of neighbouring spectral
  channels that are assumed to be correlated. This is needed by the
  FDR algorithm to calculate the normalisation constant $c_N$ (see
  \S\ref{sec-detection}). 
\end{Lentry}

\secC*{Merging detections}
\begin{Lentry}
\item[{minPix [2]}] The minimum number of spatial pixels for a
  single detection to be counted.
\item[{minChannels [3]}] At least one contiguous set of this many
  channels must be present in the detection for it to be accepted.
\item[{minVoxels [4]}] A detection must have at least this many voxels
  in it to be counted.
\item[{flagRejectBeforeMerge [false]}] A flag indicating whether to
  reject sources that fail to meet the \texttt{minPix} or
  \texttt{minChannels} criteria \textbf{before} the merging
  stage. Default behaviour is to do the rejection last.
\item[{flagTwoStageMerging [true]}] A flag indicating whether to do an
  initial merge of newly-detected sources into the source list as they
  are found. If \texttt{false}, new sources are simply added to the
  end of the list for later merging.
\item[{flagAdjacent [true]}] A flag indicating whether to use
  the ``adjacent pixel'' criterion to decide whether to merge
  objects. If not, the next two parameters are used to determine
  whether objects are within the necessary thresholds.
\item[{threshSpatial [3.]}] The maximum allowed minimum spatial
  separation (in pixels) between two detections for them to be merged
  into one. Only used if \texttt{flagAdjacent = false}.
\item[{threshVelocity [7.]}] The maximum allowed minimum channel
  separation between two detections for them to be merged into
  one. 
\end{Lentry}

\secC*{Other parameters}
\begin{Lentry}
\item[{spectralMethod [peak]}] This indicates which method is used
  to plot the output spectra: \texttt{peak} means plot the spectrum
  containing the detection's peak pixel; \texttt{sum} means sum the
  spectra of each detected spatial pixel, and correct for the beam
  size. Any other choice defaults to \texttt{peak}.
\item[{spectralUnits [km/s]}] The user can specify the units of
  the spectral axis. Assuming the WCS of the FITS file is valid, the
  spectral axis is transformed into velocity, and put into these units
  for all output and for calculations such as the integrated flux of a
  detection.
\item[{pixelCentre [centroid]}] Which of the three ways of
  expressing the ``centre'' of a detection (see \S\ref{sec-results}
  for a description of the options) to use for the X, Y, \& Z
  columns in the output list. Alternatives are: \texttt{centroid, peak,
  average}.
\item[{sortingParam [vel]}] The parameter on which to sort the output
  list of detected objects. Options are: x-value, y-value, z-value,
  ra, dec, vel, w50, iflux,  pflux (integrated and peak flux
  respectively), or snr. 
\item[{drawBorders [true]}] A flag indicating whether to draw
  borders around the detected objects in the moment maps included in
  the output (see for example Fig.~\ref{fig-spect}).
\item[{drawBlankEdges [true]}] A flag indicating whether to
  draw the dividing line between BLANK and non-BLANK pixels on the
  2D images (see for example Fig.~\ref{fig-moment}).
\item[{verbose [true]}] A flag indicating whether to print the
  progress of any computationally intensive algorithms (\eg
  reconstruction, searching or merging algorithms) to the screen.
\end{Lentry}

